\section{Grundlagen}
In diesem Kapitel werden die grundlegenden Begriffe erläutert, die für das Verständnis der weiterführenden Kapitel notwendig sind.

\subsection{Dokumenten-Management-Systeme}

Um ein Dokumenten-Management-System (\gls{DMS})  zu erläutern muss sich zuerst mit dem Begriff des \textbf{"`Dokuments"'} auseinander gesetzt werden.
In \cite{DMS08} S. 2 wird ein Dokument durch folgende Punkte definiert:

\begin{itemize}
\item Ein Dokument fasst inhaltlich zusammengehörende Informationen strukturiert zusammen, die nicht ohne erheblichen Bedeutungsverlust weiter unterteilt werden können. 
\item Die Gesamtheit der Information ist für einen gewissen Zeitraum zu erhalten.
\item Dokumente dienen oft dem Nachweis von Tatsachen.
\item Ein Dokument ist als Einheit ablegbar (speicherbar) und/oder versendbar und/oder wahrnehmbar (sehen, hören, fühlen).
\item Das Dokument ist eigentlich der Träger, der die Informationen speichert, egal ob das Dokument ein Stück Paper, eine Datei auf einem Rechner, ein Videoband oder eine Tontafel etc. ist. Dies bedeutet auch, dass es keine Bindung an Papier oder ein geschriebenes Wort gibt.
\end{itemize}

Desweiteren gibt es eine Differenzierung in zwei Definitionen:

\begin{quote}"`Als \textbf{Dokument im konventionellen Sinne} werden Dokumente bezeichnet, die als körperliches Dokumente (z. B. Papier) vorliegen, ursprünglich als körperliches Dokument vorlagen oder für die Publizierung auf einem körperlichen Medium vorgesehen sind.

Die Begrifflichkeit des \textbf{Dokuments im weiteren Sinne} erweitert den Begriff des Dokuments um semantisch zusammengehörende Informationsbestände , die für die Publikation in nicht-körperlichen Medien, z. B. Webseiten, Radio, Fernsehen o. ä. vorgesehen sind. Derartige Dokumente werden oft dynamisch gestaltet und zusammengestellt."' \begin{flushright}\cite{DMS08} S. 2\end{flushright}\end{quote}

Unter \textbf{Dokumenten-Management} werden primär die Verwaltungsfunktionen Erfassung, Bearbeitung, Verwaltung und Speicherung von Dokumenten verstanden. \cite{DMS08} S. 344.

Darunter fallen laut \cite{DMS08} S. 3 folgende Punkte:

\begin{itemize}
\item Kennzeichnung und Beschreibung von Dokumenten (auch Metadaten des Dokuments genannt) 
\item Fortschreibung, Versionierung und Historienverwaltung von Dokumenten
\item Ablage und Archivierung von Dokumenten
\item Verteilung und Umlauf von Dokumenten
\item Suche nach Dokumenten bzw. Dokumenteninhalten
\item Schutz der Dokumente vor Verfälschung, Missbrauch und Vernichtung
\item Langfristiger Zugriff auf die Dokumente und Lesbarkeit der Dokumente
\item Lebenslauf und Vernichtung von Dokumenten
\item Regelung von Verantwortlichkeiten für Inhalt und Verwaltung von Dokumenten
\end{itemize}

Der Begriff \textbf{"`Dokumenten-Management-System"'} muss auch in zwei verschiedene Sichtweisen differenziert werden:
\begin{quote}"`Bei \textbf{Dokumenten-Management-Systemen im engeren Sinne} geht es um die Logik der Verwaltung von Dokumenten, deren Status, Struktur, Lebenzyklus und Inhalt. Dokumente werden beschrieben, klassifiziert und in einer bestimmten logischen Struktur eingeordnet, damit sie einfach wieder gefunden werden können. Dokumente entstehen, werden verändert und (irgendwann) vernichtet.

Den \textbf{Dokumenten-Management-Systemen im weiteren Sinne} ordnet man auch noch weitere Funktionalitäten zu, wie z. B. Schrifterkennung, automatische Indizierung, [...], Publizierung. Hier lassen sich die Grenzen nicht mehr genau bestimmten!"' \begin{flushright}\cite{DMS08} S. 5\end{flushright}\end{quote}

\subsection{Content-Management-Systeme}
Bei einem Content-Management-System (\gls{CMS})  steht nicht mehr das eigentliche Dokument im Vordergrund, sondern vielmehr der enthaltene Informationsgehalt des Dokuments.
Der Unterschied zwischen einem DMS und einem CMS besteht laut \cite{DMS08} S. 114 im/in Folgenden/m:

\begin{quote}"`Abgrenzend zum Dokumenten-Management handelt es sich beim Content-Management nicht vordergründig um die Verwaltung von Dokumenten, sondern um die Verwaltung von Informationseinheiten, die miteinander verknüpft sein können. [...] Je nach ausprägung kann nun ein konkretes System als Dokumenten-Management-System mit Content-Management-Funktionen definiert werden und umgekehrt. [...]
Der Ansatz des Content-Management unterscheidet sich vom "`klassischen"' Dokumenten-Mangement vor allem in Bezug auf die betrachteten Objekte: Ein DMS hat als kleinestes Objekt der Betrachtung eines einzelnen Dokument. [...] Content-Management ist auf logische Informationseinheiten ausgerichtet. Es ist z.B. das Ziel des Content-Managements, Inhalte, die auf mehrere Quellen verteilt sind, neue zusammenzustellen und daraus z.B. ein neues Dokument zu generieren."'
\begin{flushright}\cite{DMS08} S. 114f\end{flushright}\end{quote}

Die folgende Abbildung soll den (charakteristischen) Unterschied zwischen CMS-Systemen und DMS-Systemen verdeutlichen.

\begin{center}
Bild \cite{DMS08} S. 115
\end{center}

Wie zuvor beschrieben ist die Sichtweise eines DMS nur auf die einzelnen Dokumente beschränkt, während ein CMS einzelne Elemente / Informationen aus den Dokumenten extrahieren und daraus ggf. ein neues Dokument generieren kann. Die Sichtweise des CMS wird durch das gestrichelte Polygon dargestellt, welches hier dokumentenübergreifend abgebildet ist.

Der (theoretische/beabsichtigte) Zweck, weshalb ein CMS-System eingesetzt wird, ist laut Oracle folgendermaßen definiert:

\begin{quote}"`The key to a successful content management implementation is unlocking the value of content by making it as easy as possible for it to be consumed. This means that any piece of content must be available to any consumer, no matter what their method of access."'
\begin{flushright}\cite{UCM07} S. 12\end{flushright}\end{quote}

Ein CMS soll die Informationen jedes/jedwedem (Inhalts) extrahieren/aufnehmen und jedes Einzelteil / Element dieser Information den Benutzern zugänglich machen, unabhängig von der Art des Zugriffs.
Dieses Konzept soll in Abbildung \ref{ucm-a2a} verdeutlicht werden.

\begin{figure}[ht]
	\centering
	   \fbox{\includegraphics[width=0.95\textwidth]{bilder/ucm.png}}
		\caption["`any-to-any"' Content-Management Konzept]{"`any-to-any"' Content-Management Konzept\protect\footnote}
		\label{ucm-a2a}
\end{figure}
\footnotetext{Quelle: \cite{UCM07} S. 12}

Das CMS steht hier in der Mitte der Abbildung als Medium zwischen den verschiedenen Inhalten, eingestellt von den \textit{Contributors} (links), und den Anwendern, die auf transformierte Versionen der Inhalte durch unterschiedliche Arten zugreifen (rechts).


\subsection{Enterprise-Content-Management-Systeme - optional}
In diesem Zusammenhang / Kontext sei auch der Begriff Enterprise-Content-Management (\gls{ECM}) genannt.
Laut der "`Association for Information and Image Management"' (\gls{AIIM}\footnote{Die AIIM ist eine Gesellschaft von internationalen Herstellern und Anwendern von Informations- und Dokumenten-Mangement-Systemen}), welche sich mit umfasst dieser Begriff die Verwaltungfunktionen von Unternehmensinformationen in unterschiedlichen Dokumentformaten.\footnote{Quelle: \url{http://www.aiim.org/What-is-ECM-Enterprise-Content-Management.aspx}}
Diese Funktionen werden laut \cite{DMS08} S. 116 durch verschiedene "`Systeme wie Dokumenten-Management, Groupware, Workflow, Input- und Output-Management, (Web-)Content-management, Archivierung, Records-Management und andere"' bereitsgestellt.


\subsection{Universal-Content-Management-Systeme}



























\begin{center}

\begin{itemize}
\item ads - ADS Benutzer
\item cronjobs
\item evtl Oracle DB
\item false+-true+-
\item Farbraum
\item Metadaten allg
\end{itemize}

\end{center}

\subsection{Nagios}
Nagios dient zum Überwachen von Hosts und deren Services in komplexen Infrastrukturen(Host und Services erklären?) und wurde von dem Amerikaner Ethan Galstad seit 1999\footnote{Quelle: \url{http://www.netsaint.org/changelog.php}} - damals unter der Vorgängerversion NetSaint - entwickelt und bis heute gepflegt.
Galstad gründete aufgrund der vielfältigen(ansturmmäßig) und positiven Resonanz am 9. November 2007 die "`Nagios Enterprises LLC"', welche Nagios als kommerzielle Dienstleistung anbietet.
Die Software selbst blieb weiterhin unter der freien Lizenz "`GNU General Public License version 2"'\footnote{Quelle: \url{http://www.gnu.org/licenses/old-licenses/gpl-2.0.txt}} verfügbar.
Diese erlaubt Einblick in den Programmcode und Modifizieren der Anwendung nach eigenen Vorstellungen.

Nagios erfreut sich hoher Beliebtheit aufgrund der (bereits vorhandenen [macht kein sinn hohe beliebtheit aufgrund der großen community?]) großen Community, die Tipps, Ratschläge und auch eigene Nagios-Plugins kostenlos anbietet.
Außerdem können selbst mit geringen Programmierkenntnissen zusätzliche Skripte zur Überwachung geschrieben werden, wenn ein spezieller Anwendungsfall dies erfordert.
Warum wird Nagios engesetzt und nicht was anders -> andere kandidaten finden openview, big brother? -> das buch vom jäger verwenden!

OpenSource halt, recht einfach plugins programmierbar (auf plugin kapitel verweisen)

\newpage
\subsection{Aufbau / Architektur}


Barth schreibt über Nagios: \begin{quote}"`Die große Stärke von Nagios - auch im Vergleich zu anderen Netzwerküberwachungstools - liegt in seinem modularen Aufbau: Der Nagios-Kern enthält keinen einzigen Test, stattdessen verwendet er für Service- und Host-Checks externe Programme, die als \textit{Plugins} bezeichnet werden."'
\begin{flushright}\cite{Barth08} S. 25\end{flushright}\end{quote} 

Dieser "`Kern"' beinhaltet das komplette Benachrichtigungssystem mit Kontaktadressen und Benachrichtigungsvorgaben (Zeit, Art, zusätzliche Kriterien), die Hosts- und Servicedefinitionen inklusive deren Gruppierungen und schließlich das Webinterface.
Siehe hierfür auch Abbildung \ref{nagios-proc}.

\begin{figure}[ht]
	\centering
	   \fbox{\includegraphics[width=0.9\textwidth]{bilder/nagios-proc.png}}
		\caption[Logische Nagios Struktur]{Logische Nagios Struktur\protect\footnote}
		\label{nagios-proc}
\end{figure}
\footnotetext{Quelle: \url{http://www.nagioswiki.org/w/images/8/81/Plugins.png}}

Damit Nagios die gewünschten Server überwachen kann, müssen sie der Anwendung zuerst bekannt gemacht werden.
Dies wird über das Anlegen einer Konfigurationsdatei mit einem Host-Objekt erreicht.
Dabei richtet sich die Definition des Host-Objektes nach dem Schema, welches für alle Objektedefinitionen (Services, Kontakt, Gruppen, Kommandos etc.) gilt:

\begin{lstlisting}[captionpos=b, caption=Nagiosschema für Objektdefinitionen, label=schmeaobj, breaklines = false]
define object-type {
	parameter value
	parameter value
	...
}
\end{lstlisting}


Eine gültige Host-Definition muss mindestens folgende Elemente besitzten:

\begin{lstlisting}[captionpos=b, caption=Definition eines Hostobjektes, label=hostobj, breaklines = true, language=sh]
define host{
	host_name		example.kit.edu #Referenzname des Servers
	alias			Oracle UCM Server #Weitere Bezeichnung
	address			example.kit.edu #FQDN des Rechners
	max_check_attempts	4 #Anzahl der Checks zum Wechsel von Hard- zu Soft-State
	check_period		24x7 #Zeitraum der aktiven Checks
	contact_groups		UCM-admins #Zu alarmierende Benutzergruppe
	notification_interval	120 #Minuten bis Alarmierung wiederholt wird
	notification_period	24x7 #Zeitraum der Benachrichtigungen
}
\end{lstlisting}

Gewöhnlich / In der Praxis werden öfters verwendete Attribute wie die Kontaktgruppe oder der Zeitraum für die aktiven Checks durch Verwendung eines übergeordneten Host-Objektes nach unten vererbt.
Dadurch müssen nurnoch die spezifischen Inforamtionen und der Name des übergeordneten Host-Objektes eingetragen werden.
\begin{lstlisting}[captionpos=b, caption=Verkürzte Definition eines Hostobjektes, label=vhostobj, breaklines = true, language=sh]
define host {
        use             windows-server #Oberklasse dieses Host-Objektes
        host_name       example.kit.edu
        alias           Oracle UCM Server
        address         example.kit.edu
}
\end{lstlisting}

Mit dieser Hostdefinition wird der Rechner im Webinterface von Nagios bereits angezeigt:

\begin{figure}[ht]
	\centering
	   \fbox{\includegraphics[width=0.95\textwidth]{bilder/example-ping.png}}
		\caption{Anzeige des Servers im Webinterface von Nagios}
		\label{check-swap}
\end{figure}

Jedoch wird nur die Erreichbarkeit über das Netzwerk mit einem Ping überwacht.
%Umfangreichere / Komplexere / Weitergehende / 
Um andere Dienste zu überwachen müssen die gewünschten Plugins explizit aus dem Nagios Repertoire dem zu überwachendem Computer mit einem ähnlichen Schema zugeteilt werden.

Eine beispielhafte Servicedefinition für die Überwachung des Webservers auf dem Host \textit{example.kit.edu} wird in Codelisting \ref{servdef} gezeigt.

\begin{lstlisting}[captionpos=b, caption=Verkürzte Definition eines Hostobjektes, label=servdef, breaklines = true, language=sh]
define service{
        use                     generic-service #Oberklasse dieses Service-Objektes
        host_name               example.kit.edu
        service_description     HTTP Server #Bezeichnung des Checks
        check_command           check_http #Angabe des NagiosPlugins (hier ohne Parameter)
        }

\end{lstlisting}

Die Plugins werden durch die Servicedefinitionen mit den jeweiligen Hosts verbunden und durch das Attribut \textit{check\_command} mit ggf. veränderten Parametern durch Nagios aufgerufen.

Nagios wird in einem festlegbarem / veränderbarem Zeitintervall alle vom Benutzer definierten Host- und Servicechecks überprüfen und die Ergebnisse der entsprechenden Plugins verarbeiten / auswerten.

Weiterhin beschreibt Barth die Plugins folgendermaßen:
\begin{quote}"`Jedes Plugin, das bei Host- und Service-Checks zum Einsatz kommt, ist ein eigenes, selbständiges Programm, das sich auch unabhängig von Nagios benutzen lässt."' \begin{flushright}\cite{Barth08} S. 105\end{flushright}\end{quote} 

Daher lassen sich die Parameter eines Plugins folgendermaßen überprüfen:

\begin{figure}[ht]
	\centering
	   \fbox{\includegraphics[width=0.85\textwidth]{bilder/check-swap-white.png}}
		\caption{Beispielhafte manuelle Ausführung eines Servicechecks}
		\label{check-swap}
\end{figure}

Die Ausgabe des Plugins gibt den Zustand des Services an; in diesem Fall wird kein Schwellwert überschritten, daher die Meldung \pictext{SWAP OK}.
Dieses Plugin liefert noch zusätzliche Performance-Informationen, die mit externen Programmen ausgewertet, gespeichert und visualisiert werden können.
Standardmäßig werden die Performancedaten von der normalen Ausgabe mit einem \pictext{|} getrennt.
Jedoch können auch Werte aus der normalen Textausgabe verwendet werden, so dass in diesem Beispiel keine Berechnung des Prozentsatzes notwendig wäre.

Um diesen Service mit den angegebenen Schwellwerten von Nagios überwachenzulassen, muss folgende Servicedefinition in die Konfigurationsdatei eingetragen werden:

\begin{lstlisting}[captionpos=b, caption=Beispielhafte (Definition) eines Servicechecks, label=servdef, breaklines = true, language=bash]
# Define a service to check the swap disk space
# on the local machine.  Warning if =< 20% free, 
# critical if =< 10% free space on swap partition.

define service{
 use         generic-service 
 host_name   example.kit.edu 	
 service_description  Swap Disk Space
 check_command   check_swap!-w 20% -c 10%	# Angabe des zu verwendenden Plugins mit WARNING (respektiv) CRITICAL Schwellwertparameter
}
\end{lstlisting}



Dabei wird in vier verschiedene Rückgabewerte / Antworten der Plugins unterschieden:

%\setlength{\tabcolsep}{50pt}
\begin{table}[!htbp]
\centering

\begin{tabular}{l l p{7cm}}
%\hline
\textbf{Status} \hspace{2 mm} & \textbf{Bezeichnung} \hspace{3 mm} & \textbf{Beschreibung}\\
\hline
%\textit{features} & complete\tnote{1} & complete\tnote{1} \\
%\hline
0 & OK & Alles in Ordnung \\
%\hline
1 & WARNING & Die Warnschwelle wurde überschritten, die kritische Schwelle ist aber noch nicht erreicht.\\
%\hline
2 & CRITICAL & Entweder wurde die kritische Schwelle überschritten oder das Plugin hat den Test nach einem Timeout  abgebrochen. \\
%\hline
3 & UNKNOWN &  Innerhalb des Plugins trat ein Fehler auf (zum Beispiel weil falsche Parameter verwendet wurden)\\
%\hline
\bottomrule
\end{tabular}
\caption[Rückgabewerte für Nagios Plugins]{Rückgabewerte für Nagios Plugins\protect\footnote} %\protect\footnote
%\end{twoparttable}
\end{table}
\footnotetext{Quelle: \cite{Barth08} S. 105f}

Anhand dieser Werte wertet Nagios gezielt den Status des jeweiligen Objektes (Host oder Service) aus.
Weiterhin gibt es weiche (Soft States) und harte Zustände (Hard States):

\begin{figure}[ht]
	\centering
	   \fbox{\includegraphics[width=0.85\textwidth]{bilder/hs-states.png}}
		\caption[Beispiel für den zeitlichen Verlauf durch vers. Zustände]{Beispiel für den zeitlichen Verlauf durch vers. Zustände\protect\footnote}
		\label{hs-states}
\end{figure}
\footnotetext{Quelle: \cite{Barth08} S. 95}
Ausgehend von einem OK Zustand wird in diesem Beispiel jede fünf Minuten periodisch überprüft, ob sich der Status des überwachten Objektes verändert hat.                                                                                                                                                                                         Nach zehn Minuten wird ein Umschwenken / Änderung des Zustandes durch das jeweilige Plugin gemeldet.
Hier im Beispiel wechselt der Zustand nach CRITICAL, zunächst allerdings als Soft State
Daher wird durch Nagios noch keine Benachrichtigung versendet, da es sich um eine Falschmeldung, auch False Positive genannt, handeln kann.
Aufgrund einer kurzweiligen / kurzfristigen (besseres Wort? peak mäßig) hohen Auslastung des Netzwerkes oder um ein kurzzeitiges Problem, welches sich von alleine wieder normalisiert. (Bspw. Prozessorauslastung)

Um einen False Positive auszuschliessen, wird der im Soft State befindliche Service bzw. Host mit einer höheren Frequenz überprüft.
Sollten diese Überprüfungen den vorherigen Zustand bestätigen, verfestigt sich der aktuelle Zustand, man spricht nun von einem Hard State / wechselt der Zustand in den Hard State.
Erst in diesem Moment werden die entsprechenden Kontaktpersonen über den in diesem Beispiel kritischen Zustand benachrichtigt.
Sollte sich der Zustand wieder in den Normalzustand begeben und dieser Zustandsübergang wird von dem (von Nagios ausgeführten) Plugin festgestellt, wird dies an den Nagios Server gemeldet.

Ein Übergang zu dem OK Status wird sofort als Hard State festgelegt / festgehalten / festgesetzt / und führt dadurch zur sofortigen Benachrichtigung durch Nagios.

\begin{itemize}
\item Betroffene OSI Schichten auflisten und erklären
\item Wie werden die Info von Nagios gesammelt und wie gespeichert -> FlapDetection
\item FLapping \footnote{\url{http://nagios.sourceforge.net/docs/2\_0/flapping.html}}
\item Benachrichtigung durch email oder sms sogar per Telefon geht usw.
\item Hierarchie \url{http://nagios.sourceforge.net/docs/3_0/networkreachability.html}
\end{itemize}

\subsection{Überprüfungsmethoden}
%Service Checks und deren / ihre Realisierung / Ausführung / (
Dienste, die im Netzwerk zur Verfügung stehen (Netzwerkdienste), wie ein Web- oder \gls{FTP}-Server , lassen sich einfach / simpel direkt über das Netz auf ihren Zustand (hin) überprüfen /  testen.
Hierfür muss dem entsprechende Plugin lediglich die Netzwerkadresse mitgeteilt werden, siehe Abbildung \ref{check-http} als beispielhafte Überprüfung eines Webservers.

\begin{figure}[ht]  
	\centering
	   \fbox{\includegraphics[width=0.85\textwidth]{bilder/check-http.png}}
		\caption{Beispielhafte manuelle Ausführung eines netzwerkbasierenden Servicechecks / HTTP Server Check}
		\label{check-http}
\end{figure}

(Bitte beachten, dass das Plugin immernoch auf dem Nagios Server ausgeführt wird / sich immernoch auf dem Nagios Server befindet)

Dienste, die sich nicht standardmäßig / ohne weiteres / ohne weitere Anpassung(en) über das Netzwerk überprüfen lassen, wie die Kapazität einer Festplatte auf einem entfernten Server(, das (Laufen) eines Prozesses) oder die Durchsuchung einer Logdatei nach bestimmten (Stop)wörtern.

Nagios bietet verschiedene Möglichkeiten an solche Dienste zu überprüfen:

\begin{figure}[ht]
	\centering
	   \fbox{\includegraphics[width=0.65\textwidth]{bilder/nagios-kern.png}}
		\caption[Verschiedene Überwachungsmöglichkeiten von Nagios]{Verschiedene Überwachungsmöglichkeiten von Nagios\protect\footnote}
		\label{nagios-kern}
\end{figure} 
\footnotetext{Quelle: \cite{Barth08} S. 98}

\paragraph{Methode 1 - Netzwerkdienste}
Der zuvor, in Abbildung \ref{check-http}, gezeigte Test eines netzwerkbasierenden Dienstes wird im obigen Bild mit dem Client-Rechner (mit der Nummer) 1 abgebildet.
%Die Überprüfung von nicht netzwerkbasierenden Diensten soll mit den restlichen Client-Rechnern aufgezeigt werden.
Dies ist die einfachste Überwachungsmethode, da keine zusätzlichen Programme oder aufwändige Konfiguration benötigt wird.
Vorteilhaft ist auch, dass der Dienst über das Netzwerk getestet wird, so wie der Benutzer auch auf den Dienst zugreift.
Damit können auch gleichzeitig andere Knotenpunkte wie Switches überwacht werden.

\paragraph{Methode 2 - SSH}

Falls es sich beim Client um ein Unixderivat handelt, ist der entfernte Zugriff auf diesen Client per SSH\footnote{Durch eine Secure Shell (\gls{SSH}) kann man sich eine verschlüsselte Netzwerkverbindung zum entfernten Rechner aufbauen.}-Dienst möglich.
Dazu muss auf dem Client ein \gls{SSH}-Benutzerkonto angelegt sein, mit dem sich Nagios anmelden kann und die öffentlichen Schlüssel (zwischen Nagios Server und Client) ausgetauscht werden, damit keine passwortabhängige Benutzerauthentifizierung (Eingabe von PW) notwendig ist.
Danach können lokale Ressourcen, wie Festplattenkapazität oder Logdateien mit dem entsprechenden Plugin direkt auf dem entfernten Rechner überwacht werden.
Damit der Client diese Plugins verwenden kann, müssen sich die gewünschten Plugins (auch) auf dem Client (lokal) befinden.
Eine beispielhafte Verwendung mit dem dafür gedachten Nagios Plugin \pictext{check\_by\_ssh} (von dieser Überwachungsmethode) wird in Abbildung \ref{check-ssh} gezeigt.

\begin{figure}[ht]
	\centering
	   \fbox{\includegraphics[width=0.85\textwidth]{bilder/check_by_ssh.png}}
		\caption{Beispielhafte manuelle Ausführung eines Servicechecks über SSH}
		\label{check-ssh}
\end{figure}

(Hier beachten, dass kein Passwort abgefragt wird, daher zuvor Schlüsselaustauschen)

\paragraph{Methode 3 - NRPE}

Eine alternative Möglichkeit solche Dienste auf entfernten Rechnern zu überwachen, ist durch den sogenannten Nagios Remote Plugin Executor (\gls{NRPE}).
Hier muss auf dem Client ein "`Agent"' installiert werden, welcher einen Port öffnet mit dem der Agent mit dem Nagios Server kommuniziert.

\begin{figure}[ht]
	\centering
	   \fbox{\includegraphics[width=0.9\textwidth]{bilder/nrpe.png}}
		\caption[Aktive Checks mit NRPE]{Aktive Checks mit NRPE\protect\footnote}
		\label{aktivchecks}
\end{figure}
\footnotetext{Quelle: \url{http://www.nagios.org/images/addons/nrpe/nrpe.png}}

Der Nagios Server kann dann Anforderungen über das Nagios-Plugin \pictext{check\_nrpe} an den Client verschicken.
Ein Aufruf dieses Plugins ist dem des \pictext{check\_by\_ssh} Plugins, siehe dazu Abbildung \ref{check-ssh}, sehr ähnlich.

Der Nachteil dieser Variante ist ein zusätzlich geöffneter Port und der höhere / erhöhte Aufwand beim Installieren des Agenten im Gegensatz zum (vermutlich /meistens) bereits laufendem \gls{SSH}-Dienst.
Zusätzlich gibt es nur die Möglichkeit die Anfragen auf diesem Port auf bestimmte IPs zu beschränken, jedoch nicht den Zugriff durch ein Passwort zu sichern.
Dafür beschränkt sich der \gls{NRPE} (lediglich) auf die auf dem entfernten Client liegenden Nagios Plugins und kann nicht System- bzw. Benutzerkommandos aufrufen, wie bspw. das \pictext{rm} Kommando zum Löschen von Dateien, welche durch den Einsatz von \pictext{check\_by\_ssh} standardmäßig möglich wären.
Sicherheitstechnisch gesehen ist die SSH-Variante kritischer, da es einem Angreifer ermöglicht auf diese System- bzw. Benutzerkommandos zuzugreifen, wenn er die Kontrolle über den Nagios Server erlangt.
Beide Verfahren unterstützen die Verschlüsselung der Datenübertragung zwischen Nagios-Server und Client, so dass keine Informationen im Klartext übertragen werden.

\paragraph{Methode 4 - SNMP}

%nur grob angerissen / kurz / 
Diese Variante wird nur verkürzt behandelt, da sich diese Arbeit hauptsächlich mit der Überwachung von Servern beschäftigt und nicht von Netzwerkkomponenten wie Switches oder Router, die nur durch das Simple Network Management Protocol (\gls{SNMP}) überwacht werden können, wenn mehr Informationen als eine schlichte Erreichbarkeit überprüft / gesammelt werden soll.

Barth schreibt über diese Variante / Überwachungsmethode:
\begin{quote}"`Mit dem Simple Network Management Protocol \gls{SNMP} lassen sich ebenfalls lokale Ressourcen übers Netz abfragen [...]. Ist auf dem Zielhost ein \gls{SNMP}-Daemon installiert [...] kann Nagios ihn nutzen, um lokale Ressourcen wie Prozesse, Festplatten oder Interface-Auslastung abzufragen."' \begin{flushright}\cite{Barth08} S. 101\end{flushright}\end{quote} 

Durch \gls{SNMP} kann auf die strukturierte Datenhaltung der \gls{MIB}\footnote{Die Management Information Base (\gls{MIB}) dient als \gls{SNMP}-Informationstruktur und besteht aus einem hierarchischen, aus Zahlen aufgebauten Namensraum. Ähnliche Struktur wie andere hierarchische Verzeichnisdiensten wie \gls{DNS} oder \gls{LDAP}. Quelle: \cite{Barth08} S.233} in den entfernten Netzwerkknoten zugegriffen werden.
%###########################
Die \gls{MIB}-Struktur ist folgendermaßen aufgebaut:

\begin{figure}[ht]
	\centering
	   \fbox{\includegraphics[width=0.95\textwidth]{bilder/mib.png}}
		\caption[Struktur der Management Information Base]{Struktur der Management Information Base\protect\footnotemark}
		\label{munin-mib}
\end{figure}
\footnotetext{Quelle: \cite{Mu08} S. 156}
Anhand dieser Anordnung können die \gls{SNMP}-Plugins von Nagios den gewünschten Wert über das Netzwerk abfragen.
Bei einem Switch werden die auslesbaren Informationen vom Hersteller bestimmt.
Wenn auf einem Rechner eigene Ergebnisse in der \gls{MIB} abgespeichert werden sollen, muss dies durch einen \gls{SNMP}-Daemon eingetragen werden.
Dessen Konfiguration ist im Vergleich zu den anderen Überwachungsmethoden deutlich komplexer.

Es gibt zwei verschiendene Möglichkeiten Dienste mit \gls{SNMP} zu überwachen.
Der Server frägt aktiv den Inhalt der entsprechenden MIB Einträgen periodisch ab oder der Client sendet asychron seine Statusmeldungen an den Nagios-Server.
Beim letzteren spricht man auch von so genannten \gls{SNMP}-Traps.


\begin{itemize}
\item Warum wird SNMP nicht verwendet?
\item klartextübertragung bis SNMP 2c
\item Schreibrechte können Schaden anrichten
\item Brute Force attacken ausgesetzt
\item Beschränkte Ausgabemögliochkeit / maximale Datengrösse der Ausgabe -> Logüberwachung nur mit Aufwand möglich autarkes Programm von Nöten, dann muss selbst das Ergebniss in die MIB geschrieben werden, damit Nagios darauf Zugriff erlangt -> zu aufwändig im Vergleich mit Agenten
\end{itemize}

%Einen beispielhaften Zugriff auf SNMP-fähige Geräte wird in Abbildung \ref{munin-snmp} gezeigt.

%\begin{figure}[ht]
%	\centering
%	   \fbox{\includegraphics[width=0.5\textwidth]{bilder/snmp.png}}
%		\caption[Beispielhafter Zugriff auf SNMP-fähige Geräte]{Beispielhafter Zugriff auf SNMP-fähige Geräte\protect\footnotemark}
%		\label{munin-snmp}
%\end{figure}
%\footnotetext{Quelle: \cite{Mu08} S. 156}

%##################

Weiterhin unterscheidet man generell zwischen aktiven und passiven Checks.
\paragraph{Methode 5 - NSCA}
%asynchron!
Bei passiven Tests führt der zu überwachende Computer das statuserzeugende Plugin selbst aus und sendet es über ein weiteres Plugin zum Nagios-Server.
Hierfür muss das Testprogramm bzw. Script und das entsprechende Plugin \pictext{send\_ncsa}, welches zum Versenden der Informationen zuständig ist, auf dem Host vorhanden sein.
Auf der anderen Seite muss der \pictext{\gls{NSCA}} (Nagios Service Check Acceptor) auf dem Nagios-Server als Daemon gestartet sein, damit die übermittelten Ergebnisse von Nagios entgegengenommen werden.

Folgende Abbildung soll das Prinzip der passiven Checks verdeutlichen:
%Das Prinzip der passiven Checks lässt sich durch folgende vereinfachte Abbildung
\begin{figure}[ht]
	\centering
	   \fbox{\includegraphics[width=0.9\textwidth]{bilder/nsca.png}}
		\caption[Passive Checks mit NSCA]{Passive Checks mit NSCA\protect\footnote}
		\label{passivchecks}
\end{figure}
\footnotetext{Quelle: \url{http://www.nagios.org/images/addons/nsca/nsca.png}}

Das Testprogramm \textit{Remote Application} wird selbständig vom zu überwachenden Rechner \textit{Remote Host} aufgerufen und übermittelt durch das \pictext{send\_ncsa} Plugin die Ergebnisse über das Netzwerk an den Nagios-Server \textit{Monitoring Host}.
Da auf diesem der NSCA als Daemon läuft können die Ergebnisse an die Nagios-Anwendung zur Auswertung weitergegeben werden.

\begin{itemize}
\item Kurz agenten, zeigen auf f. Kapitel -> SNMP erklären (MIB, OID) Sicherheitsrisiko
\end{itemize}

\subsection{Überwachungslogik (mit Alarmierung/Benachrichtigung)}
\label{dismoni}
\begin{center}
TODO: Distributed Monitoring bezug auf allg überwachungssysteme
\end{center}


\subsubsection{Plugins}
Gedacht für Linux umgebung

Verschiedene Möglichkeiten Checks zu realisieren unter Unix Systemen:

\begin{figure}[ht]
	\centering
	   \fbox{\includegraphics[width=0.9\textwidth]{bilder/unix-agents.png}}
		\caption{Übersicht der verschiedenen Unix Agenten\protect\footnotemark}
		\label{nix-agents}
\end{figure}
\footnotetext{Quelle: \url{http://www.kilala.nl/Sysadmin/index.php?id=708}}

Leicht programmierbar -> perl
Extra Plugins für Windows

\subsubsection{(Windows) Agenten oder allgemein Einholen von Infos}
Warum nicht einfach alles über SNMP? -> ODI muss man erst beantragen, hoher Aufwand und dann doch nicht so universell/alles abdeckend wie aktive checks, man kann keine logfiles durchuchen -> könnte es aber als standalone prog auf dem client laufen lassen und dieser sendet dann passive checks

Sagen das auf alten NSClient verzichtet wird und OpMon Agent nicht behandelt
\begin{enumerate}
\item Bilder ausm Nagios Buch Seite 472ff!
\item NSClient++
\item NC\_Net
\item NRPE\_NT
\end{enumerate}

Zusammenfassung?

\begin{figure}[ht]
	\centering
	   \fbox{\includegraphics[width=0.9\textwidth]{bilder/win-agents.png}}
		\caption{Übersicht der verschiedenen Windows Agenten}\footnote{Quelle: \url{http://www.kilala.nl/Sysadmin/index.php?id=709}}
		\label{win-agents}
\end{figure}

Welche wird jetzt eingesetzt und warum?

Erwähne sichheitsstechnisch Parameter erlauben oder nicht erlauben

Dabei sagen, dass wenn nicht erlaubt sind keine zentrale Konfiguration der Checks auf dem Nagios server möglich ist -> abwägen

\section{Oracle UCM}

\subsection{Allgemein}
Oracle Universal Content Management basiert auf der Software Stellent von der gleichnamigen Firma Stellent, welche im November 2006\footnote{Quelle: \cite{OraPress}} von Oracle gekauft wurde.

\subsection{Aufbau}

Die Architektur des \gls{OracleUCM}-Systems gliedert sich in separate Komponenten auf wie in Abbildung \ref{ucm-arch} gezeigt wird.

\begin{figure}[ht]
	\centering
	   \fbox{\includegraphics[width=0.5\textwidth]{bilder/basic_architecture.png}}
		\caption[Oracle UCM Architektur]{Oracle UCM Architektur\protect\footnote}
		\label{ucm-arch}
\end{figure}
\footnotetext{Quelle: \cite{ClubOra}}

Die Anwendung \gls{OracleUCM} ist aus folgenden Kernkomponenten aufgebaut:

\paragraph{Content Server}

Der Content Server ist das Herzstück der Oracle UCM Anwendung und basiert auf einer Java-Anwendung.
Er dient als Grundgerüst (Framework) für darüber liegende Funktionen, da er für die Ablage der Dokumente sowie deren Verwaltung, siehe Abbildung \ref{lifecircle}, verantwortlich ist.

\begin{figure}[ht]
	\centering
	   \fbox{\includegraphics[width=0.85\textwidth]{bilder/contenserver.png}}
	  % \fbox{Quelle: \cite{Huff06} S. 17}
		\caption[Beispielhafter Einsatz eines Content Servers]{Beispielhafter Einsatz eines Content Servers\protect\footnote}
		\label{ucm-cs}
\end{figure}
\footnotetext{Quelle: \cite{Huff06} S. 17}

Dieses Framework ist als Service-Oriented Architecture (\gls{SOA}) aufgebaut.
Im Kontext des Content Servers wird als Service ein diskreter Aufruf einer Funktion verstanden.
Dabei kann diese Funktion das Hinzufügen, die Bearbeitung, die Konvertierung oder das Herunterladen eines Dokumentes bedeuten.
Diese Services und ihre einzelnen Funktionen werden durch das \gls{SOA}-Framework verdeckt und stehen als Web Service zur Verfügung.

\paragraph{Inbound Refinery}
Die Inbound Refinery ist für die Konvertierung der Dokumente zuständig und ist keine interne Komponente des Content Servers, sondern kann sich auch auf einem anderen Server befinden.
Dabei werden spezielle Add-ons (Filter) für die Konvertierung verwendet.
In zeitlichen Abständen überprüft die Inbound Refinery, ob die bisher eingecheckten Dokumente konvertiert werden müssen, und speichert die konvertierte Datei in den Web Layout-Ordner.


\paragraph{Data Storage} Der Content Server verwaltet die Datenbank, die die Metadaten über die Dokumente beinhaltet.
Diese Metadaten werden für die Versionierung, Verwaltung und Suchanfragen verwendet.

\paragraph{Content Storage} Der Content Storage liegt auf dem Dateisystem und ist in Vault und Web Layout aufgeteilt.

\paragraph{Vault und Web Layout}
Der \textbf{Vault} ist ein Ordner auf dem Server in dem die Originaldateien der Benutzer in ihrem nativen Format gespeichert werden.
Im Gegensatz dazu werden im \textbf{Web Layout} die konvertierten Versionen der Dokumente abgelegt. Beispielsweise eine \gls{PDF}-Version einer Microsoft Word-Datei.

\paragraph{Search Engine}
Eine Suchanfrage eines Benutzers wird zuerst an den Webserver gesendet, der die Anfrage an den Content Server weitergibt.
Der Content Server verwendet anschließend seine Search Engine um ein Suchergebnis zu erhalten.
Das Suchergebnis wird dem Webserver übergeben, der das Ergebnis an den Benutzer sendet.
Die Search Engine verwendet einen Suchindex, der aus den Metadaten und Referenzen zu den Volltextversionen der Dokumente besteht.

\paragraph{Webserver}
Der Webserver ist hauptsächlich für die Präsentation und Ausgabe der gespeicherten Dokumenten und Informationen zuständig.
Dabei ist er auch für die Authentifizierung der Benutzer zuständig.


\subsection{Konkrete Verwendung}

\gls{OracleUCM} wird als Enterprise-Content-Management für die Verwaltung von Webseiten, Dokumenten und Bilder im Forschungszentrum Karlsruhe eingesetzt.

Dabei wird im konkreten Anwendungsfall \gls{OracleUCM} als Bilddatenbank verwendet.
Diese Bilddatenbank nimmt Fotos und Bilder der Benutzer entgegen (\textit{Einchecken}) und konvertiert das Originalbild dabei in andere Bildversionen wie beispielsweise eine verkleinerte Version für Webseiten.

Dieser typische Ablauf soll durch Abbildung \ref{bdbanw} verdeutlicht werden.
\begin{figure}[ht]
	\centering
	   \fbox{\includegraphics[width=0.9\textwidth]{bilder/bdb.png}}
		\caption{Bilddatenbank als Anwendung}
		\label{bdbanw}
\end{figure}

Die untere Tabelle zeigt die verschiedenen Bildversionen einer bereits konvertierten Bilddatei.\\

Da die Bilddatenbank unter einem Windows-Server betrieben werden soll, muss dies für die Überwachung bei der Auswahl der Überwachungselemente und Realisierung der Überwachung durch Nagios berücksichtigt werden.

\newpage
\section{Überwachungselemente}
\label{elemente}
Die Überwachung eines Dienstes über ein Netzwerk verteilt sich auf verschiedenen Ebenen mit unterschiedlichen Gewichtungen.
Zum Beispiel stellt das simple Senden eines Pings an den entsprechenden Server die niedrigste und primitivste Stufe dar, da hier lediglich die Netzwerkschnittstelle des Servers auf ihre Funktionalität und dabei der Status der Netzwerkstrecke getestet wird.
Ob die Anwendung überhaupt auf dem Server läuft und in welchem Zustand sie sich befindet, muss auf eine andere Weise herausgefunden werden.

Dabei lassen sich aus den verschiedenen Überwachungselemente vier Kategorien Statusabfragen, Funktionalitätstests, Auswertung von Logdateien und Benutzersimulation bilden.

\subsection{Statusabfragen}
\label{syschecks}
Diese Kategorie besteht aus einfacheren Überprüfungen, die jeweils den Status des Überwachungselementes überwachen.
Dabei können weitere Untergruppen gebildet werden:

\paragraph{System}
\begin{itemize}
\item \textbf{Ping} Überprüft, ob der Rechner vom Nagios-Server über das Netzwerk erreichbar ist.
\item \textbf{Prozessorauslastung} Überwacht die Auslastung des Prozessors und schlägt bei ungewöhnlich hohen Werten Alarm.
\item \textbf{Festplattenspeicherausnutzung} Überwacht die Speicherplatzauslastungen der verschiedenen Festplattenpartitionen, damit immer genügend Speicherplatz für Anwendungen und Betriebssystem verfügbar ist.
\item \textbf{Arbeitsspeicherauslastung} Beobachtet wie viel Arbeitsspeicher vom System verwendet wird und wie viel davon noch zur Verfügung steht.
\end{itemize}

\paragraph{Prozesse}
\begin{itemize}
\item \textbf{IdcServerNT.exe} Der Prozess der \gls{OracleUCM}-Awendung.
\item \textbf{IdcAdminNT.exe} Der Prozess für das Administration-Webinterface von \gls{OracleUCM}.
\item \textbf{w3wp.exe} Der Prozess des Webservers Microsoft "`Internet Information Services"'
\end{itemize}

\paragraph{Services}
\begin{itemize}
\item \textbf{IdcContentService}  Den Zustand des \pictext{sccdms01}-Dienst überprüfen.
\item \textbf{IdcAdminService}  Den Zustand des \pictext{sccdms01\_admin}-Dienst für die Administration überprüfen.
\item \textbf{Zeitsynchronisationsdienst} Überprüfen, ob der \pictext{W32TIME}-Dienst, der für den Zeitabgleich mit einem Zeitserver zuständig ist, läuft und die Abweichung zwischen Client und Zeitserver festhalten.
\item \textbf{Antivirusdienst} Den Zustand des Dienstes \pictext{Symantec AntiVirus} überprüfen, der für die Updates des Virusscanners notwendig ist.
\end{itemize}

\subsection{Überwachung der Funktionalität}
\label{funztest}
Durch die vorherigen Tests kann herausgefunden werden, ob eine Anwendung oder ein Dienst auf dem Server gestartet wurde.
Die Funktionalität kann durch solche Überprüfungen jedoch \textbf{nicht} sichergestellt werden.
Da beispielsweise der Prozess bzw. Dienst des Webservers gestartet ist, jedoch keine Webseite aufgerufen werden kann.
%Da beispielsweise der Webserver aufgrund eines kritischen Fehlers nicht erreichbar ist, der Prozess bzw. Dienst dennoch läuft.
Daher muss eine weitere Art von Überprüfungen die Anwendungen auf ihre Funktionalität testen.

\begin{itemize}
\item \textbf{Webserver} Aufruf einer Webseite auf dem Server. Wenn auf diese Anfrage eine gültige Antwort in Form einer Statuscode-Meldung erfolgt, kann die korrekte Funktion des Webservers festgestellt werden.

\item \textbf{Webinterface des Oracle UCM} Zusätzlich wird mit dieser Abfrage die Integration des Content-Management-Systems in den Webserver überwacht, da hier nicht nur der Webserver, sondern eine \gls{OracleUCM} spezifische Webseite abgefragt wird.

\item \textbf{Benutzeranmeldung am Oracle UCM} Hier wird getestet, ob sich ein Benutzer erfolgreich am System anmelden kann.
Dies wird mit Anmeldungsdaten eines lokalen Benutzers und eines Active Directory-Benutzers durchgeführt um gleichzeitig die Verbindung zum Active Directory-Server zu testen.

\item \textbf{Oracle Datenbank} Überprüft den Verbindungsaufbau zur Datenbank. Wenn keine Verbindung zur Oracle-Datenbank möglich ist, können keine neuen Informationen gespeichert werden. 

%\item \textbf{Status von Cronjobs} In periodischen Zeitabständen werden Programme aufgerufen, deren Aufruf und Endstatus/Endergebniss überwacht werden muss. 
%Damit nicht das vorherige Ergebnis zu einem False Negative führt, müssen hier zusätzliche Zeitinformationen/zeitliche Parameter beachtet/bedacht werden.

\item \textbf{Anzahl Datenbankverbindungen} Anzahl der Verbindungen zur Datenbank, da aus Performanzgründen eine Obergrenze mit einer maximalen Anzahl festgelegt ist.
\end{itemize}

\subsection{Auswerten von Logdateien}
\label{checklog}
%Die zwei bisherigen Kategorien beinhalten simple Zustandsüberprüfungen oder aktive Funktionaltests.
In dieser Kategorie werden zusätzlich verschiedene Logdateien auf spezielle Warnungs- und Fehlermeldungen anhand eindeutigen Stopwörtern untersucht.
Dies ist notwendig um reaktiv Fehlverhalten der Anwendung zu erkennen, das nicht mit den vorherigen Überwachungselementen entdeckt wurde.
Des weiteren können durch die Analyse der Logdateien etwaige Alarmmeldungen der bisherigen Tests bestätigt, begründet oder aufgehoben werden.
Somit bietet das Auswerten der Logdateien zusätzliche Sicherheit False Positive- oder False Negative-Meldungen auszuschließen.

Die Oracle UCM Anwendung erstellt drei verschiedene Arten von Logdateien:\footnote{Quelle: \cite{UCMlog09}}

\begin{itemize}
\item \textbf{Content Server Log} 
\item \textbf{Inbound Refinery Log}
\item \textbf{Archiver Log}
\end{itemize}

Um alle Logs ohne Probleme im Internetbrowser anzuzeigen, liegen alle Logdateien im HTML-Format vor.
Alle drei Arten von Logs bestehen jeweils aus 30 verschiedenen Dateien, die sich täglich abwechseln.
Dadurch wird für jeden Tag im Monat eine separate Datei verwendet, um bei vielen Warnungs- und Fehlermeldungen durch die chronologische Anordnung den Überblick zu behalten.
Dabei werden die Logdateien zwangsweise nach 30 Tagen nacheinander überschrieben.

Diese Rotation der Logdateien muss bei der Durchsuchung nach Stopwörter beachtet werden, damit stets die aktuelle Logdatei überwacht wird und keine veralteten Informationen für False Positive-Meldungen durch Nagios sorgen.

%wieso nur plugin check\_log und nicht einfahc umfangreicheres Standalone Programmw ie syslog\_nd oder 8pussy

\subsection{Benutzersimulation}

Ein Dokument nimmt in seinem Lebenszyklus, siehe Abbildung \ref{lifecircle}, verschiedene Zustände an.
So kommt es nach dem Einchecken in den Zustand \pictext{genwww}.
Der darauf folgende Zustand \pictext{fertig} gibt die erfolgreiche Konvertierung des Dokumentes bekannt.
Die folgende Indizierung wird durch den Zustand \pictext{freigegeben} angezeigt.

Die Benutzersimulation hat die Aufgabe alle Schritte der Zustandsänderung durch typische Abfragen zu überprüfen.

\begin{itemize}

\item \textbf{Einchecken von Dokumenten} Damit die eigentliche Aufgabe des Dokumentenverwaltungssystem überwacht werden kann, werden verschiedene Datenformate testweise eingecheckt. 
Dabei wird die Antwort der Anwendung auf das Hinzufügen der Dateien analysiert.

\item \textbf{Konvertierung} Da das hinzugefügte Dokument nicht nur einfach auf dem Server gespeichert wird, sondern dabei auch in ein anderes Format umgewandelt wird, muss diese Konvertierung zusätzlich überwacht werden. 
Wird beispielsweise ein Bild eingecheckt, wird dieses mehrfach in verschiedenen Auflösungen oder in einem anderen Bilddateiformat gespeichert. 
Ob diese Transformation erfolgreich ablief, kann anhand dieser neuen Dateien festgestellt werden.

\item \textbf{Indizierung} Bei dem Einchecken sollen auch gleichzeitig zusätzliche Informationen über das Dokument festgehalten werden. 
Diese Informationen können beispielsweise der Name des Autors, das Erstellungsdatum der Datei oder - bei Bildern - der verwendete Farbraum sein. 
Bei der Suche nach einem Dokument können diese Informationen als zusätzliche Suchkriterien verwendet werden.
Daher muss überprüft werden, ob diese Dateien richtig ausgelesen, der Datenbank hinzugefügt und vom Anwender abgefragt werden können.
Dabei werden auch zuvor ausgewählte Testdateien verwendet.

\item \textbf{Suchfunktion} Nach einer erfolgreichen Indizierung muss das eingecheckte Dokument per Suchanfrage gefunden werden.
%Ob die Suche und Indizierung erfolgreich abgelaufen ist, wird zusätzlich überprüft. 
Dabei wird der Suchbegriff an dem Dateinamen des Testbildes festgelegt.
\end{itemize}

\subsection{Zusammenfassung}

Die Basis für die alle anderen Tests bildet die Systemüberwachung.
An erster Stelle der Systemüberwachung steht die schlichte Erreichbarkeit über das Netzwerk per Ping.
Wenn der Server nicht erreichbar ist, können auch keine weiterführende Prüfungen durchgeführt werden.
Zur Systemüberwachung gehören auch allgemeine Informationen über die Systemressourcen wie freier Festplattenspeicher oder Prozessorauslastung.
Die nächste Stufe bildet die Überprüfung der laufenden Prozesse und der Status verschiedener Dienste bzw. Services.
Sollten bestimmte Prozesse nicht gefunden werden oder wichtige Dienste nicht gestartet sein, können auf diese Prozesse und Dienste aufbauende Checks nicht funktionieren.
Beispielweise kann der Funktionalitätstest der Benutzeranmeldung nicht realisierbar sein, wenn bereits zuvor in der Systemüberwachung der Prozess für den Webserver \gls{IIS} nicht gefunden werden konnte.\\


Alle Überwachungselemente lassen sich inklusive ihrer Abhängigkeiten in Form der Pyramide (Abbildung \ref{moniele}) darstellen.

\begin{figure}[ht]
	\centering
	   \fbox{\includegraphics[width=1\textwidth]{bilder/pyramide_fertig2.png}}
		\caption{Überwachungselemente}
		\label{moniele}
\end{figure}











