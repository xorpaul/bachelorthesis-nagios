\section{Zusammenfassung und Ausblick}

Im Laufe dieser Arbeit wurde eine Lösung entwickelt um die Funktionalität des im Forschungszentrum Karlsruhe verwendeten Dokumenten-Management-Systems \gls{OracleUCM} mit Nagios zu überwachen.
Eine solche Überwachung war notwendig um den Mitarbeitern des Forschungszentrums Karlsruhe einen möglichst zuverlässigen Dienst anbieten zu können, da sie proaktiv auf mögliche Fehlzustände testet und bei einer Störung eine Alarmmeldung an die verantwortlichen Kontaktpersonen versendet. 
Das Überwachungssystem sorgt dafür, dass jeder Fehler sofort gemeldet wird, damit die Problemquellen gefunden und behoben werden können, noch bevor die eigentlichen Anwender Störungen bei der Nutzung des Dienstes bemerken.\\

Für die Realisierung der Aufgabe war es notwendig sich zuerst mit den Grundlagen von Überwachungssystemen wie die Beachtung der Netzwerkstruktur und der Sicherheitsaspekte beim Einsatz einer Überwachungssoftware auseinander zusetzten. 
Um die eigentliche Funktions- und Arbeitsweise eines Dokumenten-Management-Systems zu verstehen wurde die grundsätzliche Art eines Dokumentes im Vergleich zu Daten betrachtet.
Dabei definiert die Strukturiertheit der enthaltenen Informationen die Zuordnung zu Daten oder Dokumente, obwohl die Grenzen nicht eindeutig festgelegt sind.
Auf diesem Wissen aufbauend konnten die Aufgabenbereiche Eingabe, Verwaltung, Archivierung und Ausgabe eines Dokumenten-Management-Systems untersucht werden und Vergleiche zu Content-Management-Systemen gezogen werden.
Das kleinste Objekt in einem \gls{DMS} kann nur ein einzelnen Dokument sein, während ein \gls{CMS} einzelne Informationen aus verschiedenen Dokumenten erfassen kann, um ein neues Dokument zu generieren.

Für die spätere Umsetzung der Aufgabenstellung wurde der Ansatz von Service-Orientierte Architekturen in Verbindung mit dem Aufbau und Funktionsweise einer Web-Services-Architektur verwendet.
Hierfür war es notwendig sich mit Grundprinzipien dieser Architekturen, verwendete Elemente (bspw. \gls{SOAP}, \gls{WSDL}) und deren Ablauf vertraut zu machen, damit für die Realisierung der Überwachung auf die Funktionen korrekt zugegriffen werden konnte.\\

Im Forschungszentrum Karlsruhe wird als Überwachungssoftware das Open Source Programm Nagios für die Überwachung von Netzwerken, Server und Dienste verwendet.
Damit mögliche auftretenden Fehler korrekt von Nagios erkannt werden, musste sich mit der Funktionsweise und Aufbau dieser Software auseinandergesetzt werden.
Die Realisierung der Aufgabenstellung wurde durch die Anpassung der verschiedenen Konfigurationsdateien erreicht, da über diese Dateien die zu überwachenden Dienste und Server mit Nagios verbunden werden.
Die eigentliche Überwachung in Form von Informationen wird durch Plugins ermöglicht.
Das Verständnis über die Struktur und Richtlinien dieser Plugins wurde benötigt um sie effektiv einzusetzen und später eigene Plugins zu entwickeln.
Dabei galt es die speziellen Funktionen von Nagios wie die Hard und Soft States oder das Flapping von Zustände bei der spätere Verwendung zu beachten.
Da Nagios verschiedene Möglichkeiten bietet die benötigten Informationen zu sammeln wurde eine Übersicht dieser unterschiedlichen Überwachungsmethoden erstellt und miteinander verglichen.

Da \gls{OracleUCM} im Forschungszentrum Karlsruhe für die Verwaltung von Webseiten, Dokumenten und Bilder eingesetzt wird, wurde der allgemeine interne Aufbau und Arbeitsweise dieser Anwendung für die Ermittlung der Überwachungselemente untersucht.
Um die Überwachung zu realisieren wurde der konkrete Einsatz von \gls{OracleUCM} als Bilddatenbank untersucht und typische Benutzeraktionen für die später folgenden Benutzersimulation ermittelt.

Die einzelnen Überwachungselemente wurden in die Ebenen Statusabfragen, Funktionalitätstest, Auswerten von Logdateien und Benutzersimulation unterteilt.
Dabei führte die Abhängigkeit der Elemente zueinander zu der Einordnung in die verschiedenen Ebenen.

Für die Umsetzung wurde ein Testsystem aufgesetzt, das aus einer eigenen Nagios-Installation zum Testen der Überwachung und einer virtuellen Maschine als Klon der Bilddatenbank zum Simulieren der einzelnen Fehlzustände bestand.
Für den konkreten Einsatz von Nagios als Überwachungssystem musste sich mit den verschiedenen Nagios-Agenten auseinandergesetzt werden, da diese die benötigten Informationen von entfernten Servern generieren.
Die zuvor beschriebenen Überwachungsmethoden von Nagios wurden dabei in Verbindung mit den Agenten, die Unix- und Windows-Agenten aufgeteilt wurden, nochmals aufgegriffen.
Da es sich bei der zu überwachenden Bilddatenbank um einen Windows-Server handelte, wurde ein passender Nagios-Agent ausgewählt und dessen Installation und Konfiguration erläutert.
Durch den Einsatz dieses Agenten konnten die verschiedenen Ebenen der Überwachung durch Verwendung von verschiedenen Überwachungsmethoden realisiert werden.
Dabei wurde für die Benutzersimulation eigene Plugins entwickelt, die die Benutzeraktionen per Web Service ausführen.
%Zu den typischen Aktionen der Anwender zählt das Hochladen eines Bildes (Einchecken).
Das Plugin simuliert das Hinzufügen eines Testbildes die Erreichbarkeit und Funktionalität des \gls{OracleUCM}-Servers.
Die Konvertierung und Indizierung wird durch ein weiteres Plugin überprüft, welches verschieden andere Web Services dazu verwendet.
Dafür mussten die Konsequenzen der automatischen Benutzersimulation bedacht werden, da sie durch Nagios ständig periodisch ausgeführt wird und die Ressourcen des Servers wie der Festplattenspeicherplatz beschränkt sind.
Zur Lösung dieses Problems wurde ein eigener Web Service zum Löschen des Testbildes und der konvertierten Version erstellt.

Dadurch konnte die Überwachung aller zuvor ermittelten Elemente realisiert werden und die Informationen der einzelnen Plugins im Webinterface von Nagios eingesehen werden.\\

Bei einer Überwachung ist es es notwendig zuvor verschiedene Schwellwerte zu setzen an denen die Überwachungssoftware festlegen kann, ob ein Objekt einen kritischen Zustand besitzt oder nicht.
Für die Ermittelung dieser Größen müssen die Werte der Überwachungselemente über einen längeren Zeitraum beobachtet werden.
Aufgrund des begrenzten zeitlichen Rahmens dieser Arbeit konnte dies nicht vollständig umgesetzt werden, so dass es während dem Betrieb fortgesetzt werden muss.
Hierunter fallen vor allem spezifische Merkmale eines Servers wie eine ungewöhnlich hohe Prozessorauslastung, die durch eine zeitlich gesteuerte Sicherung entstanden ist und deshalb von der Überwachungssoftware nicht als Fehlverhalten interpretiert werden sollte.
Auch die Liste der Stopwörter für die Auswertung der Logdateien muss für neue bisher unbekannte Fehler immer wieder erweitert werden.

Das entwickelte Plugin für die Benutzersimulation kann auch mit weiteren Funktionen versehen werden, die weitere Web Services und damit Funktionalitäten überprüfen können.
Dabei muss sich die Benutzersimulation nicht auf weitere Bilddatenbank ähnliche Dokumenten-Management-Systeme beschränken, sondern kann auch auf anderen Systemen eingesetzt werden.
Hierfür müsste nur anstatt eines Testbildes beispielsweise eine \gls{PDF}-Testdatei oder Word-Dokumenten verwendet werden und die Validierung der Indizierung und Konvertierung entsprechend angepasst werden.

%Die angepassten Konfigurationsdateien können durch ihren Aufbau einfach auf einen anderen Nagios exportiert und verwendet werden.

%\begin{itemize}
%
%\item Zusammenfassung
%\item Eigener Nagios-Server aufgesetzt / Nagios
%\item Verwendung der Bilddatenbank VM / Oracle UCM
%\item Überwachung der verschiedenen Ebenen wurde realisiert
%\item Dabei auf Sicherheit geachtet (Port, Passwort, Verschlüsselung)
%
%\item Ausblick
%\item Übernahme der Konfigurationsdateien auf vorhandenen Nagios-Server möglich
%\item Überwachung von anderen Content-Servern via Webservice möglich (PDF, DOC …)
%
%
%\item Geeignete Stopwörter für Logdateien müssen noch gefunden / eruiert werden
%\item Passende Schwellwertdefinitionen können erst nach einer gewissen Laufzeit festegelegt werden
%\item Export der entwickelten Überwachung auf den produktiven Haupt-Nagios Server
%\end{itemize}