\section{Zusammenfassung}

Im Rahmen dieser Arbeit wurde eine Lösung entwickelt um mit der Open Source-Überwachungssoftware Nagios den Betrieb des im Forschungszentrum Karlsruhe verwendeten Dokumenten-Management-Systems \gls{OracleUCM} zu überwachen.
Eine solche Überwachung ist notwendig um den Mitarbeitern des Forschungszentrums Karlsruhe einen möglichst zuverlässigen Dienst anbieten zu können.
Dabei sollte die Überwachung proaktiv auf mögliche Fehlzustände testen und bei einer Störung eine Alarmmeldung an die verantwortlichen Kontaktpersonen versenden. 
Das Überwachungssystem sorgt dafür, dass jeder Fehler sofort gemeldet wird, damit die Problemquellen vom Administrator gefunden und eventuell behoben werden können, bevor die Endbenutzer Störungen bei der Nutzung des Dienstes bemerken.\\

Für die Bearbeitung der Aufgabe war es notwendig sich mit den Grundlagen von Überwachungssystemen auseinander zusetzten.
Darunter fielen die Punkte Netzwerkstruktur , -abhängigkeit und verschiedene Sicherheitsaspekte die beim Einsatz einer Überwachungssoftware eine Rolle spielen.
Um die eigentliche Funktions- und Arbeitsweise eines Dokumenten-Management-Systems zu verstehen wurde die grundsätzliche Art eines Dokumentes im Vergleich zu Daten betrachtet.
%Dabei definiert die Strukturiertheit der enthaltenen Informationen die Zuordnung zu Daten oder Dokumente, obwohl die Grenzen nicht eindeutig festgelegt sind.
Auf diesem Wissen aufbauend konnten die Aufgabenbereiche Eingabe, Verwaltung, Archivierung und Ausgabe eines Dokumenten-Management-Systems untersucht und Vergleiche zu Content-Management-Systemen gezogen werden.
%Das kleinste Objekt in einem \gls{DMS} kann nur ein einzelnen Dokument sein, während ein \gls{CMS} einzelne Informationen aus verschiedenen Dokumenten erfassen kann, um ein neues Dokument zu generieren.

Für die Umsetzung wurde die Service-orientierte Architektur der \gls{OracleUCM}-Anwendung in Verbindung mit Web-Services verwendet.
Hierfür war es notwendig sich mit Grundprinzipien dieser Architekturen, deren Funktionsweise und verwendete Elemente vertraut zu machen.
Dadurch konnte später korrekt auf die benötigten Funktionen zugegriffen werden.\\

Im Forschungszentrum Karlsruhe wird als Überwachungssoftware das Open Source-Programm Nagios für die Überwachung von Netzwerken, Server und Dienste verwendet.
Damit Fehler korrekt von Nagios erkannt werden, bestand die Notwendigkeit die Funktionsweise und den Aufbau dieser Software zu studieren.
%Um die gewünschte Überwachung durch Nagios zu erreichen / ermöglichen
%Die Realisierung der Aufgabenstellung wurde durch die Anpassung der verschiedenen Konfigurationsdateien erreicht, da über diese Dateien die zu überwachenden Dienste und Server mit Nagios verbunden werden.
Das Einholen von Informationen zur Auswertung wird durch Plugins ermöglicht.
Das Verständnis über die Struktur und Richtlinien dieser Plugins wurde benötigt um später eigene zu entwickeln und sie effektiv zu verwenden.
Dabei galt es die speziellen Funktionen von Nagios wie die Hard und Soft States oder das Flapping von Zustände bei der spätere Verwendung zu berücksichtigen.
Über die verschiedene Möglichkeiten die benötigten Informationen zu sammeln wurde ein kurzer Überblick gegeben.

\gls{OracleUCM} wird im Forschungszentrum Karlsruhe für die Verwaltung von Webseiten, Dokumenten und Bilder eingesetzt.
Durch die Untersuchung des allgemeinen internen Aufbaus und der Arbeitsweise dieser Anwendung konnten die notwendigen Überwachungselemente ermittelt werden.
%Für die Ermittlung der Überwachungselemente wurde der  in allen Einsatzfällen untersucht.
Für diese Arbeit wurde der konkrete Einsatz von \gls{OracleUCM} als Bilddatenbank verwendet.
Die dabei auftretenden typischen Benutzerinteraktionen wurden für die später folgende Benutzersimulation verwendet.

Die einzelnen Überwachungselemente wurden in die Ebenen Statusabfragen, Funktionalitätstest, Auswerten von Logdateien und Benutzersimulation unterteilt.
Dabei führte die Abhängigkeit der Elemente zueinander zu der Einordnung in die verschiedenen Ebenen.
Unter den Statusabfragen befinden sich einfache Test wie ein Ping, Arbeitsspeicherauslastung oder der Zustand eines Prozesses.
Bei den Funktionalitätstest werden Anwendungen verwendet und die Antwort ausgewertet wie beispielsweise eine Anmeldung an Webserver mit Benutzerdaten.
Die Benutzersimulation beinhaltet verschiedene Benutzeraktionen und überprüft, ob die Anwendung noch alle Funktionen erfüllt.
Diese Einteilung in die verschiedenen Überwachungsebenen gibt den Verantwortlichen einen besseren Überblick über die Fehlersituation, so dass Fehlerquellen schneller entdeckt werden können.

Für die Umsetzung wurde ein Testsystem aufgesetzt, das aus einer separaten Nagios-Installation zum Testen der Überwachung und einer virtuellen Maschine, als Klon der Bilddatenbank zum Simulieren der einzelnen Fehlzustände, bestand.
%Für den konkreten Einsatz von Nagios als Überwachungssystem musste sich mit den verschiedenen Nagios-Agenten auseinandergesetzt werden, da diese die benötigten Informationen von entfernten Servern generieren.
%Die zuvor beschriebenen Überwachungsmethoden von Nagios wurden dabei in Verbindung mit den Agenten, die Unix- und Windows-Agenten aufgeteilt wurden, nochmals aufgegriffen.
Da es sich bei der zu überwachenden Bilddatenbank um einen Windows-Server handelte, wurde ein passender Nagios-Agent ausgewählt und dessen Installation und Konfiguration erläutert.
Durch den Einsatz dieses Agenten konnten die verschiedenen Ebenen der Überwachung durch Verwendung von verschiedenen Überwachungsmethoden realisiert werden.
Dabei wurde für die Benutzersimulation eigene Plugins entwickelt, die die Benutzeraktionen per Web Service ausführen.
%Zu den typischen Aktionen der Anwender zählt das Hochladen eines Bildes (Einchecken).
Das Plugin testet mit dem Hinzufügen eines Testbildes die Erreichbarkeit und einen Teil der Funktionalität des \gls{OracleUCM}-Servers.
Durch eine Suchanfrage wird die Indizierung überprüft und die Konvertierung wird anhand der angeforderten Testbilder validiert.
Dazu werden von den Plugins Web Services der \gls{OracleUCM}-Anwendung aufgerufen.
Die Konsequenzen einer automatischen Benutzersimulation mussten beachtet werden.
Durch die ständige Ausführung der Benutzersimulation würden die Ressourcen des Servers wie der Festplattenspeicherplatz an ihre Kapazitäten stoßen.
Damit das Testbild und die konvertierten Versionen gelöscht werden konnten, musste ein Web Service in der \gls{OracleUCM}-Anwendung angelegt werden.

Alle Überwachungselemente sind im Webinterface vom Nagios aufgeführt und erlauben den Administratoren einen schnellen Überblick über die korrekte Funktion der Anwendung bzw. über die aufgetretenen Fehler.
Die korrekte Benachrichtigung über Störungen konnte durch die Simulation der Fehlzustände in der virtuellen Maschine sichergestellt werden.
\newpage
\section{Ausblick}
Bei einer Überwachung ist es notwendig zuvor verschiedene Schwellwerte zu setzen.
An diesen Werten kann die Überwachungssoftware festlegen, ob ein Objekt einen kritischen Zustand erreicht hat oder nicht.
Für die Ermittelung dieser Größen müssen die Werte der Überwachungselemente über einen längeren Zeitraum beobachtet und analysiert werden.
Aufgrund des begrenzten zeitlichen Rahmens dieser Arbeit konnte dies nicht vollständig umgesetzt werden, so dass es während dem Betrieb fortgesetzt werden muss.
Hierunter fallen vor allem spezifische Merkmale eines bestimmten Servers.
Eine ungewöhnlich hohe Prozessorauslastung, die durch eine zeitlich gesteuerte Sicherung entstehen kann, sollte von der Überwachungssoftware nicht als Fehlverhalten interpretiert werden.
Auch die Liste der Stopwörter für die Auswertung der Logdateien muss für neue bisher unbekannte Fehler immer wieder erweitert werden.

Das entwickelte Plugin für die Benutzersimulation kann auch mit zusätzliche Funktionen versehen werden.
Durch die Verwendung von anderen Web Services können weitere Funktionalitäten überprüft werden.
Dabei kann die Benutzersimulation auch auf anderen Dokumenten-Management-Systemen eingesetzt werden.
Das Plugin kann leicht angepasst werden um, anstatt eines Testbildes, beispielsweise den Check-In, Indizierung und Konvertierung einer \gls{PDF}-Testdatei oder eines Word-Dokuments zu überwachen.
%Hierfür müsste nur anstatt eines Testbildes beispielsweise eine \gls{PDF}-Testdatei oder Word-Dokument verwendet werden.
%Für die Validierung der Indizierung und Konvertierung müsste das Plugin nur leicht angepasst werden.

%Die angepassten Konfigurationsdateien können durch ihren Aufbau einfach auf einen anderen Nagios exportiert und verwendet werden.

%\begin{itemize}
%
%\item Zusammenfassung
%\item Eigener Nagios-Server aufgesetzt / Nagios
%\item Verwendung der Bilddatenbank VM / Oracle UCM
%\item Überwachung der verschiedenen Ebenen wurde realisiert
%\item Dabei auf Sicherheit geachtet (Port, Passwort, Verschlüsselung)
%
%\item Ausblick
%\item Übernahme der Konfigurationsdateien auf vorhandenen Nagios-Server möglich
%\item Überwachung von anderen Content-Servern via Webservice möglich (PDF, DOC …)
%
%
%\item Geeignete Stopwörter für Logdateien müssen noch gefunden / eruiert werden
%\item Passende Schwellwertdefinitionen können erst nach einer gewissen Laufzeit festegelegt werden
%\item Export der entwickelten Überwachung auf den produktiven Haupt-Nagios Server
%\end{itemize}