\subsection{Umsetzung der Funktionlitätstest}

Für die Ausführung der einfachen Funktionlitätstest aus Kapitel \ref{funztest} werden Benutzerinformationen zur Anmeldung benötigt.
Nagios besitzt extra hierfür die Möglichkeit diese Benutzerinformationen in Variablen zu speichern, damit sie nicht einzeln bei jeder Servicedefinition verändert werden müssen.
Da sich die Definition dieser Variablen in einer externen Datei befindet, können die Zugriffsrechte auf diese Datei eingeschränkt werden, wodurch die Anmeldedaten bei den Servicedefinitionen nicht auslesbar sind.

\begin{lstlisting}[captionpos=b, caption=Funktionalitätstest der Benutzeranmeldung, label=userauthdef, breaklines = true, language=sh]
#Anmeldung an Oracle UCM mit lokalem Benutzerkonto
define service{
        use                     generic-service
        host_name               example.kit.edu
        service_description     Anmeldung Oracle UCM als lokaler Benutzer
        check_command           check_http!-u "/bdb/idcplg?IdcService=LOGIN&Action=GetTemplatePage&Page=HOME\_PAGE&Auth=Internet"  -a $USER3$:$USER4$ -e "Sie sind angemeldet als" -S
        }
\end{lstlisting}


Dabei werden dem Nagios-Plugin \textit{check\_http} mit dem \pictext{u}-Parameter die URL zur Benuteranmeldungseite und mit dem Parameter \pictext{a} der benutzername und -passwort mitgegeben.
Der nach dem Parameter \pictext{e} folgende String wird dann in der Antwort des Servers gesucht.
Sollte dieser String nicht gefunden werden ist die Authentifizierung fehlgeschlagen und es wird durch Nagios eine Meldung versendet.
Mit dem Parameter \pictext{S} wird angegeben, dass eine \gls{SSL}-verschlüsselte Verbindung zum Webserver über HTTPS hergestellt werden soll, ansonsten würden die Benutzerinformationen im Klartext übertragen werden, wodurch sie leicht für Angreifer auslesbar wären.

Für das Auslesen von Informationen aus der Statusseite der Oracle UCM-Anwendung wurde ein einfaches BASH-Script entwickelt:

\begin{lstlisting}[captionpos=b, caption=Auslesen der Verbindungen zur Datenbank, label=dbcon, breaklines = true, language=sh]
#!/bin/bash
E_BADARGS=2
if [ ! -n "$6" ]
then
        echo "Usage: `basename $0` -url <URL> -u <username> -p <password>"
        exit $E_BADARGS
fi

DBCONNECTIONS=$(wget -qO-  --user $4 --password $6 $2 | grep "System Database")
DBCONNECTIONS=${DBCONNECTIONS##*>}
DBCONNECTIONS=${DBCONNECTIONS%% *}
echo $DBCONNECTIONS
\end{lstlisting}

Dabei muss als URL die Seite mit den Datenbankverbindungen und gültige Benutzerinformationen mitgegeben werden.
Anschliessend wird die aufgerufene Seite nach der gewünschte Informationen untersucht und ausgegeben.

Dieses einfaches Script kann auch dazu verwendet um andere Informationen von der Statusseite abzufragen.