\section{Abstract}

Diese Bachelorarbeit ermöglicht eine intensive Überwachung des Dokumenten-Management-Systems \gls{OracleUCM}\footnote{Oracle Universal-Content-Management} durch die Open Source-Software Nagios um proaktiv auftretende Fehler zu entdecken.
Dabei werden die Grundlagen von Dokumenten-Management-Systemen, Aufbau von \gls{OracleUCM} und Nagios beleuchtet und beschrieben.
Dadurch kann eine geeignete Methode aus den unterschiedlichen Überwachungsmethoden von Nagios ausgewählt werden.
Notwendige Kenntnisse über Service-orientierte Architektur (\gls{SOA}) und Web Services werden für die Umsetzung angeeignet.

Die Überwachung besteht aus den Ebenen: Statusabfragen, Funktionalitätstests, Auswertung von Logdateien und Benutzersimulation.
Auf dem Windows-Servers der \gls{OracleUCM}-Anwendung wird ein passender Nagios-Agent installiert, der aus einer vorherigen Übersicht ausgewählt wurde.
Die Konfiguration und der Einsatz von bereits erhältlichen Nagios-Plugins decken die ersten drei Ebenen ab.
Die automatisierte Benutzersimulation verwendet verschiedene Web Services der \gls{OracleUCM}-Anwendung.


%Ein Abstract ist eine prägnante Inhaltsangabe, ein Abriss ohne Interpretation und Wertung einer wissenschaftlichen Arbeit.

%Zusammenfassung von allem.

%Aufgabenstellung, Erwartendes Ergebnis
%\begin{itemize}
%\item Objektivität: soll sich jeder persönlichen Wertung enthalten
%\item Kürze: soll so kurz wie möglich sein
%\item Verständlichkeit: klare, nachvollziehbare Sprache und Struktur
%\item Vollständigkeit: alle wesentlichen Sachverhalte sollen explizit enthalten sein
%\item Genauigkeit: soll genau die Inhalte und die Meinung der Originalarbeit wiedergeben
%\end{itemize}
