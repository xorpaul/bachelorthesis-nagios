\begin{itemize}
\item Kurz agenten, zeigen auf f. Kapitel -> SNMP erklären (MIB, OID) Sicherheitsrisiko
\end{itemize}

\subsection{Überwachungslogik (mit Alarmierung/Benachrichtigung)}
\label{dismoni}
\begin{center}
TODO: Distributed Monitoring bezug auf allg überwachungssysteme
\end{center}

\subsection{Plugins}
Gedacht für Linux umgebung

Verschiedene Möglichkeiten Checks zu realisieren unter Unix Systemen:

Leicht programmierbar -> perl 
Extra Plugins für Windows

\subsection{(Windows) Agenten oder allgemein Einholen von Infos}
Warum nicht einfach alles über SNMP? -> ODI muss man erst beantragen, hoher Aufwand und dann doch nicht so universell/alles abdeckend wie aktive checks, man kann keine logfiles durchuchen -> könnte es aber als standalone prog auf dem client laufen lassen und dieser sendet dann passive checks

Sagen das auf alten NSClient verzichtet wird und OpMon Agent nicht behandelt
\begin{enumerate}
\item Bilder ausm Nagios Buch Seite 472ff!
\item NSClient++
\item NC\_Net
\item NRPE\_NT
\end{enumerate}

Zusammenfassung?


Welche wird jetzt eingesetzt und warum?

Erwähne sichheitsstechnisch Parameter erlauben oder nicht erlauben

Dabei sagen, dass wenn nicht erlaubt sind keine zentrale Konfiguration der Checks auf dem Nagios server möglich ist -> abwägen

\subsection{Visualisierung der eingesammelten Daten}