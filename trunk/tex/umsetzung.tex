\section{Umsetzung}
In diesem Kapitel wird die Vorgehensweise der zuvor beschriebenen Problemstellungen erörtert.

\subsection{Überprüfen der Prozesse und Services}
\begin{itemize}
\item Prozesse
\item Services
\item Bsp Aufruf
\end{itemize}

\subsection{Einrichten des Windows Agenten}
\begin{itemize}
\item Port ändern -> RPC
\item Verschlüsselung durch PW und Algo
\item Hinzufügen der Plugins
\item Bsp Aufruf aktiver Check
\end{itemize}

\begin{lstlisting}[captionpos=b, caption=Aufruf eines aktiven Checks, label=activecheckexample, breaklines = true, language=bash]
root@iwrpaul:/usr/local/nagios/libexec# ./check_nc_net -H secret.kit.edu -p 123456 -s secret -v RUNSCRIPT -l check_uname.exe
Operating System OK - Microsoft(R) Windows(R) Server 2003 Standard Edition Service Pack 2
\end{lstlisting}

Das auf dem Nagios Server liegende Script \pictext{check\_nc\_net} stellt eine Verbindung zum angegebenen Server her und führt die mit dem Parameter \pictext{l} angegebene Datei aus. Dafür muss sich diese Datei in dem Script Verzeichnis des NC\_Net befinden.


