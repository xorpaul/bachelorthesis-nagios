\subsection{Überwachungssysteme}
\begin{center}
was ist wichtig was nicht, gewichtung, klassifizierung, organisationsstrategie
HOst,Services erklären
\end{center}

Geräte nicht nur Server bzw. Rechner, sondern auch Switches Router, oder auch explizite Hardwarekomponenten wie Sensoren für Temperatur, Luftfeuchtigkeit oder Rauchmelder.

\subsubsection{Ressourcenbelastung}
Die Einführung einer Überwachungssoftware bringt bei größeren Serverlandschaften eine nicht zu verachtende Netzwerk- und Prozessorbelastung mit sich.
Dabei gilt es die anfallende Belastung durch zwei unterschiedliche Arten der Überwachung zu unterscheiden:

\paragraph{Lokale / Zentrale Bearbeitung}
Die Durchführung der Überprüfungen findet durch einen zentralen Überwachungsserver statt, der die Informationen über die einzelnen Hosts und Services über das Netzwerk abfrägt /  abfragt.
Diese Methode ist in der Regel vorzuziehen, da hierbei die zu überwachenden Geräte weniger belastet werden und die Konfiguration der einzelnen Kontrollschritte zentral möglich / realisierbar ist.

\paragraph{Entfernte / Ausgelagerte Bearbeitung}
Bei einer sehr hohen Anzahl von zu überwachenden Objekten ist eine zentralisierte Ausführung nicht mehr von einem einzelnen Server tragbar.
In diesem Fall ist das Überwachungssystem darauf angewiesen, dass die einzelnen Hosts die kontrollierenden Überprüfungen selbständig durchführen und deren Ergebnisse an den Überwachungsserver weiterzuleiten.

\newline


Nagios bietet zusätzlich noch eine weitere, dritte Möglichkeit durch das \textit{Distributed Monitoring} (Verteilte Überwachung) an, siehe Kapitel \ref{dismoni}.

\subsubsection{Netzwerkstruktur und Abhängigkeiten}


\subsubsection{Sicherheitsaspekte}
\subsubsection{Port- versus Anwendungsüberwachung}
\begin{itemize}
\item E2E
\end{itemize}