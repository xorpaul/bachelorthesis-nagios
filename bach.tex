\documentclass[12pt, a4paper, headsepline]{article}
\usepackage[utf8x]{inputenc}
%\usepackage[latin1]{inputenc}
\usepackage[T1]{fontenc}
\usepackage[ngerman]{babel}
\usepackage{graphicx} 
\usepackage[automark]{scrpage2}
\usepackage{subfigure} 

\usepackage{ae} 
\pagestyle{scrheadings}
\usepackage{amsmath}
\usepackage{amsfonts}
\usepackage{amssymb}
\usepackage{floatflt}
\usepackage{pdfpages}
%\usepackage{fancyhd}
\usepackage{pdfpages}
\usepackage{listings}
 \usepackage{verbatim} %Mehrzeilige Kommentare durch \begin{comment} mglich  ??
\usepackage{blindtext} % Erzeugung von lngeren Textpassagen
\usepackage{xcolor}  
\usepackage{hyperref}
\definecolor{hellgrau}{rgb}{0.9,0.9,0.9}
\definecolor{darkgreen}{rgb}{0,0.5,0}
\definecolor{darkblue}{rgb}{0,0,2}
\lstset{inputencoding=utf8x, extendedchars=\true, numbers=left, numberstyle=\small, numbersep=5pt, basicstyle=\ttfamily\scriptsize, backgroundcolor=\color{hellgrau}, commentstyle=\color{darkgreen}, keywordstyle=\color{blue}, showspaces=false, showtabs=false, emph={String}, emphstyle=\color{blue}, emph={Boolean}, emphstyle=\color{blue}, showstringspaces=false}

\newcommand{\cappclause}[2]{
   \thispagestyle{empty}
   \
\begin{center}
\begin{Large}\textbf{Eidesstattliche Erklärung}\end{Large}
\end{center}   
   
   \vfill

Hiermit erkläre ich an Eides Statt, dass ich die vorliegende Arbeit selbst angefertigt habe; die aus fremden Quellen direkt oder indirekt übernommenen Gedanken sind als solche kenntlich gemacht. 


Die Arbeit wurde bisher keiner Prüfungsbehörde vorgelegt und auch noch nicht veröffentlicht.


   Ich versichere hiermit wahrheitsgemäß, die Arbeit bis auf die dem
   Aufgabensteller bereits bekannte Hilfe selbständig angefertigt, alle
   benutzten Hilfsmittel vollständig und genau angegeben und alles kenntlich
   gemacht zu haben, was aus Arbeiten anderer unverändert oder mit
   Abänderung entnommen wurde.\\

   \begin{center}
       \raggedright #2\\
       \vspace*{-2ex}
       \dotfill\\
       Ort, Datum \hfill (#1)\\
   \end{center}
}
\renewcommand{\baselinestretch}{1.0}

\author{Andreas Paul}
\title{Munin - Benachrichtigungsdienst und graphische Aufbereitung von Netzwerk- und Serviceüberwachunginformationen}

%\pagestyle{fancy}
%Funoten
\usepackage[bottom,hang]{footmisc}
\setlength{\footnotemargin}{0pt}
\addto{\captionsngerman}{\renewcommand*{\listfigurename}{}}
\makeatletter
\renewcommand*\l@section{\@dottedtocline{1}{1.5em} {2.3em}}
\renewcommand*\l@subsection{\@dottedtocline{2}{3.8 em}{3.2em}}
\renewcommand*\l@subsubsection{\@dottedtocline{3}{ 7.0em}{4.1em}}
\renewcommand*\l@paragraph{\@dottedtocline{4}{10em }{5em}}
\renewcommand*\l@subparagraph{\@dottedtocline{5}{1 2em}{6em}}
\renewcommand*\l@figure{\@dottedtocline{1}{1.5em}{ 2.3em}}
\newcommand{\pictext}[1]{\glqq\texttt{#1}\grqq }
%\newcommand{\brainmarker}[1]{\textbf{\textt{#1}}}
% Abb. anstatt Abbildung verwenden %
\renewcommand{\figurename}{Abb.}
% Tab. anstatt Tabelle verwenden %
\renewcommand{\tablename}{Tab.} 
\renewcommand\subitem{\@idxitem \hspace*{30\p@}}
 % Tiefe der Nummerierung einstellen %
\setcounter{secnumdepth}{3}
\setlength{\textheight}{620pt}
\makeatother 
\renewcommand{\refname}{Quellenverzeichnis} 
\linespread{1.50} %zeilenabstand 1.5}
\setlength{\headheight}{3.0\baselineskip}
\setlength{\fboxsep}{3mm}
\ihead{\includegraphics[width=0.06\textwidth]{bilder/fzk2.png}}
%\chead{\rightmark}
\chead{\leftmark}
\cfoot{Andreas Paul - Forschungszentrum Karlsruhe}
\begin{document}
\setlength{\parindent}{0mm}
\includepdf{Titelblatt.pdf}
%\maketitle
\thispagestyle{empty}
\newpage 
\renewcommand{\contentsname}{Inhalt}
\tableofcontents
\newpage
 \newcommand{\cappclause}[2]{
   \thispagestyle{empty}
   \
\begin{center}
\begin{Large}\textbf{Eidesstattliche Erklärung}\end{Large}
\end{center}   
   
   \vfill

Hiermit erkläre ich an Eides Statt, dass ich die vorliegende Arbeit selbst angefertigt habe; die aus fremden Quellen direkt oder indirekt übernommenen Gedanken sind als solche kenntlich gemacht. 


Die Arbeit wurde bisher keiner Prüfungsbehörde vorgelegt und auch noch nicht veröffentlicht.


   Ich versichere hiermit wahrheitsgemäß, die Arbeit bis auf die dem
   Aufgabensteller bereits bekannte Hilfe selbständig angefertigt, alle
   benutzten Hilfsmittel vollständig und genau angegeben und alles kenntlich
   gemacht zu haben, was aus Arbeiten anderer unverändert oder mit
   Abänderung entnommen wurde.\\

   \begin{center}
       \raggedright #2\\
       \vspace*{-2ex}
       \dotfill\\
       Ort, Datum \hfill (#1)\\
   \end{center}
}
\renewcommand{\baselinestretch}{1.0}
 
 \cappclause{Andreas Paul}          % author name
        {Karlsruhe, den \today}    % location, date (for legal clause)
\newpage
\section{Einleitung}

%Mit dem Zusammenschluss......ist eine Einrichtung mit ??? Angestellten, ??? Studierenden und ca. 300 externen Mitarbeitern und Gästen entstanden. Die IT-Infrastruktur für den organisatorischen und wissenschaftlichen Betrieb liegt in der Verantwortung des ...(SCC)...das aus der Verschmelzung von ...und...hervorgegangen ist. Für alle Schichten der IT-Infrastruktur und alle angebotenen Dienstleistungen muss der Betrieb duch das Rechenzentrum überwacht werden.

%Die Überwachung des Dokumentenmanagmentsystem, eines wichtigen zentralen Dienstes, war Ziel dieser Arbeit.



%8000 Wissenschaftler und Mitarbeiter, 18000 Studierende mit
%einem Jahresbudget von 0.5 Mrd €


Mit dem Zusammenschluss des Forschungszentrum Karlsruhe und der Universität Karlsruhe (TH) zum Karlsruhe Insitute of Technology (KIT) ist eine Einrichtung mit 8000 Wissenschaftlern und Mitarbeitern, 18000 Studierenden und circa 300 externen Mitarbeitern und Gästen entstanden.

%IWR FZK + RZ Uni

Die IT-Infrastruktur für den organisatorischen und wissenschaftlichen Betrieb liegt in der Verantwortung des Steinbuch Center für Computing (SCC), das aus der Verschmelzung des Rechenzentrums der Universität und dem Institut für Wissenschaftliches Rechnen (IWR) hervorgegangen ist.

Für alle Schichten der IT-Infrastruktur und alle angebotenen Dienstleistungen muss der Betrieb durch das Rechenzentrum überwacht werden.
Die Überwachung des Dokumenten-Management-System (\gls{DMS}), einer zentralen Dienstleistung, ist Ziel dieser Arbeit.


%In Unternehmen werden den Benutzern verschiedene IT-Dienstleistungen angeboten.
%Eine Dienstleistung ist die Bereitstellung einer Plattform für die zentrale Speicherung, Bearbeitung und Verwaltung von Dokumenten.

Die Hauptaufgaben eines Dokumenten-Management-Systems sind die zentrale Speicherung, Bearbeitung und Verwaltung von Dokumenten.
Dabei können diese Dokumente Dateien in unterschiedlicher Form sein wie Microsoft Word Dateien, Excel Tabellen, Dateien im Portable Document Format (\gls{PDF}) oder auch Bilder.
%Die wichtigsten Funktionen

%%Danach kannst Du mit dem Abschnitt ...Eine Dienstleistung ist die Bereitstellung .....in umgeschriebener Form weitermachen. Punkte aus 4.2 aber nicht schon detailliert erläutern, sondern anreissen und eher darauf hinweisen wie wichtig solche Funktionen, wie Revisionierung z.B. für Patentverträge sind. Bei der Beschreibung von "Überwachung" hast Du auch den Nutzen für den RZ-Betrieb und den Kunden in den Vordergrund der "einleitenden Worte" gestellt.

%Die Darbietung / Die Versorgung / Das Angebot / Das Bereitstellen einer solchen Dienstleistung wird mit einem Dokumenten-Management-System (\gls{DMS}) realisiert.
%Vorteile, die für den Einsatz eines Dokumenten-Management-Systems sprechen, sind die Möglichkeiten, die sich durch die computergestützte Erfassung und Indexierung (auch Indizierung genannt) der Dokumente eröffnen.
%Die bekannteste / geläufigste Implementierung dieser Möglichkeiten ist die automatische Verschlagwortung von Dokumenten für die Zuordnung von Deskriptoren zu einem Dokument zur Erschließung der darin enthaltenen Sachverhalte.
%Durch die Aufnahme dieser Schlagwörter in einen Suchindex können Anwender bestimmte Dokumenten gezielt finden und anfordern.
%Ein wichtiger Punkt ist die Versionierung der Dateien in einem Dokumenten-Management-System.
%Dadurch können auf ältere Versionen der Dokumente zugegriffen werden, Änderungen angezeigt oder (komplett) zurückgesetzt werden.
%Durch die zentrale Struktur / Zugriff eines Dokumenten-Management-Systems ist es notwendig Zugriffsrichtlinien für die Dokumenten zu implementieren, die anhand von Gruppen- und Benutzerinformationen gesetzt / ausgewählt werden.

Aufgrund der Vielzahl an angebotenen Dienstleistungen ist es schwierig herauszufinden, ob die angebotenen Dienstleistungen noch fehlerfrei arbeiten oder aus welchem Grund die Benutzer nicht mehr auf einen Dienst zugreifen können.
Für diesen Zweck wurden Überwachungssysteme entwickelt, die den Status von verschiedenen Komponenten und den davon abhängigen Diensten überwachen und bei Veränderungen die Verantwortlichen darüber informieren.

Für einen möglichst störungsfreien Betrieb ist es notwendig, dass die Ergebnisse der Überwachung in periodischen Zeitabständen erneuert werden, damit ein auftretendes Problem schnellstmöglich erkannt und behoben werden kann.
Das Überwachungssystem soll so implementiert werden, dass Fehler erkannt werden, bevor die Nutzung der angebotenen Dienstleistungen davon beeinträchtigt wird.
Dabei muss die zusätzliche Belastung der Netzwerkes und der überwachten Objekte durch die Überwachung eingeplant, die verwendete Netzwerkstruktur und die dadurch entstehende Abhängigkeit (von Netzwerkknoten) beachtet und sicherheitstechnische Aspekte einer automatischen Überwachung bedacht werden.\\


Im Laufe dieser Arbeit soll eine Überwachung eines Dokumenten-Management-Systems unter Berücksichtigung der Funktions- und Arbeitsweise des eingesetzten Dokumenten-Management-Systems durch eine Open Source Überwachungsanwendung realisiert werden.
%popular open source computer system and network monitoring software application 
 

%Einleitung halt. Test
%Kurz was ist Nagios, warum überhaupt überwachen?
%Was soll überwacht werden -> Stellent/UCM kurz was ist das? Warum gerade das überwachen -> Aktive Benutzung durch User - kritisch 
\newpage
\section{Abstract}


Dokumenten-Management-Systeme bilden eine zentralen Dienstleistung im Karlsruhe Insitute of Technology.
Diese Systeme sind komplex aufgebaut und benötigen ausgefeilte Kontrollmaßnahmen / Überwachungsroutinen um einen stabilen Betrieb zu garantieren / ermöglichen.
Zum gegenwärtigen Zeitpunkt gibt es keine ausgereifte Überwachungssoftware, die diese Aufgabe zufriedenstellend erfüllt.

Diese Bachelorarbeit beschreibt die Entwicklung von neuen Werkzeugen für die Open Source-Überwachungssoftware Nagios um das Dokumenten-Management-System \gls{OracleUCM}\footnote{Oracle Universal-Content-Management} auf Fehlverhalten hin zu kontrollieren.
Diese sogenannten Plugins lassen sich in bestehende Nagios-basierende Systeme einbinden und erweitern deren Bandbreite an zu überwachenden Elementen.
Da Hauptaugenmerk lag dabei, auf der Simulation von Benutzerverhalten und der Erkennung der dabei auftretenden Fehler.
Die Verantwortlichen sollen durch die Benachrichtigung dieser Fehler sofort alarmiert werden und durch die unterschiedliche Tiefe der Überwachungstests die Problemquellen eingrenzen können.
%, welche möglichst zu Laufzeit erkannt und deren Fehl
%Meldungen können die Problemquellen vom Administrator gefunden und eventuell behoben werden,
%Diese Bachelorarbeit beschäftigt sich mit der Entwicklung eines Plugins für die Open Source-Überwachungsoftware Nagios intensiven Überwachung des Dokumenten-Management-Systems \gls{OracleUCM}\footnote{Oracle Universal-Content-Management} durch  um proaktiv auftretende Fehler zu entdecken.
%Dabei werden die Grundlagen von Dokumenten-Management-Systemen, Aufbau von \gls{OracleUCM} und Nagios beleuchtet und beschrieben.
%Dadurch kann eine geeignete Methode aus den unterschiedlichen Überwachungsmethoden von Nagios ausgewählt werden.
%Notwendige Kenntnisse über Service-orientierte Architektur (\gls{SOA}) und Web Services werden für die Umsetzung angeeignet.

Die Überwachung besteht aus den Ebenen: Statusabfragen, Funktionalitätstests, Auswertung von Logdateien und Benutzersimulation.
Auf dem Windows-Servers der \gls{OracleUCM}-Anwendung wird ein passender Nagios-Agent installiert, der aus einer vorherigen Übersicht ausgewählt wurde.
Die Konfiguration und der Einsatz von bereits erhältlichen Nagios-Plugins decken die ersten drei Ebenen ab.
Die automatisierte Benutzersimulation verwendet verschiedene Web Services der \gls{OracleUCM}-Anwendung.


%Ein Abstract ist eine prägnante Inhaltsangabe, ein Abriss ohne Interpretation und Wertung einer wissenschaftlichen Arbeit.

%Zusammenfassung von allem.

%Aufgabenstellung, Erwartendes Ergebnis
%\begin{itemize}
%\item Objektivität: soll sich jeder persönlichen Wertung enthalten
%\item Kürze: soll so kurz wie möglich sein
%\item Verständlichkeit: klare, nachvollziehbare Sprache und Struktur
%\item Vollständigkeit: alle wesentlichen Sachverhalte sollen explizit enthalten sein
%\item Genauigkeit: soll genau die Inhalte und die Meinung der Originalarbeit wiedergeben
%\end{itemize}
 \newpage
\section{Aufgabenstellung}

Um den Mitarbeitern des Karlsruhe Insitute of Technology eine möglichst ausfallsichere Plattform für die zentrale Speicherung, Bearbeitung und Verwaltung von Dokumenten anbieten zu können, soll eine Überwachung implementiert werden.
Diese Überwachung soll nicht nur die Anwendung, sondern auch den darunterliegenden Server bezüglich seiner Systemressourcen berücksichtigen.
Dabei müssen Überwachungselemente gefunden werden, mit deren Überprüfung der eindeutige Zustand der Anwendung festgestellt und der störungsfreie Betrieb sichergestellt werden kann.

Für die Verwaltung von Webseiten, Dokumenten und Bildern wird das Dokumenten"=Management"=System \gls{OracleUCM} der Firma Oracle eingesetzt.
Um die zu überwachenden Objekte zu ermitteln, ist das Verständnis über den Aufbau und der spezifischen Funktions- und Arbeitsweise des verwendeten Dokumenten-Management-Systems notwendig.

Als Überwachungssoftware wird die Open Source"=Software Nagios verwendet.
%Damit der fehlerfreie Betrieb von \gls{OracleUCM} als Dienst durch die Überwachung der ermittelten Überwachungselemente eindeutig festgestellt werden kann, muss sich mit dem Aufbau, der internen Funktionsweise und den verschiedenen Methoden bezüglich der Ermittlung der Statusinformationen untersucht werden.
Zur Realisierung der Überwachung muss auf die interne Logik und auf die verschiedenen Methoden bezüglich der Ermittlung der Statusinformationen eingegangen werden.
Dabei soll eine Übersicht über die unterschiedlichen Überwachungsmethoden von Nagios erstellt und unter Berücksichtigung des späteren Einsatzes bewertet werden.
%Hierbei sind für die spätere Umsetzung beispielsweise die verschlüsselte Datenübertragung zwischen Überwachungs- und Anwendungsserver ein Kriterium.
Mit den ausgewählten Methoden soll die Überwachung auf verschiedenen Ebenen realisiert werden.

Die Klassifizierung der Überwachungselemente ergibt sich aus der Gewichtung der einzelnen Elemente.
%In diesem Schritt werden die zuvor gefunden 
%Die Erreichbarkeit über das Netzwerk bildet die Grundlage für darüber liegende Überwachungsobjekte, wie beispielsweise der Zustand eines Prozesses.
Dabei soll die Anwendung auch reaktiv durch eine Auswertung von Logdateien auf Fehler überwacht werden.


Zur eindeutigen Erkennung von Fehlern, die während der Benutzung durch Anwender auftreten, sollen die typischen Aktionen der Benutzer simuliert werden. 
Für die Realisierung dieser Benutzersimulation muss die Anwendung über eine Schnittstelle verfügen, die sich durch ein Programm über das Netzwerk ansprechen lässt.
Dieses Programm soll die Benutzeraktionen automatisiert durchführen und der Überwachungssoftware Nagios die Ergebnisse der einzelnen Schritte übermitteln, damit der Fehlerzustand sofort erkannt und gleichzeitig seine Ursache eingegrenzt werden kann.

Dabei müssen bei der Programmentwicklung mögliche Konsequenzen aufgrund verschiedener Szenarien bedacht werden.
Sollte die Anwendung bereits durch eine Vielzahl von Benutzern stark belastet sein, wird dadurch auch der Ablauf der Benutzersimulation verzögert.
%In diesem Fall soll die Überwachungssoftware bzw. Benutzersimulation keine falsche Informationen melden.
Eine solche Verzögerung soll von der Überwachungssoftware bzw. Benutzersimulation bei der Auswertung berücksichtigt werden.

Die Nutzung der Anwendung durch die eigentlichen Benutzer darf dabei nicht beeinträchtigt werden.
Da die Ausführung der Benutzersimulation durch Nagios in kurzen Zeitabständen periodisch aufgerufen wird, müssen auch langfristige Auswirkungen wie das Überlaufen der Datenbank der Anwendung oder die Überfüllung des Festplattenspeichers des Anwendungsservers bedacht werden.

Für die Entwicklungsumgebung wird ein eigener Nagios-Server eingesetzt, deshalb muss die entwickelte Lösung auf den bereits vorhanden Nagios-Server exportierbar sein.

%\textit{Export auf vorhanden Nagios-Server ermöglichen.

%Hinzufügen von Services ermöglichen%
%Konfigurieren der Überwachungsparameter ermöglichen (CPU last, df) protokollieren} \newpage
\section{Grundlagen}
In diesem Kapitel werden die grundlegenden Begriffe erläutert, die für das Verständnis der weiterführenden Kapitel notwendig sind.

\subsection{Überwachungssysteme}
\label{monitor}
Überwachungssysteme wurden für den Zweck entwickelt den Status von verschiedenen Objekten meist über das Netwerk zu überwachen und im Falle einer Statusänderung diese Information an die zugewiesenen Kontaktpersonen weiterleitet.

%Bei diesen Objekten kann es sich um viele verschiedene Komponenten handeln.
Generell unterscheidet man zwischen der Überwachung ermöglichten zu Grunde liegenden Hardware den so genannten Hosts und den auf diesen Hardwarekomponenten aufsitzenden Diensten auch Services genannt.

Unter Hosts fallen nicht nur Server bzw. Computer, sondern auch Switches, Router oder auch dedizierte Überwachungshardware wie Sensoren für Temperatur, Luftfeuchtigkeit oder Rauchmelder.
Die Services dieser Hosts weichen je nach Art der Hosts stark voneinander ab.
Auf einem Server kann als Service ein Webserver im Betrieb sein, dessen Funktionalität sich über einen Aufruf einer Webseite überprüfen lässt.
Bei einem Switch können beispielsweise als Service die Übertragungsrate, der Paketverlust oder der Portzustand überwacht werden.

Sehr wichtig ist bei einem Überwachungssystem die Gewichtung der erhaltenen Überwachungsinformationen.


%\begin{center}
%was ist wichtig was nicht, Gewichtung, Klassifizierung, Organisationsstrategie
%Host,Services erklären
%\end{center}
\newpage
Vor der Einführung eines Überwachungssystems muss sich mit den folgenden Punkten auseinandergesetzt werden.

\subsubsection{Ressourcenbelastung}
Die Einführung einer Überwachungssoftware bringt bei größeren Serverlandschaften eine nicht zu verachtende Netzwerk- und Prozessorbelastung mit sich.
Dabei unterscheidet Josephsen die anfallende Belastung in zwei unterschiedliche Arten der Überwachung\footnote{Quelle: \cite{Jose07} S. 4}:

\paragraph{Zentralisierte Überwachung}
Die Durchführung der Überprüfungen findet durch einen zentralen Überwachungsserver statt, der die Informationen über die einzelnen Hosts und Services über das Netzwerk abfragt.
Diese Methode ist in der Regel vorzuziehen, da hierbei die zu überwachenden Geräte weniger belastet werden und die Konfiguration der einzelnen Kontrollschritte zentral möglich ist.

\begin{figure}[ht]
	\centering
	   \fbox{\includegraphics[width=0.5\textwidth]{bilder/dist_mon12.png}}
	   %\fbox{Quelle: \cite{Jose07} S. 5}
		\caption[Zentralistische Bearbeitung]{Zentralistische Bearbeitung}
		\label{distmon2}
\end{figure}
%\footnotetext{Quelle: \url{http://www.nagios-wiki.de/\_media/nagios/howtos/dist\_mon.png}}


\paragraph{Dezentralisierte Überwachung}
Bei einer sehr hohen Anzahl von zu überwachenden Objekten ist eine zentralisierte Ausführung nicht mehr von einem einzelnen Server tragbar.
In diesem Fall ist das Überwachungssystem darauf angewiesen, dass die einzelnen Hosts die kontrollierenden Überprüfungen selbständig durchführen und deren Ergebnisse an den Überwachungsserver weiterzuleiten.

%Kascadierende Überwachungssysteme kannst Du noch erwähnen

%Uberwachungsredundanzen vermeiden: Dein Beispiel mit dem Port führt in beiden Fällen dazu, dass der Test 1 überflüssig ist. Wenn die Webseite über 2 Ports abgefragt wird, z.B. 8000 Intranet, 8080 Internet, so wäre der Test 1 sinnvoll
\begin{figure}[ht]
	\centering
	   \fbox{\includegraphics[width=0.45\textwidth]{bilder/dist_monp.png}}
	   %\fbox{Quelle: \cite{Jose07} S. 5}
		\caption{Ausgelagerte Bearbeitung}
		\label{distmonp}
\end{figure}

Um nicht komplett vom Überwachungsserver abhängig zu sein, kann ein zweiter Überwachungsserver bzw. weitere Server hinzugefügt werden.
Diese können bei einem Ausfall des Hauptüberwachungsservers die Verantwortlichen informieren oder die zu überwachenden Objekte zur Lastenteilung untereinander aufteilen.


%Nagios bietet zusätzlich noch eine weitere, dritte Möglichkeit durch das \textit{Distributed Monitoring} (Verteilte Überwachung) an, siehe Kapitel \ref{dismoni}.

\subsubsection{Netzwerkstruktur und Abhängigkeiten}
Die Überwachung von Hosts und Services über das Netzwerk erzeugt normalerweise immer zusätzlichen \gls{IP}-Traffic.
Das bedeutet, dass jede Überquerung weiterer Netzwerkknoten, die zwischen dem Überwachungsserver und den zu überwachenden Geräten liegen, eine weitere Belastung für das Netzwerk bedeutet, sowie eine Abhängigkeit zwischen Host und Server einführt.

\begin{figure}[ht]
	\centering
	   \fbox{\includegraphics[width=0.5\textwidth]{bilder/dependent.png}}
	   %\fbox{Quelle: \cite{Jose07} S. 5}
		\caption[Zusätzliche Netzwerkabhängigkeit und Netzwerkbelastung]{Zusätzliche Netzwerkabhängigkeit und Netzwerkbelastung\protect\footnote}
		\label{depend}
\end{figure}
\footnotetext{Quelle: \cite{Jose07} S. 5}
\newpage
In der Abbildung \ref{depend} erzeugt der Router 1 die zuvor beschriebene zusätzliche Netzwerkabhängigkeit und Netzwerkbelastung, da der Server 1 bei einem Ausfall des Routers nicht mehr durch den Überwachungsserver erreichbar ist und jede Überprüfung, die vom Überwachungsserver gesendet wird den Router mit dem Routing der Pakete belastet.

Deshalb gilt es laut \cite{Jose07} S. 5 folgende zwei Punkte beim Erstellen eines Überwachungssystems zu beachten:

\paragraph{Überwachungsredundanzen vermeiden}
Redundante Überwachung entsteht dadurch, dass der gleiche Service durch zwei Arten in unterschiedlicher Tiefe geprüft wird.
Ein einfaches Beispiel ist die Überwachung eines Webservers auf dem Standardport 80.
Eine Überwachungsmethode ist es diesen Port abzufragen und die entsprechende Rückantwort des Servers auszuwerten.
Als zweiter Test soll die auf dem Webserver laufende Webseite überwacht werden. 
Dafür kann die jeweilige Webseite über die Adresse nach einem bestimmten Inhalt untersucht werden.

In beiden Fällen wird getestet, ob der Webserver über das Netzwerk ansprechbar ist, jedoch sagt der zweite Test zusätzlich noch aus, dass die korrekte Webseite angezeigt wird, somit wäre der erste Test überflüssig.
Jedoch muss zuvor abgewogen werden, ob eine redundante Überwachung nicht sogar hilfreich bei der Ermittlung der Fehlerursache ist.
%Wenn im oberen Beispiel der Inhalt der überwachten Webseite verändert wird, so können beide Tests die Fehlerursache eingrenzen.
So können beide Tests die Fehlerursache eingrenzen.
Der erste Test überprüft, ob der Webserver erreichbar ist und der zweite Test kann erkennen, ob eine falsche bzw. veraltete Seite ausgeliefert wird.
%, dass der Webserver einwandfrei funktioniert.

\paragraph{Minimale Netzwerkbelastung}
Um bereits stark belastete Netzwerkpunkte zu entlasten, bietet es sich an, die Frequenz mit der die Test über das Netzwerk gesendet werden zu verringern.
Die Aufstellung des Überwachungsservers ist dadurch gerade bei größeren Serverlandschaften sehr wichtig, da durch eine effiziente Platzierung womögliche Flaschenhälse  in Form von veralteten Switches oder ähnlichem vermieden werden können.

\subsubsection{Sicherheitsaspekte}
Um erweiterte Statusinformationen über einen Prozess oder über die Arbeitsspeicherauslastung auszulesen, ist zusätzliche Software auf den Hosts nötig.
Diese Software benötigt oft einen zusätzlichen geöffneten Port auf dem zu überwachendem Rechner, die einen neuen Angriffspunkt für Angreifer darstellen kann.
Außerdem erhält der Überwachungsserver Ausführungsrechte auf dem Client, so dass eine weitere potentielle Sicherheitslücke in einem vermeintlich zuvor sicherem System entsteht.
Jeder, der die Kontrolle über den Überwachungsserver besitzt oder sich als solcher ausgibt, kontrolliert somit gleichzeitig alle anderen überwachten Hosts.

Um dies zu verhindern gibt es verschiedene Ansätze.
Als ersten Ansatz sollte der Port durch den der Überwachungsserver mit dem Host kommuniziert vom Standardwert abweichen, damit nicht sofort erkennbar ist, dass sich eine angreifbare Überwachungssoftware auf dem Rechner befindet.\label{changeport}
Damit die über diesen Port versendeten Informationen nicht für Dritte zugänglich sind, bietet es sich an die auszutauschende Informationen mit einem Algorithmus zu verschlüsseln.
Durch den Einsatz eines Verschlüsselungsalgorithmus werden die Informationen nicht mehr im Klartext ausgetauscht, sondern
Da die Möglichkeit einer Verschlüsselung der Datenübertragung nicht von jeder Überwachungssoftware angeboten wird, gilt diese Option als Auswahlkriterium in der späteren Umsetzung bzw. im produktivem Betrieb.

Des weiteren sollte die Erlaubnis der Abfrage der Überwachungsinformationen anhand der \gls{IP}-Adresse eingeschränkt werden, so dass der Client nur Anfragen des Überwachungsservers akzeptiert.
Durch diese Einschränkung kann vermieden werden, dass sensible Informationen aus den Antworten an unberechtigte Dritte übermittelt werden oder ein Denial of Service-Angriff (\gls{DoS}) durch eine übermäßig hohe Anzahl an Anfragen an den Client gesendet wird, um eine Überlastung des Servers zu erreichen und diesen somit arbeitsunfähig zu machen.

%\begin{itemize}
%\item Verschlüsselung der Informationen, die zwischen dem Server und dem Host hin- und hergesendet werden, damit man nicht die Inforamtionen im Klartext einfach auslesen kann.
%\item Firewall regeln, dazu Bild aus dem Jose07 Buch S9
%\end{itemize}
%\subsubsection{Port- versus Anwendungsüberwachung}
%\begin{itemize}
%\item E2E
%\end{itemize}






























\subsection{Dokumenten-Management-Systeme}

Um ein Dokumenten-Management-System (\gls{DMS})  zu erläutern muss sich zuerst mit dem Begriff des \textbf{"`Dokuments"'} auseinander gesetzt werden.
In \cite{DMS08} S. 2 wird ein Dokument durch folgende Punkte definiert:

\begin{itemize}
\item Ein Dokument fasst inhaltlich zusammengehörende Informationen strukturiert zusammen, die nicht ohne erheblichen Bedeutungsverlust weiter unterteilt werden können. 
\item Die Gesamtheit der Information ist für einen gewissen Zeitraum zu erhalten.
\item Dokumente dienen oft dem Nachweis von Tatsachen.
\item Ein Dokument ist als Einheit ablegbar (speicherbar) und/oder versendbar und/oder wahrnehmbar (sehen, hören, fühlen).
\item Das Dokument ist eigentlich der Träger, der die Informationen speichert, egal ob das Dokument ein Stück Paper, eine Datei auf einem Rechner, ein Videoband oder eine Tontafel etc. ist. Dies bedeutet auch, dass es keine Bindung an Papier oder ein geschriebenes Wort gibt.
\end{itemize}

Desweiteren gibt es eine Differenzierung in zwei Definitionen:

\begin{quote}"`Als \textbf{Dokument im konventionellen Sinne} werden Dokumente bezeichnet, die als körperliches Dokumente (z. B. Papier) vorliegen, ursprünglich als körperliches Dokument vorlagen oder für die Publizierung auf einem körperlichen Medium vorgesehen sind.

Die Begrifflichkeit des \textbf{Dokuments im weiteren Sinne} erweitert den Begriff des Dokuments um semantisch zusammengehörende Informationsbestände , die für die Publikation in nicht-körperlichen Medien, z. B. Webseiten, Radio, Fernsehen o. ä. vorgesehen sind. Derartige Dokumente werden oft dynamisch gestaltet und zusammengestellt."' \begin{flushright}\cite{DMS08} S. 2\end{flushright}\end{quote}

Unter \textbf{Dokumenten-Management} werden primär die Verwaltungsfunktionen Erfassung, Bearbeitung, Verwaltung und Speicherung von Dokumenten verstanden. \cite{DMS08} S. 344.

Darunter fallen laut \cite{DMS08} S. 3 folgende Punkte:

\begin{itemize}
\item Kennzeichnung und Beschreibung von Dokumenten (auch Metadaten des Dokuments genannt) 
\item Fortschreibung, Versionierung und Historienverwaltung von Dokumenten
\item Ablage und Archivierung von Dokumenten
\item Verteilung und Umlauf von Dokumenten
\item Suche nach Dokumenten bzw. Dokumenteninhalten
\item Schutz der Dokumente vor Verfälschung, Missbrauch und Vernichtung
\item Langfristiger Zugriff auf die Dokumente und Lesbarkeit der Dokumente
\item Lebenslauf und Vernichtung von Dokumenten
\item Regelung von Verantwortlichkeiten für Inhalt und Verwaltung von Dokumenten
\end{itemize}

Der Begriff \textbf{"`Dokumenten-Management-System"'} muss auch in zwei verschiedene Sichtweisen differenziert werden:
\begin{quote}"`Bei \textbf{Dokumenten-Management-Systemen im engeren Sinne} geht es um die Logik der Verwaltung von Dokumenten, deren Status, Struktur, Lebenzyklus und Inhalt. Dokumente werden beschrieben, klassifiziert und in einer bestimmten logischen Struktur eingeordnet, damit sie einfach wieder gefunden werden können. Dokumente entstehen, werden verändert und (irgendwann) vernichtet.

Den \textbf{Dokumenten-Management-Systemen im weiteren Sinne} ordnet man auch noch weitere Funktionalitäten zu, wie z. B. Schrifterkennung, automatische Indizierung, [...], Publizierung. Hier lassen sich die Grenzen nicht mehr genau bestimmten!"' \begin{flushright}\cite{DMS08} S. 5\end{flushright}\end{quote}

\subsection{Content-Management-Systeme}
Bei einem Content-Management-System (\gls{CMS})  steht nicht mehr das eigentliche Dokument im Vordergrund, sondern vielmehr der enthaltene Informationsgehalt des Dokuments.
Der Unterschied zwischen einem DMS und einem CMS besteht laut \cite{DMS08} S. 114 im/in Folgenden/m:

\begin{quote}"`Abgrenzend zum Dokumenten-Management handelt es sich beim Content-Management nicht vordergründig um die Verwaltung von Dokumenten, sondern um die Verwaltung von Informationseinheiten, die miteinander verknüpft sein können. [...] Je nach ausprägung kann nun ein konkretes System als Dokumenten-Management-System mit Content-Management-Funktionen definiert werden und umgekehrt. [...]
Der Ansatz des Content-Management unterscheidet sich vom "`klassischen"' Dokumenten-Mangement vor allem in Bezug auf die betrachteten Objekte: Ein DMS hat als kleinestes Objekt der Betrachtung eines einzelnen Dokument. [...] Content-Management ist auf logische Informationseinheiten ausgerichtet. Es ist z.B. das Ziel des Content-Managements, Inhalte, die auf mehrere Quellen verteilt sind, neue zusammenzustellen und daraus z.B. ein neues Dokument zu generieren."'
\begin{flushright}\cite{DMS08} S. 114f\end{flushright}\end{quote}

Die folgende Abbildung soll den (charakteristischen) Unterschied zwischen CMS-Systemen und DMS-Systemen verdeutlichen.

\begin{center}
Bild \cite{DMS08} S. 115
\end{center}

Wie zuvor beschrieben ist die Sichtweise eines DMS nur auf die einzelnen Dokumente beschränkt, während ein CMS einzelne Elemente / Informationen aus den Dokumenten extrahieren und daraus ggf. ein neues Dokument generieren kann. Die Sichtweise des CMS wird durch das gestrichelte Polygon dargestellt, welches hier dokumentenübergreifend abgebildet ist.

Der (theoretische/beabsichtigte) Zweck, weshalb ein CMS-System eingesetzt wird, ist laut Oracle folgendermaßen definiert:

\begin{quote}"`The key to a successful content management implementation is unlocking the value of content by making it as easy as possible for it to be consumed. This means that any piece of content must be available to any consumer, no matter what their method of access."'
\begin{flushright}\cite{UCM07} S. 12\end{flushright}\end{quote}

Ein CMS soll die Informationen jedes/jedwedem (Inhalts) extrahieren/aufnehmen und jedes Einzelteil / Element dieser Information den Benutzern zugänglich machen, unabhängig von der Art des Zugriffs.
Dieses Konzept soll in Abbildung \ref{ucm-a2a} verdeutlicht werden.

\begin{figure}[ht]
	\centering
	   \fbox{\includegraphics[width=0.95\textwidth]{bilder/ucm.png}}
		\caption["`any-to-any"' Content-Management Konzept]{"`any-to-any"' Content-Management Konzept\protect\footnote}
		\label{ucm-a2a}
\end{figure}
\footnotetext{Quelle: \cite{UCM07} S. 12}

Das CMS steht hier in der Mitte der Abbildung als Medium zwischen den verschiedenen Inhalten, eingestellt von den \textit{Contributors} (links), und den Anwendern, die auf transformierte Versionen der Inhalte durch unterschiedliche Arten zugreifen (rechts).


\subsection{Enterprise-Content-Management-Systeme - optional}
In diesem Zusammenhang / Kontext sei auch der Begriff Enterprise-Content-Management (\gls{ECM}) genannt.
Laut der "`Association for Information and Image Management"' (\gls{AIIM}\footnote{Die AIIM ist eine Gesellschaft von internationalen Herstellern und Anwendern von Informations- und Dokumenten-Mangement-Systemen}), welche sich mit umfasst dieser Begriff die Verwaltungfunktionen von Unternehmensinformationen in unterschiedlichen Dokumentformaten.\footnote{Quelle: \url{http://www.aiim.org/What-is-ECM-Enterprise-Content-Management.aspx}}
Diese Funktionen werden laut \cite{DMS08} S. 116 durch verschiedene "`Systeme wie Dokumenten-Management, Groupware, Workflow, Input- und Output-Management, (Web-)Content-management, Archivierung, Records-Management und andere"' bereitsgestellt.


\subsection{Universal-Content-Management-Systeme}



























\subsection{[cronjobs]}

%\subsection{[Farbraum]}
\subsection{[ds - ADS Benutzer]}
\subsection{[Metadaten allg]}
\subsection{[alse+-true+-]}
%\subsection{[evtl Oracle DB]}







 \newpage
\input{tex/architektur.tex} \newpage
\section{Umsetzung}
In diesem Kapitel wird die Vorgehensweise der zuvor beschriebenen Problemstellungen erörtert.

\subsection{Aufbau der Testumgebung}

Die für die Umsetzung notwendigen Ressourcen in Form eines Test-Servers und einer virtuellen Maschine.
\paragraph{Aufsetzen eines Nagios-Test-Systems}
Da die einzelnen Überwachungselemente in der Überwachungssoftware Nagios sukzessiv eingetragen werden müssen, ist ein häufiges Neustarten der Nagios-Anwendung notwendig, damit die neuen Konfigurationsdateien übernommen werden.

Damit dies nicht auf dem bereits verwendetem Nagios-Server durchgeführt werden muss, wird ein Nagios-Testserver für diesen Zweck eingesetzt.

Da Nagios ein Unix-ähnliches Betriebssystem erfordert, wird für diesen Zweck die Linux-Distribution Debian als Betriebssystem des Testservers verwendet.
Diese Distribution wird auch auf den Produktivservern des KIT verwendet.

\paragraph{Bilddatenbank als virtuelle Maschine}
Für die Simulation der verschiedenen Fehlerzuständen der einzelnen Überwachungselemente wird eine virtuelle Maschine mit einer \gls{OracleUCM} Prototypinstallation  als Entwicklungsplattform verwendet.

\subsection{Auswahl der geeigneten Überwachungsmethode}
\label{sectunixagents}
Wie in Kapitel \ref{methoden} aufgeführt, bietet Nagios verschiedene Möglichkeiten Informationen über zu überwachende Objekte zu sammeln.

\begin{itemize}
\item Überwachung direkt über das Netzwerk
\item Ausführung der Plugins durch \gls{SSH}-Verbindung
\item Ausführung von selbst vorkonfigurierten Kommandos durch \gls{NRPE}
\item Abfrage von Informationen durch \gls{SNMP}
\item Passiver Erhalt der Ergebnisse durch \gls{NSCA}
\end{itemize}

Für Dienste, die sich über das Netzwerk erreichen lassen, können die dafür entwickelten Nagios-Plugins direkt vom Nagios-Server verwendet werden.
Da Windows als Betriebssystem auf dem \gls{OracleUCM}-Server eingesetzt wird, kann die \gls{SSH}-Variante nicht eingesetzt werden.
Die \gls{NRPE}-Methode wird für die Auswertung der Logdateien verwendet.
Die Überwachung per \gls{SNMP} wird im Karlsruhe Insitute of Technology nicht eingesetzt, da man eine \gls{DoS}-Attacke durch das Senden von vielen \gls{SNMP}-Traps zu dem Nagios-Server verhindern will.\label{snmpkom}
%der höheren Komplexität dieser Variante und des begrenzten zeitlichen Rahmens dieser Arbeit nicht benutzt
Die passive Vorgehensweise mit \gls{NSCA} wird für die Umsetzung nicht benötigt und deshalb nicht verwendet.

Für die Überwachung von Windows-Servern wurde eine weitere Methode entwickelt, die auf dem Prinzip von \gls{NRPE} basiert.
Diese Variante wird NSClient genannt und benötigt, wie \gls{NRPE}, die Installation eines Nagios-Agenten auf dem zu überwachenden Server.
Daher muss eine Übersicht über verschiedene Nagios-Agenten erstellt werden.

\subsection{Übersicht Nagios-Agenten}
In diesem Unterkapitel werden die populärsten Agenten für Unix und Windows Betriebssysteme aufgelistet und nach den Punkten Sicherheit, subjektiver Aufwand für die Konfiguration und Art der Abfragemethode (aktiv oder passiv) verglichen.

\subsubsection{Unix-Agenten}
Für die auf Unix basierenden Betriebssysteme werden fünf verschiedene Möglichkeiten angeboten, die in Abbildung \ref{nagios-kern} als verschiedene Überwachungsmöglichkeiten von Nagios aufgelistet wurden.

%Tabelle Unix-Agenten


\begin{table}[h!]
\centering
\begin{threeparttable}[ht]
\begin{tabular}{l p{1.5cm} l p{1.5cm} l p{1.5cm} l p{1.5cm} l p{1.5cm} l p{1.5cm} p{1.5cm} p{1.5cm} p{1.5cm} p{1.5cm}}
 & \begin{turn}{50}\textbf{SSH}\end{turn} & \begin{turn}{50}\textbf{NRPE}\end{turn} & \begin{turn}{50}\textbf{SNMP}\end{turn} & \begin{turn}{50}\textbf{SNMP Traps}\end{turn} & \begin{turn}{50}\textbf{NSCA}\end{turn}\\ 
\hline
\textbf{Methode} & & & & & \\
\textit{aktiv} & \checkmark & \checkmark & \checkmark & - & - \\
\textit{passiv} & - & - & - & \checkmark & \checkmark\\
\textbf{Sicherheit} &  &  &  &  &  \\
\textit{Passwort} & - & - & \checkmark (v3) & \checkmark (v3) & -\\
\textit{Accesslist}\tnote{*} & \checkmark &  \checkmark & \checkmark (v2) & \checkmark (v2) & \checkmark \\
\textit{Verschlüsselung} &  \checkmark & \checkmark & \checkmark (v3) & \checkmark (v3) &  \checkmark \\
\textbf{Aufwand}\tnote{**} & \begin{footnotesize}leicht\end{footnotesize} & \begin{footnotesize}normal\end{footnotesize} & \begin{footnotesize}hoch\end{footnotesize} & \begin{footnotesize}hoch\end{footnotesize} & \begin{footnotesize}normal\end{footnotesize} \\
\end{tabular}
%\footnotesize
%* Einschränkung der Abfrage der Überwachungsinformationen anhand der \gls{IP}-Adresse
%\\
%** Subjektive Einschätzung
\begin{tablenotes}\footnotesize
      \item[*] Einschränkung der Abfrage der Überwachungsinformationen anhand der \gls{IP}-Adresse
        \item[**] Subjektive Einschätzung
    \end{tablenotes}
\caption{Übersicht der verschiedenen Unix-Agenten}
\end{threeparttable}
\end{table}




Dabei werden drei Agenten genannt, die eine aktive Ausführung der Nagios-Plugins benutzten.
Alle drei Agenten unterscheiden sich jedoch in den Punkten Sicherheit und Aufwand.
Der auf \gls{SSH} basierende Agent besitzt einen relativ geringen Aufwand für die Installation, da für den Aufbau der Kommunikation zwischen Nagios-Server und Client nur der öffentliche Schlüssel des Servers auf dem Client eingetragen werden muss.
Dadurch kann der Nagios-Server sich ohne Passwortabfrage an dem zu überwachendem Host anmelden und die sich darauf befindlichen Nagios-Plugins ausführen.
Da auf den meisten Unix-Servern bereits ein \gls{SSH}-Server läuft und deshalb kein weiterer Port geöffnet oder eine weitere Software installiert werden muss, ist diese Methode den anderen meist vorzuziehen.

Bei der \gls{NRPE}-Methode wird eine weitere Softwarekomponente auf dem Client installiert, die einen separaten Port für die Kommunikation mit dem Nagios-Server öffnet.
Wie bei dem Aufruf per \gls{SSH} müssen sich hier die Nagios-Plugins bereits auf dem Rechner befinden.
Dabei gilt als Unterschied dieser zwei ähnlichen Methoden zu beachten, dass für die Ausführung der Checks per \gls{SSH} ein extra Benutzerkonto auf dem Client erstellt werden muss und somit beliebige Systembefehle ausgeführt werden können, während die Ausführung von Kommandos bei \gls{NRPE} nur auf vorkonfigurierte Befehle beschränkt ist, wie in Kapitel \ref{sshnrpe} aufgeführt.
\label{unixagents}
Da \gls{SNMP} plattformunabhängig funktioniert ist es möglich diese Variante bei Unix- sowie bei Windowsservern einzusetzen.
Die verwendete \gls{SNMP}-Version bestimmt welche Sicherheitsmerkmale zur Verfügung stehen.
Zwar gibt es bereits seit Version 1 die Möglichkeit den Zugriff per Passwort in drei Gruppen aufzuteilen: kein Zugriff, Leserecht und Lese- mit Schreibrecht\footnote{Quelle: \cite{Barth08} S. 237}, jedoch wird dieses Passwort im Klartext übertragen, so dass es leicht auslesbar ist.
Auch die \gls{SNMP}-Version 2 inklusive der erweiterten Version 2c verwendet die gleiche unsichere Authentifizierung.
Erst ab Version 3 wird das Passwort verschlüsselt übertragen.
Während Barth behauptet, dass man bei \gls{SNMP} generell kein Passwort verwenden soll\footnote{Quelle: \cite{Barth08} S. 238}, da es leicht per Netzwerkmitschnittprogramme, wie WireShark, ausgelesen werden kann, wird in \cite{Jose07} S. 121 klargestellt, dass die Version 3 eine verschlüsselte Authentifizierung durch den MD5- oder SHA-Algorithmus ermöglicht.

Die passive Variante über \gls{SNMP} bei der der Client die Ergebnisse der Checks an den Nagios-Server sendet, auch \gls{SNMP}-Traps genannt, funktioniert nach dem gleichen Prinzip.
Da das Auslesen der \gls{MIB} per \gls{SNMP} im Gegensatz zu den anderen Varianten deutlich komplexer ist, wird der Aufwand als hoch eingestuft.

Ein weiterer Vertreter, der passive Checks ermöglicht, ist der \gls{NSCA}-Agent.
Wie die anderen Unix-Agenten bietet es die Möglichkeit den Datenaustausch zwischen Nagios-Server und Client zu verschlüsseln.
Alle Unix-Agenten erlauben es den Zugriff auf die Nagios-Plugins auf bestimmte \gls{IP}-Adressen zu beschränken.
Die Liste mit diesen \gls{IP}-Adressen nennt man auch \textit{Accesslist}.

\subsubsection{Windows-Agenten}
Da die zu überwachende Oracle UCM Anwendung auf einem Windows-Server betrieben wird und die bereits vorgestellten Agenten mit Ausnahme der \gls{SNMP}-Varianten nur unter Unix einsetzbar sind, müssen zusätzlich die explizit für Windows entwickelten Nagios-Agenten untersucht werden.
Dabei wird die Auswahl der Kandidaten auf vier Bewerber beschränkt, siehe Tabelle \ref{tab:winagents}.
%Tabelle Windows-Agenten
%\vspace{1.6cm}
\begin{table}[!cht]
\centering
\begin{threeparttable}
\begin{tabular}{l p{1.3cm} l p{1.3cm} l p{1.3cm} l p{1.3cm} l p{1.3cm} l p{1.3cm} p{1.3cm} p{1.3cm} p{1.3cm} p{1.3cm}}
 & \begin{turn}{50}\textbf{NSClient}\end{turn} & \begin{turn}{50}\textbf{NRPE\_NT}\end{turn} & \begin{turn}{50}\textbf{NC\_net}\end{turn} & \begin{turn}{50}\textbf{NSClient++}\end{turn} & \begin{turn}{50}\textbf{OpMon Agent}\end{turn}\\ 
\hline
\textbf{Methode} & & & & & \\
\textit{aktiv} & \checkmark & \checkmark & \checkmark & \checkmark & \checkmark\\
\textit{passiv} & - & - & \checkmark & \checkmark & -\\
\textit{check\_nt}\tnote{1} & \checkmark & - & \checkmark & \checkmark & \checkmark\\
\textit{NRPE}\tnote{2} & - & \checkmark & \checkmark & \checkmark & \checkmark\\
\textbf{Sicherheit} &  &  &  &  &  & \\
\textit{Passwort} & \checkmark & \checkmark & - & \checkmark & \checkmark\\
\textit{Accesslist}\tnote{3} & - & - & \checkmark & \checkmark & \checkmark\\
\textit{Verschlüsselung} & - & \checkmark & \checkmark & \checkmark & -\\
\textbf{Aufwand}\tnote{4} & \begin{footnotesize}normal\end{footnotesize} & \begin{footnotesize}hoch\end{footnotesize} & \begin{footnotesize}normal\end{footnotesize} & \begin{footnotesize}normal\end{footnotesize} & \begin{footnotesize}normal\end{footnotesize}\\
\end{tabular}
\begin{tablenotes}\footnotesize
		\item[1] Kompatibelität mit dem Standard check\_nt Plugin
		\item[2] Erlaubt Ausführung von vorkonfigurierten Kommandos
        \item[3] Einschränkung der Abfrage der Überwachungsinformationen anhand der \gls{IP}-Adresse
        \item[4] Subjektive Einschätzung
    \end{tablenotes}
\caption{Übersicht der verschiedenen Windows-Agenten}
\label{tab:winagents}
\end{threeparttable}
\end{table}

\newpage
Der NSClient-Dienst liefert die Möglichkeit lokale Windows-Ressourcen über das Netzwerk mit eigenem Port (Standport 1248) abzufragen.
Das Plugin \textit{check\_nt} wurde explizit für diesen NSClient-Dienst entwickelt und steht durch die Nagios-Plugins standardmäßig zur Verfügung.
Dadurch können die grundlegende Informationen für die Systemüberwachung aus Kapitel \ref{syschecks}, wie Zustände von Prozesse, Services, CPU-Auslastung, Festplattenplatz, usw. abgefragt werden.
Der Zugriff auf den NSClient-Dienst per \textit{check\_nt} wird in Abbildung \ref{fig:cknt} gezeigt.
\begin{figure}[ht]
	\centering
	   \fbox{\includegraphics[width=0.8\textwidth]{bilder/monitoring-windows.png}}
		\caption[Abfrage von Windows-Ressourcen durch \textit{check\_nt}]{Abfrage von Windows-Ressourcen durch \textit{check\_nt}\protect\footnote}
		\label{fig:cknt}
\end{figure}
\footnotetext{Quelle: \url{http://nagios.sourceforge.net/docs/3\_0/images/monitoring-windows.png}}

Diese Abfrage kann durch die Ausführung auf der Kommandozeile getestet werden:

\begin{figure}[ht]
	\centering
	   \fbox{\includegraphics[width=0.75\textwidth]{bilder/check_nt-stdp.png}}
		\caption{Zugriff auf den NSClient-Dienst durch check\_nt}
		\label{fig:ckntsh}
\end{figure}

Der erste und zugleich älteste Agent NSClient wird nicht mehr aktiv entwickelt und ist als aktuellste Version 2.0.1 aus dem Jahre 2003 bereits recht alt.
Daher wird auch keine Verschlüsselung der ein- und ausgehenden Daten unterstützt.
Auch bietet NSClient keine Möglichkeit aktiv vom Nagios-Server aus Nagios-Plugins oder weite Programme auszuführen, die sich auf dem zu überwachendem Host befinden.

Um dies auch für Windows-Server zu ermöglichen gibt es eine auf Windows portierte \gls{NRPE}-Variante, die sich NRPE\_NT nennt.
Hier lassen sich die Plugins direkt über den Nagios-Server aufrufen und die ausgetauschten Informationen werden verschlüsselt über das Netzwerk übertragen.

Beide bisher genannte Windows-Agenten bieten keine Möglichkeit eine \textit{Accesslist} anzulegen, erst das Programm NC\_net bietet diese Möglichkeit inklusive dem Sicherheitsmerkmal Verschlüsselung an.
Außerdem können durch den eingebauten \gls{NRPE}-Dienst aktiv Nagios-Plugins auf dem Client aufgerufen werden.
Als Besonderheit lassen sich durch NC\_net sowohl aktiv als auch passiv Testergebnisse an den Nagios-Server übertragen. 

Der Nagios-Agent NSClient++ besitzt diesselben Merkmale wie NC\_net, jedoch kann der Nagios-Server noch über ein Passwort zusätzlich verifiziert werden.

Die Möglichkeit Informationen per \gls{SNMP} und \gls{SNMP}-Traps abzufragen ist auch unter Windows möglich.
Dabei gelten die gleichen Richtlinien, Hinweise und Einschränkungen wie zuvor in Kapitel \ref{unixagents} aufgeführt.

\subsubsection{Auswahl und Konfiguration des Nagios-Agenten}

\paragraph{Auswahl}
Anhand der im vorherigen Kapitel beschriebenen Übersicht der Windows-Agenten und der daraus resultierenden Übersichtstabelle \ref{tab:winagents} wird ein geeigneter Kandidat für die Testumgebung ausgewählt.
Da nur ein Windows-Agent alle drei Sicherheitsmerkmale anbietet und dabei aktive und passive Überwachungsmethoden erlaubt, fällt die Wahl auf das Programm \textbf{NSClient++}.

\paragraph{Installation und Konfiguration}

Der Nagios-Agent NSClient++ kann im Gegensatz zu den meisten anderen Windows-Agenten komfortabel über einen graphischen Benutzerdialog installiert werden.
Während des Installationsvorgangs kann auch festgelegt werden welche Komponenten installiert werden sollen.
Dabei werden diese Komponenten nicht standardmäßig geladen, sondern im nächsten Dialogfenster auswählbar:

\begin{figure}[ht]
	\centering
	   \fbox{\includegraphics[width=0.55\textwidth]{bilder/nsc2.png}}
		\caption{Konfiguration des NSClient++ während der Installation}
		\label{nscs2}
\end{figure}

Außerdem können direkt während der Installation die \gls{IP}-Adresse bzw. der \gls{FQDN} des Nagios-Servers und das gewünschte Passwort eingetragen werden.

Durch die während des Installationsprozesses geladenen Komponenten für den NSClient- und \gls{NRPE}-Dienst können die Standard-Nagios-Plugins \textit{check\_nt} und \textit{check\_nrpe} mit dem Windows-Server verwendet werden.
Dabei läuft die Kommunikation zwischen dem Nagios- und dem Windows-Server folgendermaßen ab:

\begin{figure}[ht]
	\centering
	   \fbox{\includegraphics[width=0.7\textwidth]{bilder/nagios-active-nsclient-and-nrpe.png}}
		\caption[Kommunikation zwischen Nagios und NSClient++]{Kommunikation zwischen Nagios und NSClient++\protect\footnote}
		\label{nscs2}
\end{figure}
\footnotetext{\url{http://nsclient.org/nscp/}}
%http://nsclient.org/nscp/raw-attachment/wiki/doc/usage/nagios/nagios-active-nsclient.png

Bei Windows-Server mit vielen Verbindungen und Diensten können Remote Procedure Calls (\gls{RPC}), bei denen dynamisch Ports ab 1025 verwendet werden, bereits den Standardport des NSClient-Dienstes (1248) unter Umständen bereits vor dem Start des Dienstes belegen.\footnote{Quelle: \cite{Barth08} S. 481}
Um dies zu verhindern wurde der Port des NSClient-Dienstes beim NSClient++ bereits vom Entwickler auf einen höheren Port (12489) gewechselt.

Alle bisherigen Einstellungen können in der Konfigurationsdatei \textit{NSC.ini}, die sich in dem Installationsverzeichnis des NSClient++ befindet, verändert werden.
In dieser Datei befinden sich noch mehr Einstellungsmöglichkeiten; im Folgenden werden nur für die Umsetzung relevanten (notwendig essentiell benötigten) Parameter aufgelistet.

\begin{lstlisting}[captionpos=b, caption=NSClient++ Konfigurationsdatei, label=code:nsc, breaklines = true, language=sh]
;# NSCLIENT PORTNUMMER
;#  Die Portnummer des NSClient-Dienstes
port=13596

;# NRPE PORTNUMMER
;#  Die Portnummer des NRPE-Dienstes
port=13597

;# SSL SOCKET
;#  Die Aktivierung von SSL der Kommunikation zwischen Nagios- und Windows-Server 
use_ssl=1

;# NRPE BEFEHLSDEFINITIONEN
;# Definitionen der Befehle, die durch den NRPE-Dienst aufrufbar sind
check_uname=scripts\check_uname.exe
check_reflog=scripts\check_logfiles.exe -f scripts\logfile.cfg
\end{lstlisting}

Damit die vorgenommen Änderung übernommen werden, muss der Dienst des NSClient++ neu gestartet werden. 

%\begin{figure}[ht]
%	\centering
%	   \fbox{\includegraphics[width=0.85\textwidth]{bilder/nsc3.png}}
%		\caption{NSClient++ Windowsdiensteintrag}
%		\label{nscs3}
%\end{figure}

Durch das Ausweichen auf höher gelegene Portnummern können die zuvor genannten Probleme aufgrund der \gls{RPC}s verhindert werden.

Der bereits höher liegende Standardport des NSClient-Dienstes beim NSClient++ wird zusätzlich noch abgeändert, damit die Tatsache, dass sich ein Nagios-Agent auf dem Computer befindet, nicht sofort ersichtlich ist.
Dieser sicherheitstechnische Aspekt wurde bereits in Kapitel \ref{changeport} behandelt.

Damit die \gls{SSL}-Verschlüsselung zwischen den Servern aktiviert wird, muss man es explizit in der Konfigurationsdatei mit der Option \textit{use\_ssl=1} angeben.

Die Definitionen der \gls{NRPE}-Kommandos dienen dafür, dass durch den Nagios-Server per \textit{check\_nrpe} mit dem Befehlsnamen der darauf folgende Befehl ausgeführt wird.

Aufgrund der abgeänderten Portnummer muss man den Port bei dem Aufruf explizit angeben.
Ein Aufruf eines solchen \gls{NRPE}-Kommandos vom Nagios-Server wird in der folgenden Abbildung gezeigt:

\begin{figure}[ht]
	\centering
	   \fbox{\includegraphics[width=0.85\textwidth]{bilder/nrpe-check.png}}
		\caption{Aufruf eines NRPE-Kommandos}
		\label{nrpecheck}
\end{figure}



Anhand des Befehlsnamens \textit{check\_uname} führt der \gls{NRPE}-Dienst die in der Konfigurationsdatei eingetragene Datei \textit{check\_uname.exe} aus.

Der Aufruf um Informationen durch den NSClient-Dienst abzufragen sieht ähnlich aus:

\begin{figure}[ht]
	\centering
	   \fbox{\includegraphics[width=0.85\textwidth]{bilder/nsclient-check.png}}
		\caption{Aufruf des NSClient-Dienstes}
		\label{ntcheck}
\end{figure}

Die Servicedefinition des vorherigen NSClient-Aufrufs muss wie nachfolgend/folgt in der Nagios-Konfiguration eingetragen:
\begin{lstlisting}[captionpos=b, caption=Servicedefinition des NSClient-Checks, label=nt-servdef, breaklines = true, language=sh]
define service{
        use                     generic-service
        host_name               example.kit.edu
        service_description     Uptime
        check_command           check_nt!-p 13596 -s secret -v UPTIME
        }
\end{lstlisting}

Damit nicht jeder einzelne Serviceeintrag abgeändert werden muss, falls sich der Port oder das Passwort des zu überwachenden Computers ändert, können eigene Befehlsdefinitionen erstellt werden.

\begin{lstlisting}[captionpos=b, caption=Server spezifische Befehlsdefinition, label=cus-nt-servdef, breaklines = true, language=sh]        
define command{
        command_name    check_nt_example
        command_line    /usr/lib/nagios/plugins/check_nt -H $HOSTNAME$ -p 13597 -p secret -v $ARG1$
        }
\end{lstlisting}

Dadurch muss nur diese Befehlsdefinition bei einer Änderung bearbeitet werden.
Die vorherige Servicedefinition in Listing \ref{ntcheck} kann dann in verkürzter Form eingetragen werden:

\begin{lstlisting}[captionpos=b, caption=Verkürzte Servicedefinition des NSClient-Checks, label=nt-servdef, breaklines = true, language=sh]
define service{
        use                     generic-service
        host_name               example.kit.edu
        service_description     Uptime
        check_command           check_nt_example!UPTIME
        }
\end{lstlisting}




\subsection{Umsetzung der Systemüberwachung}

Die in Kapitel \ref{syschecks} aufgelisten Prozesse und Services können durch den NSClient-Dienst vom Nagios-Server überwacht werden.
Dafür wird der in Listing \ref{cus-nt-servdef} definierte verkürzte Befehl für \textit{check\_nt} benutzt.

\begin{lstlisting}[captionpos=b, caption=Prozess- und Service-Check Servicedefintionen, label=procservdef, breaklines = true, language=sh]
#Prozess des IIS Webservers
define service{
        use                     generic-service
        host_name               example.kit.edu
        service_description     IIS Prozess
        check_command           check_nt_example!PROCSTATE -l w3wp.exe
        }

#Zeitdienst
define service{
        use                     generic-service
        host_name               example.kit.edu
        service_description     Zeitdienst
        check_command           check_nt_example!SERVICESTATE -l W32TIME
        }
\end{lstlisting}

Mit diesen zwei Einträgen wird der Prozess des \gls{IIS}-Webservers und der Status des Dienstes zum Zeitabgleich überwacht.
Andere Prozesse und Dienste lassen sich nach dem gleichen Schema überwachen.

Nach einem Neustart von Nagios werden beide Einträge im Webinterface angezeigt:

\begin{figure}[ht]
	\centering
	   \fbox{\includegraphics[width=0.95\textwidth]{bilder/servproc.png}}
		\caption{Prozess- und Dienstüberwachung im Nagios-Webinterface}
		\label{servprocgui}
\end{figure}

Die Festplattenspeicherausnutzung und die Prozessorauslastung wird auf ähnliche Weise überwacht.
Hierbei muss beachtet werden, dass die Testergebnisse nicht eindeutig sind, im Gegensatz zu der Service- und Prozessüberwachung.
Wann Nagios alarmieren soll muss vom Anwender in Form von Parametern festgelegt werden.

\begin{lstlisting}[captionpos=b, caption=Überwachung der Festplatten- und Prozessorauslastung, label=cpuhdddef, breaklines = true, language=sh]
#Belegung der Partition C:
define service{
        use                     generic-service
        host_name               example.kit.edu
        service_description     C:\ Drive Space
        check_command           check_nt_example!USEDDISKSPACE -l c -w 85 -c 100
        }
        
#CPU Auslastung der letzten 5 Minuten
define service{
        use                     generic-service
        host_name               example.kit.edu
        service_description     CPU Load
        check_command           check_nt_example!CPULOAD -l 5,80,100
        }
\end{lstlisting}

Für diese Festplattenüberwachung versendet Nagios eine Warnung, wenn der belegte Speicherplatz auf der C-Partition die 85\% Marke überschreitet und meldet einen kritischen Fehler bei 100\%.
Bei der Prozessorüberwachung schlägt Nagios Alarm, wenn der Mittelwert der Auslastung in den letzten fünf Minuten mehr als 80\% bzw. 100\%  betragen hat.


\subsection{Umsetzung der Funktionlitätstest}

Für die Ausführung der einfachen Funktionlitätstest aus Kapitel \ref{funztest} werden Benutzerinformationen zur Anmeldung benötigt.
Nagios besitzt extra hierfür die Möglichkeit diese Benutzerinformationen in Variablen zu speichern, damit sie nicht einzeln bei jeder Servicedefinition verändert werden müssen.
Da sich die Definition dieser Variablen in einer externen Datei befindet, können die Zugriffsrechte auf diese Datei eingeschränkt werden, wodurch die Anmeldedaten bei den Servicedefinitionen nicht auslesbar sind.

\begin{lstlisting}[captionpos=b, caption=Funktionalitätstest der Benutzeranmeldung, label=userauthdef, breaklines = true, language=sh]
#Anmeldung an Oracle UCM mit lokalem Benutzerkonto
define service{
        use                     generic-service
        host_name               example.kit.edu
        service_description     Anmeldung Oracle UCM als lokaler Benutzer
        check_command           check_http!-u "/bdb/idcplg?IdcService=LOGIN&Action=GetTemplatePage&Page=HOME\_PAGE&Auth=Internet"  -a $USER3$:$USER4$ -e "Sie sind angemeldet als" -S
        }
\end{lstlisting}


Dabei werden dem Nagios-Plugin \textit{check\_http} mit dem \pictext{u}-Parameter die URL zur Benuteranmeldungseite und mit dem Parameter \pictext{a} der benutzername und -passwort mitgegeben.
Der nach dem Parameter \pictext{e} folgende String wird dann in der Antwort des Servers gesucht.
Sollte dieser String nicht gefunden werden ist die Authentifizierung fehlgeschlagen und es wird durch Nagios eine Meldung versendet.
Mit dem Parameter \pictext{S} wird angegeben, dass eine \gls{SSL}-verschlüsselte Verbindung zum Webserver über HTTPS hergestellt werden soll, ansonsten würden die Benutzerinformationen im Klartext übertragen werden, wodurch sie leicht für Angreifer auslesbar wären.

Für das Auslesen von Informationen aus der Statusseite der Oracle UCM-Anwendung wurde ein einfaches BASH-Script entwickelt:

\begin{lstlisting}[captionpos=b, caption=Auslesen der Verbindungen zur Datenbank, label=dbcon, breaklines = true, language=sh]
#!/bin/bash
E_BADARGS=2
if [ ! -n "$6" ]
then
        echo "Usage: `basename $0` -url <URL> -u <username> -p <password>"
        exit $E_BADARGS
fi

DBCONNECTIONS=$(wget -qO-  --user $4 --password $6 $2 | grep "System Database")
DBCONNECTIONS=${DBCONNECTIONS##*>}
DBCONNECTIONS=${DBCONNECTIONS%% *}
echo $DBCONNECTIONS
\end{lstlisting}

Dabei muss als URL die Seite mit den Datenbankverbindungen und gültige Benutzerinformationen mitgegeben werden.
Anschliessend wird die aufgerufene Seite nach der gewünschte Informationen untersucht und ausgegeben.

Dieses einfaches Script kann auch dazu verwendet um andere Informationen von der Statusseite abzufragen.

\subsection{Auswertung der Logdateien}

Um die drei genannten Logdateien aus Kapitel \ref{checklog} auszuwerten wird durch den \gls{NRPE}-Dienst das Plugin \textit{check\_logfiles} von Gerhard Laußer\footnote{Quelle: \url{http://www.consol.de/opensource/nagios/check-logfiles}} eingesetzt.

Dieses Plugin besitzt bereits einige nützliche Funktionen für die Überwachung von Logdateien.
Durch das Setzten eines Zeitstempels filtert das Plugin veraltete Einträge heraus und untersucht nur neu hinzugekommene Zeilen.
Der Rotationsalgorithmus der \gls{OracleUCM}-Logdateien kann dem Plugin durch die Verwendung einer Konfigurationsdatei mitgeteilt werden.

Die Konfigurationsdatei für das \textit{check\_logfiles}-Plugin wird im folgendem Listing gezeigt:


\begin{lstlisting}[captionpos=b, caption=Konfigurationsdatei für \textit{check\_logfiles}, label=chklogcfg, breaklines = true, language=sh]
@searches = ({
  tag => 'ucmlogs',
  type => 'rotating::uniform',
  logfile => 'D:/bdb2/weblayout/groups/secure/logs/bdb/IdcLnLog.htm',
  rotation => 'refinery\d{2}\.htm',
  warningpatterns => [
        'Cannot identify file', #Testbild korrupt
	  'Bad CRC value in IHDR chunk', #Konvertierung nicht erfolgreich
	  'Der Dateiname darf nicht länger als 80 Zeichen sein', #Zu langer Dateiname
    ],
});
\end{lstlisting}

Mit dem \pictext{tag}-Attribut wird diese Auswertung eindeutig identifizierbar gemacht, da man in der gleichen Konfigurationsdatei mehrere Logdateien bzw. weitere Durchsuchungen definieren kann.
Das Attribut \pictext{type} muss hier so gesetzt werden, da die aktuelle Logdatei und die wegrotierte Logdatei das gleichen Namensschema benutzten.
Der Pfad zu den Logdateien wird über das Attribut \pictext{logfile} gesetzt.
Da die einzelnen Logdateien mit einem bestimmten Namensmuster erstellt werden (siehe Kapitel \ref{checklog}), muss dieses Namensmuster hier direkt angegeben werden.
In diesem Falle durchsucht das \textit{check\_logfiles}-Plugin alle Logdateien mit dem Namen \textit{refinery00.htm} bis \textit{refinery99.htm}.
Alle gefundenen Dateien werden dem Datum nach sortiert und die aktuellste Datei wird untersucht.
Nach welchen Stopwörtern gesucht werden soll wird mit dem Attribut \pictext{warningpatterns} angegeben, sofern mindestens einer dieser Strings gefunden wurde liefert das Plugin ein WARNING als Rückmeldung inklusive der Zeile in dem das Stopwort gefunden wurde.

\begin{figure}[ht]
	\centering
	   \fbox{\includegraphics[width=0.95\textwidth]{bilder/logcheckwarn.png}}
		\caption{Ausgabe der betreffenden Zeile in der Logdatei}
		\label{checklogwarn}
\end{figure}


\subsection{Benutzersimulation}

Um die Funktionalität der Anwendung eindeutig festzustellen werden typische Benutzeraktionen simuliert und die Ergebnisse an Nagios übermittelt.

Solche typische Aktionen sind das Einchecken eines Bildes, Suche nach einem Bild und schließlich die Anforderung des Originalbildes und der konvertierten Bilder.
%Benutzertätigkeiten, Handlungen
Da Nagios standardmäßig ein Plugin periodisch jede fünf Minuten aufruft, würde sich die Festplatte und die Datenbank des \gls{OracleUCM}-Servers im Laufe der Zeit an ihre Kapazitäten stoßen.
Daher werden nach der Anforderung und Überprüfung der Bilder alle Testbilder vom Server entfernt.

Der Ablauf der Benutzersimulation soll in verkürzter Form durch folgendes Struktogramm verdeutlicht werden:

\begin{figure}[ht]
	\centering
	   \includegraphics[width=0.9\textwidth]{bilder/Benutzersimulation.png}
		\caption{Geplanter Ablauf der Benutzersimulation}
		\label{user-sim}
\end{figure}



Per Webservice soll ein Testbild an den Server geschickt und eingecheckt werden.
In diesem ersten Schritt wird auch gleichzeitig die Erreichbarkeit der Anwendung über das Netzwerk getestet.

Wenn die Übertragung des Bildes erfolgreich wird anschließend nach dem soeben eingecheckten Bild per Dateinamen gesucht um die Funktionalität der Indizierung zu kontrollieren.

Sollte das Bild gefunden werden, wird es vom Nagios-Server angefordert und auf seine Korrektheit überprüft.
Der gleiche Test wird mit den konvertieren Bildversionen durchgeführt, um die Funktion der Konvertierung zu überwachen.

Falls alle Tests erfolgreich waren, wird das Testbild und alle konvertierten Bilder vom \gls{OracleUCM}-Server gelöscht.
Bei den anderen Szenarien gibt das Plugin den Wert 2 für den Status CRITICAL zurück.\\

Die Realisierung dieser Simulation wird durch zwei Plugins realisiert.

\paragraph{Einchecken eines Testbildes}
Das erste Plugin dient zum Einchecken des Testbildes.
Dabei ruft der Nagios-Server ein auf \gls{PHP}-basierendes Script auf.
In diesem Script wird die \gls{PHP}-Klasse \textit{nuSOAP} eingebunden, damit man vereinfacht auf Web Services zugreifen kann.
Die Kommunikation zwischen Client und Server bei der Benutzung eines Web Services findet, wie im Kapitel \ref{webservice} beschrieben, im \gls{XML}-Format statt.
Um den Aufwand zu vermeiden diese \gls{XML}-Datei immer selbst zu erstellen, wird mit Hilfe der \gls{WSDL}-Datei auf dem \gls{OracleUCM}-Server die benötigten Parameter beim Aufruf eines Web Services von \textit{nuSOAP} ausgelesen.

In der folgenden Abbildung werden aus der \gls{WSDL}-Datei alle möglichen Anforderungsparameter für den Web Service \textit{CheckInUniversal} gezeigt.

\begin{figure}[ht]
	\centering
	   \fbox{\includegraphics[width=0.9\textwidth]{bilder/wsdlscrn.png}}
		\caption{Anforderungsparameter für CheckInUniversal aus der WSDL-Datei}
		\label{wsdl1}
\end{figure}


%\begin{lstlisting}[captionpos=b, caption=Anforderungsparameter aus der WSDL-Datei, label=1stwsdl, breaklines = true, language=xml]
%<s:element name="CheckInUniversal">
% <s:complexType>
%  <s:sequence>
%   <s:element minOccurs="0" maxOccurs="1" name="dDocTitle" type="s:string"/>
%   <s:element minOccurs="0" maxOccurs="1" name="dDocType" type="s:string"/>
%   <s:element minOccurs="0" maxOccurs="1" name="dDocAuthor" type="s:string"/>
%   <s:element minOccurs="0" maxOccurs="1" name="dSecurityGroup" type="s:string"/>
%   <s:element minOccurs="0" maxOccurs="1" name="dDocAccount" type="s:string"/>
%   <s:element minOccurs="0" maxOccurs="1" name="primaryFile" type="s0:IdcFile"/>
%  </s:sequence>
% </s:complexType>
%</s:element>
%\end{lstlisting}

Im PHP-Script werden nach dem Einlesen dieser \gls{WSDL}-Datei und der Authentifizierung am \gls{OracleUCM}-Server die benötigten Parameter beim Aufruf des Web Services gesetzt und die Ausgabe des Servers ausgewertet.

\begin{lstlisting}[captionpos=b, caption=Aufruf des Web Services CheckInUniversal, label=1stplugin, breaklines = true, language=PHP]
$soap = new soapclient($WSDL-URL,  //WSDL-Datei einlesen 
array('login' => $user, 'password' => $password)); //Authentifizierung am Oracle UCM-Server

//Aufruf des Web Services
$ergebnis = $soap->CheckInUniversal(array(
	'dDocAuthor'=>$user, //Autor des Bildes
	'dDocTitle'=>'testBild4nagios', //Titel des Bildes
	'dSecurityGroup'=>'private', //Sichtbarkeit des Bildes
	'dDocAccount'=>'NAGIOS/TEST', //Angabe einer Gruppe
	'dInDate'=>date("d.m.y H:i"), //Aktuelles Datum
	'dDocType'=>'Picture', //Dokumententyp
	'doFileCopy'=>'1', //Datei nur kopieren, nicht verschieben
	'dDocFormat'=>'image/png', //MIME-Type
	'primaryFile'=>array(
		'fileName'=>'testBild4nagios',
 		'fileContent'=>$content) //Byteweise eingelesenes Bild
));
[...]
//Auswertung der Antwort des Servers
if (ereg(' erfolgreich eingecheckt.', $output)) {
  echo('CHECKIN OK - '.$output);
  die(0); //Einchecken erfolgreich
} else {
 echo('CHECKIN CRITICAL - '.$output);
 die(2); //Einchecken fehlgeschlagen
}
\end{lstlisting}

Die Bilddatei muss für die Übertragung über \gls{HTTP} zuerst byteweise eingelesen werden und anschließend mit dem base64-Algorithmus kodiert werden.
Dabei übernimmt die nuSOAP-Klasse die base64-Enkodierung.

Der Ablauf dieses Plugins soll durch folgende Abbildung verdeutlicht werden: 
\begin{figure}[ht]
	\centering
	   \fbox{\includegraphics[width=0.93\textwidth]{bilder/wsdl.png}}
		\caption{Einchecken eines Testbildes}
		\label{usersim}
\end{figure}

\begin{enumerate}
\item Im ersten Schritt wird das \gls{PHP}-Script vom Nagios-Server aufgerufen und liest das Testbild ein. Anschließend verwendet es die \gls{WSDL}-Datei des Web Services um das entsprechende \gls{XML}-Dokument zu erstellen. Diese \gls{XML}-Datei wird anhand der nuSOAP-Klasse \gls{SOAP}-konform an den \gls{OracleUCM}-Servers gesendet.
\item Die \gls{XML}-basierende Rückantwort des Servers wird auch anhand der \gls{WSDL}-Datei erstellt und kann vom \gls{PHP}-Script ausgewertet werden.
\end{enumerate}


\paragraph{Validierung der Indizierung und Konvertierung}
Der zweite Teil der Benutzersimulation überprüft, ob das Testbild erfolgreich indiziert und konvertiert wurde.
Dabei verwendet es die gleichen Grundfunktionen wie das erste Plugin.
Jedoch wird anstatt dem Web Service \textit{CheckInUniversal} die \gls{WSDL}-Datei des Web Services \textit{AdvancedSearch} verwendet und aufgerufen.

\begin{lstlisting}[captionpos=b, caption=Überprüfen der Indizierung anhand einer Suchanfrage, label=2stplugin, breaklines = true, language=PHP]
$ergebnis = $soap->AdvancedSearch(
	'queryText'=>"dDocTitle <substring> `testBild4Nagios"
);
\end{lstlisting}

Durch diesen Aufruf wird anhand seines Titels nach einem zuvor eingechecktem Testbild gesucht.
Dadurch kann überprüft werden, ob das Bild korrekt vom Server angenommen und indiziert wurde.
In der Rückantwort des Servers befindet sich unter anderem die eindeutige Identifikationsnummer des Testbildes.
Diese Nummer wird für die Validierung der Konvertierung verwendet.
\begin{lstlisting}[captionpos=b, caption=Überprüfen der Indizierung anhand einer Suchanfrage, label=2stplugin, breaklines = true, language=PHP]
//Test des Originalbildes
$ergebnisGet = $soap->GetFileByID('dID'=>$dID);
[...]
if(!mb_eregi('PNG', $outputGetOrig))
{
        echo('SEARCH CRITICAL - Originalbild ist nicht im PNG Format!');
        die(2); //Originalbild korrupt
}
//Test der Thumbnailversion des Testbildes
$ergebnisGetThumbnail = $soap->GetFileByID(array('dID'=>$dID, 'rendition' => 'Thumbnail'));
[...]
if(!mb_eregi('JFIF', $outputGetThumbnail))
{
        echo('SEARCH CRITICAL - Thumbnailversion des Testbildes ist nicht im JPEG Format!');
        die(2); //Thumbnailversion korrupt
}
[...] //Überprüfung der anderen konvertierten Bilder
\end{lstlisting}

Sofern das Bild gefunden wurde, wird es vom Plugin anschließend angefordert und nach dem Dateityp untersucht.
Die Konvertierung wird dadurch überprüft indem die konvertierten Versionen des Testbildes angefordert werden und wie das Originalbild auf einen gültigen Dateityp getestet werden.

Der Ablauf dieses Plugins ist dem ersten sehr ähnlich, siehe Abbildung \ref{usersim2}

\begin{figure}[ht]
	\centering
	   \fbox{\includegraphics[width=0.93\textwidth]{bilder/wsdl-valid.png}}
		\caption{Validierung der Indizierung und Konvertierung}
		\label{usersim2}
\end{figure}

\begin{enumerate}
\item Zuerst wird die \gls{WSDL}-Datei für den Web Service \textit{AdvancedSearch} eingelesen. Eine Suchanfrage nach dem Testbild wird wieder über eine \gls{XML}-Datei an den Server gesendet. Wenn eine Datei gefunden wurde, wird das Originalbild und die konvertierten Versionen anhand der Identifikationsnummer angefordert.
\item Diese Bilder werden wieder byteweise innerhalb der \gls{XML}-Datei der Serverantwort an den Nagios-Server übertragen und auf ihre Korrektheit überprüft. Sollten alle bisherigen Tests ohne Probleme abgelaufen sein, wird das Testbild und die konvertierten Bilder vom Server entfernt.
\end{enumerate}

Standardmäßig bietet \gls{OracleUCM} keinen Web Service zum Löschen von Dokumenten an.
Daher muss zunächst eine \gls{WSDL}-Datei dafür erstellt werden, siehe Abbildung \ref{cwsdl}.

\begin{figure}[ht]
	\centering
	   \fbox{\includegraphics[width=0.93\textwidth]{bilder/cwsdl2.png}}
		\caption{Anlegen eines eigenen Web Services}
		\label{cwsdl}
\end{figure}


Der Name der \gls{WSDL}-Datei und des Web Services kann beliebig gewählt werden, solange als Service die entsprechende interne Bezeichnung zum Löschen von Dokumenten \pictext{DELETE\_REV} verwendet wird.\footnote{Quelle: \cite{Huff06} S. 379}
Um Zweideutigkeiten zu vermeiden, wird als Anforderungsparameter die eindeutige Identifikationsnummer verwendet.

Der eigene Web Service wird wie die anderen im Anschluss aufgerufen:

\begin{lstlisting}[captionpos=b, caption=Aufruf des eigenen Web Services, label=2stplugin2, breaklines = true, language=PHP]
$ergebnisDelete = $soap->DeleteRevisionByID('dID'=>$dID);
\end{lstlisting}
Die erstellten PHP-Dateien müssen noch Nagios als Befehl hinzugefügt werden.
\begin{lstlisting}[captionpos=b, caption=Befehldefinitionen der Benutzersimulation, label=phpdef, breaklines = true, language=sh]
#Einchecken eines Testbildes
define command{
        command_name    check_ucm_checkin
        command_line    /usr/lib/nagios/plugins/check_ucm/nagiosCheckin.php
        }
#Validierung der Indizierung und Konvertierung
define command{
        command_name    check_ucm_search
        command_line    /usr/lib/nagios/plugins/check_ucm/nagiosSearch.php
        }
\end{lstlisting}

Die Aufteilung der Benutzersimulation in zwei Nagios-Kommandos sorgt dafür, dass es keine Garantie für die Reihenfolge der Ausführung der Plugins gibt.
Aufgrund diese Asynchronität und dem Umstand, dass die Indizierung und Konvertierung des Testbildes abhängig von der Auslastung des \gls{OracleUCM}-Servers ist, wird die Anzahl der Ausführungen des Plugins für den Wechsel von Soft- zum Hardstate erhöht.

\begin{lstlisting}[captionpos=b, caption=Angepasste Servicedefinition für die Benutzersimulation, label=phpdef, breaklines = true, language=sh]
define service{
        use                     generic-service
        host_name               example.kit.edu
        service_description     Oracle UCM Search Delete
        max_check_attempts      8
        check_command           check_ucm_search
        }
\end{lstlisting}

%\begin{itemize}
%\item \url{http://www.w3schools.com/soap/default.asp} Web Services und SOAP
%\item \url{http://www.w3schools.com/wsdl/wsdl_summary.asp} WSDL
%\item max attempts bei Search erhöhen, da Auslastung der InboundRef -> möglichst keine/geringe False Positives -> auf Ausblick verweisen, Rahmenbedingungen müssen im Feld in der Praxis erst noch gefunden werden
%\end{itemize}



 \newpage
%\input{/16/docs/_praxis3/tex/ziele.tex} \newpage
%\input{/16/docs/_praxis3/tex/programmaufbau.tex} \newpage
%\input{tex/anpassungen.tex} \newpage
%\section{Zusammenfassung}

Im Rahmen dieser Arbeit wurde eine Lösung entwickelt um mit der Open Source-Überwachungssoftware Nagios den Betrieb des im Forschungszentrum Karlsruhe verwendeten Dokumenten-Management-Systems \gls{OracleUCM} zu überwachen.
Eine solche Überwachung ist notwendig um den Mitarbeitern des Forschungszentrums Karlsruhe einen möglichst zuverlässigen Dienst anbieten zu können.
Dabei sollte die Überwachung proaktiv auf mögliche Fehlzustände testen und bei einer Störung eine Alarmmeldung an die verantwortlichen Kontaktpersonen versenden. 
Das Überwachungssystem sorgt dafür, dass jeder Fehler sofort gemeldet wird, damit die Problemquellen vom Administrator gefunden und eventuell behoben werden können, bevor die Endbenutzer Störungen bei der Nutzung des Dienstes bemerken.\\

Für die Bearbeitung der Aufgabe war es notwendig sich mit den Grundlagen von Überwachungssystemen auseinander zusetzten.
Darunter fielen die Punkte Netzwerkstruktur , -abhängigkeit und verschiedene Sicherheitsaspekte die beim Einsatz einer Überwachungssoftware eine Rolle spielen.
Um die eigentliche Funktions- und Arbeitsweise eines Dokumenten-Management-Systems zu verstehen wurde die grundsätzliche Art eines Dokumentes im Vergleich zu Daten betrachtet.
%Dabei definiert die Strukturiertheit der enthaltenen Informationen die Zuordnung zu Daten oder Dokumente, obwohl die Grenzen nicht eindeutig festgelegt sind.
Auf diesem Wissen aufbauend konnten die Aufgabenbereiche Eingabe, Verwaltung, Archivierung und Ausgabe eines Dokumenten-Management-Systems untersucht und Vergleiche zu Content-Management-Systemen gezogen werden.
%Das kleinste Objekt in einem \gls{DMS} kann nur ein einzelnen Dokument sein, während ein \gls{CMS} einzelne Informationen aus verschiedenen Dokumenten erfassen kann, um ein neues Dokument zu generieren.

Für die Umsetzung wurde die Service-orientierte Architektur der \gls{OracleUCM}-Anwendung in Verbindung mit Web-Services verwendet.
Hierfür war es notwendig sich mit Grundprinzipien dieser Architekturen, deren Funktionsweise und verwendete Elemente vertraut zu machen.
Dadurch konnte später korrekt auf die benötigten Funktionen zugegriffen werden.\\

Im Forschungszentrum Karlsruhe wird als Überwachungssoftware das Open Source-Programm Nagios für die Überwachung von Netzwerken, Server und Dienste verwendet.
Damit Fehler korrekt von Nagios erkannt werden, bestand die Notwendigkeit die Funktionsweise und den Aufbau dieser Software zu studieren.
%Um die gewünschte Überwachung durch Nagios zu erreichen / ermöglichen
%Die Realisierung der Aufgabenstellung wurde durch die Anpassung der verschiedenen Konfigurationsdateien erreicht, da über diese Dateien die zu überwachenden Dienste und Server mit Nagios verbunden werden.
Das Einholen von Informationen zur Auswertung wird durch Plugins ermöglicht.
Das Verständnis über die Struktur und Richtlinien dieser Plugins wurde benötigt um später eigene zu entwickeln und sie effektiv zu verwenden.
Dabei galt es die speziellen Funktionen von Nagios wie die Hard und Soft States oder das Flapping von Zustände bei der spätere Verwendung zu berücksichtigen.
Über die verschiedene Möglichkeiten die benötigten Informationen zu sammeln wurde ein kurzer Überblick gegeben.

\gls{OracleUCM} wird im Forschungszentrum Karlsruhe für die Verwaltung von Webseiten, Dokumenten und Bilder eingesetzt.
Durch die Untersuchung des allgemeinen internen Aufbaus und der Arbeitsweise dieser Anwendung konnten die notwendigen Überwachungselemente ermittelt werden.
%Für die Ermittlung der Überwachungselemente wurde der  in allen Einsatzfällen untersucht.
Für diese Arbeit wurde der konkrete Einsatz von \gls{OracleUCM} als Bilddatenbank verwendet.
Die dabei auftretenden typischen Benutzerinteraktionen wurden für die später folgende Benutzersimulation verwendet.

Die einzelnen Überwachungselemente wurden in die Ebenen Statusabfragen, Funktionalitätstest, Auswerten von Logdateien und Benutzersimulation unterteilt.
Dabei führte die Abhängigkeit der Elemente zueinander zu der Einordnung in die verschiedenen Ebenen.
Unter den Statusabfragen befinden sich einfache Test wie ein Ping, Arbeitsspeicherauslastung oder der Zustand eines Prozesses.
Bei den Funktionalitätstest werden Anwendungen verwendet und die Antwort ausgewertet wie beispielsweise eine Anmeldung an Webserver mit Benutzerdaten.
Die Benutzersimulation beinhaltet verschiedene Benutzeraktionen und überprüft, ob die Anwendung noch alle Funktionen erfüllt.
Diese Einteilung in die verschiedenen Überwachungsebenen gibt den Verantwortlichen einen besseren Überblick über die Fehlersituation, so dass Fehlerquellen schneller entdeckt werden können.

Für die Umsetzung wurde ein Testsystem aufgesetzt, das aus einer separaten Nagios-Installation zum Testen der Überwachung und einer virtuellen Maschine, als Klon der Bilddatenbank zum Simulieren der einzelnen Fehlzustände, bestand.
%Für den konkreten Einsatz von Nagios als Überwachungssystem musste sich mit den verschiedenen Nagios-Agenten auseinandergesetzt werden, da diese die benötigten Informationen von entfernten Servern generieren.
%Die zuvor beschriebenen Überwachungsmethoden von Nagios wurden dabei in Verbindung mit den Agenten, die Unix- und Windows-Agenten aufgeteilt wurden, nochmals aufgegriffen.
Da es sich bei der zu überwachenden Bilddatenbank um einen Windows-Server handelte, wurde ein passender Nagios-Agent ausgewählt und dessen Installation und Konfiguration erläutert.
Durch den Einsatz dieses Agenten konnten die verschiedenen Ebenen der Überwachung durch Verwendung von verschiedenen Überwachungsmethoden realisiert werden.
Dabei wurde für die Benutzersimulation eigene Plugins entwickelt, die die Benutzeraktionen per Web Service ausführen.
%Zu den typischen Aktionen der Anwender zählt das Hochladen eines Bildes (Einchecken).
Das Plugin testet mit dem Hinzufügen eines Testbildes die Erreichbarkeit und einen Teil der Funktionalität des \gls{OracleUCM}-Servers.
Durch eine Suchanfrage wird die Indizierung überprüft und die Konvertierung wird anhand der angeforderten Testbilder validiert.
Dazu werden von den Plugins Web Services der \gls{OracleUCM}-Anwendung aufgerufen.
Die Konsequenzen einer automatischen Benutzersimulation mussten beachtet werden.
Durch die ständige Ausführung der Benutzersimulation würden die Ressourcen des Servers wie der Festplattenspeicherplatz an ihre Kapazitäten stoßen.
Damit das Testbild und die konvertierten Versionen gelöscht werden konnten, musste ein Web Service in der \gls{OracleUCM}-Anwendung angelegt werden.

Alle Überwachungselemente sind im Webinterface vom Nagios aufgeführt und erlauben den Administratoren einen schnellen Überblick über die korrekte Funktion der Anwendung bzw. über die aufgetretenen Fehler.
Die korrekte Benachrichtigung über Störungen konnte durch die Simulation der Fehlzustände in der virtuellen Maschine sichergestellt werden.
\newpage
\section{Ausblick}
Bei einer Überwachung ist es notwendig zuvor verschiedene Schwellwerte zu setzen.
An diesen Werten kann die Überwachungssoftware festlegen, ob ein Objekt einen kritischen Zustand erreicht hat oder nicht.
Für die Ermittelung dieser Größen müssen die Werte der Überwachungselemente über einen längeren Zeitraum beobachtet und analysiert werden.
Aufgrund des begrenzten zeitlichen Rahmens dieser Arbeit konnte dies nicht vollständig umgesetzt werden, so dass es während dem Betrieb fortgesetzt werden muss.
Hierunter fallen vor allem spezifische Merkmale eines bestimmten Servers.
Eine ungewöhnlich hohe Prozessorauslastung, die durch eine zeitlich gesteuerte Sicherung entstehen kann, sollte von der Überwachungssoftware nicht als Fehlverhalten interpretiert werden.
Auch die Liste der Stopwörter für die Auswertung der Logdateien muss für neue bisher unbekannte Fehler immer wieder erweitert werden.

Das entwickelte Plugin für die Benutzersimulation kann auch mit zusätzliche Funktionen versehen werden.
Durch die Verwendung von anderen Web Services können weitere Funktionalitäten überprüft werden.
Dabei kann die Benutzersimulation auch auf anderen Dokumenten-Management-Systemen eingesetzt werden.
Das Plugin kann leicht angepasst werden um, anstatt eines Testbildes, beispielsweise den Check-In, Indizierung und Konvertierung einer \gls{PDF}-Testdatei oder eines Word-Dokuments zu überwachen.
%Hierfür müsste nur anstatt eines Testbildes beispielsweise eine \gls{PDF}-Testdatei oder Word-Dokument verwendet werden.
%Für die Validierung der Indizierung und Konvertierung müsste das Plugin nur leicht angepasst werden.

%Die angepassten Konfigurationsdateien können durch ihren Aufbau einfach auf einen anderen Nagios exportiert und verwendet werden.

%\begin{itemize}
%
%\item Zusammenfassung
%\item Eigener Nagios-Server aufgesetzt / Nagios
%\item Verwendung der Bilddatenbank VM / Oracle UCM
%\item Überwachung der verschiedenen Ebenen wurde realisiert
%\item Dabei auf Sicherheit geachtet (Port, Passwort, Verschlüsselung)
%
%\item Ausblick
%\item Übernahme der Konfigurationsdateien auf vorhandenen Nagios-Server möglich
%\item Überwachung von anderen Content-Servern via Webservice möglich (PDF, DOC …)
%
%
%\item Geeignete Stopwörter für Logdateien müssen noch gefunden / eruiert werden
%\item Passende Schwellwertdefinitionen können erst nach einer gewissen Laufzeit festegelegt werden
%\item Export der entwickelten Überwachung auf den produktiven Haupt-Nagios Server
%\end{itemize} \newpage
%\section{Einleitung}
In Unternehmen werden den Benutzern verschiedene IT-Dienstleistungen angeboten.
Eine Dienstleistung ist die Bereitstellung einer Plattform für die zentrale Speicherung, Bearbeitung und Verwaltung von Dokumenten.
Dabei können diese Dokumente Dateien in unterschiedlicher Form sein wie Microsoft Word Dateien, Excel Tabellen, Dateien im Portable Document Format (\gls{PDF}) oder auch Bilder in vielen weiteren Formaten.

Die Darbietung / Die Versorgung / Das Angebot / Das Bereitstellen einer solchen Dienstleistung wird mit einem Dokumenten-Management-System (\gls{DMS}) realisiert.
Vorteile, die für den Einsatz eines Dokumenten-Management-Systems sprechen, sind die Möglichkeiten, die sich durch die computergestützte Erfassung und Indexierung (auch Indizierung genannt) der Dokumente eröffnen.
Die bekannteste / geläufigste Implementierung dieser Möglichkeiten ist die automatische Verschlagwortung von Dokumenten für die Zuordnung von Deskriptoren zu einem Dokument zur Erschließung der darin enthaltenen Sachverhalte.
Durch die Aufnahme dieser Schlagwörter in einen Suchindex können Anwender bestimmte Dokumenten gezielt finden und anfordern.
Ein wichtiger Punkt ist die Versionierung der Dateien in einem Dokumenten-Management-System.
Dadurch können auf ältere Versionen der Dokumente zugegriffen werden, Änderungen angezeigt oder (komplett) zurückgesetzt werden.
Durch die zentrale Struktur / Zugriff eines Dokumenten-Management-Systems ist es notwendig Zugriffsrichtlinien für die Dokumenten zu implementieren, die anhand von Gruppen- und Benutzerinformationen gesetzt / ausgewählt werden.

Aufgrund der Vielzahl an angebotenen Dienstleistungen ist es schwierig herauszufinden, ob die angebotenen Dienstleistungen noch fehlerfrei arbeiten oder aus welchem Grund die Benutzer nicht mehr auf einen Dienst zugreifen können.
Für diesen Zweck wurden Überwachungssysteme entwickelt die den Status der verschiedenen Komponenten und den davon abhängigen Diensten überwachen und bei Veränderungen die Verantwortlichen darüber informiert.

Für einen möglichst störungsfreien Betrieb ist es notwendig, dass die Ergebnisse der Überwachung in periodischen Zeitabständen erneuert werden, damit ein auftretendes Problem schnellstmöglich erkannt und behoben werden kann.
Das Überwachungssystem sollte so implementiert werden, dass Fehler erkannt werden, bevor die Nutzung der angebotenen Dienstleistungen davon beeinträchtigt werden.
Dabei muss die zusätzliche Belastung der Netzwerkes und der überwachten Objekte durch die Überwachung eingeplant, die verwendete Netzwerkstruktur und die dadurch entstehende Abhängigkeit (von Netzwerkknoten) beachtet und sicherheitstechnische Aspekte einer automatischen Überwachung bedacht werden.\\


Im Laufe dieser Arbeit soll eine Überwachung eines Dokumenten-Management-Systems unter Berücksichtigung der Funktions- und Arbeitsweise des eingesetzten Dokumenten-Management-Systems durch eine Open Source (Netzwerk)Überwachungsandwendung realisiert werden.
%popular open source computer system and network monitoring software application 
 

%Einleitung halt. Test
%Kurz was ist Nagios, warum überhaupt überwachen?
%Was soll überwacht werden -> Stellent/UCM kurz was ist das? Warum gerade das überwachen -> Aktive Benutzung durch User - kritisch \newpage
%\section{Einleitung}
In Unternehmen werden den Benutzern verschiedene IT-Dienstleistungen angeboten.
Eine Dienstleistung ist die Bereitstellung einer Plattform für die zentrale Speicherung, Bearbeitung und Verwaltung von Dokumenten.
Dabei können diese Dokumente Dateien in unterschiedlicher Form sein wie Microsoft Word Dateien, Excel Tabellen, Dateien im Portable Document Format (\gls{PDF}) oder auch Bilder in vielen weiteren Formaten.

Die Darbietung / Die Versorgung / Das Angebot / Das Bereitstellen einer solchen Dienstleistung wird mit einem Dokumenten-Management-System (\gls{DMS}) realisiert.
Vorteile, die für den Einsatz eines Dokumenten-Management-Systems sprechen, sind die Möglichkeiten, die sich durch die computergestützte Erfassung und Indexierung (auch Indizierung genannt) der Dokumente eröffnen.
Die bekannteste / geläufigste Implementierung dieser Möglichkeiten ist die automatische Verschlagwortung von Dokumenten für die Zuordnung von Deskriptoren zu einem Dokument zur Erschließung der darin enthaltenen Sachverhalte.
Durch die Aufnahme dieser Schlagwörter in einen Suchindex können Anwender bestimmte Dokumenten gezielt finden und anfordern.
Ein wichtiger Punkt ist die Versionierung der Dateien in einem Dokumenten-Management-System.
Dadurch können auf ältere Versionen der Dokumente zugegriffen werden, Änderungen angezeigt oder (komplett) zurückgesetzt werden.
Durch die zentrale Struktur / Zugriff eines Dokumenten-Management-Systems ist es notwendig Zugriffsrichtlinien für die Dokumenten zu implementieren, die anhand von Gruppen- und Benutzerinformationen gesetzt / ausgewählt werden.

Aufgrund der Vielzahl an angebotenen Dienstleistungen ist es schwierig herauszufinden, ob die angebotenen Dienstleistungen noch fehlerfrei arbeiten oder aus welchem Grund die Benutzer nicht mehr auf einen Dienst zugreifen können.
Für diesen Zweck wurden Überwachungssysteme entwickelt die den Status der verschiedenen Komponenten und den davon abhängigen Diensten überwachen und bei Veränderungen die Verantwortlichen darüber informiert.

Für einen möglichst störungsfreien Betrieb ist es notwendig, dass die Ergebnisse der Überwachung in periodischen Zeitabständen erneuert werden, damit ein auftretendes Problem schnellstmöglich erkannt und behoben werden kann.
Das Überwachungssystem sollte so implementiert werden, dass Fehler erkannt werden, bevor die Nutzung der angebotenen Dienstleistungen davon beeinträchtigt werden.
Dabei muss die zusätzliche Belastung der Netzwerkes und der überwachten Objekte durch die Überwachung eingeplant, die verwendete Netzwerkstruktur und die dadurch entstehende Abhängigkeit (von Netzwerkknoten) beachtet und sicherheitstechnische Aspekte einer automatischen Überwachung bedacht werden.\\


Im Laufe dieser Arbeit soll eine Überwachung eines Dokumenten-Management-Systems unter Berücksichtigung der Funktions- und Arbeitsweise des eingesetzten Dokumenten-Management-Systems durch eine Open Source (Netzwerk)Überwachungsandwendung realisiert werden.
%popular open source computer system and network monitoring software application 
 

%Einleitung halt. Test
%Kurz was ist Nagios, warum überhaupt überwachen?
%Was soll überwacht werden -> Stellent/UCM kurz was ist das? Warum gerade das überwachen -> Aktive Benutzung durch User - kritisch 

%\newpage

\section{Abbildungsverzeichnis}
\listoffigures

\newpage
\renewcommand{\refname}{} 
\section{Literaturverzeichnis}
\begin{thebibliography}{xxxxxx}
	 \bibitem[Barth08]{Barth08}Wolfgang Barth (2008) "`Nagios - System- und Netzwerk-Monitoring"' 2. Auflage, \newline ISBN13: 978-3-937514-46-8, \newline Stand: ????, Einsichtnahme: 14.05.2009
	 \bibitem[Huff06]{Huff06} Brian Huff (2006) "`The Definitive Guide to Stellent Content Server Development"', \newline ISBN13: 978-1-59059-684-5, \newline Stand: ????, Einsichtnahme: 15.05.2009
\end{thebibliography}\newpage

\end{document}



