\documentclass[12pt, a4paper, headsepline]{article}
%\documentclass[12pt, a4paper, headsepline]{scrbook}
\usepackage[utf8x]{inputenc}
\usepackage[T1]{fontenc}
\usepackage[ngerman]{babel}
\usepackage{graphicx} 
\usepackage[automark]{scrpage2}
\usepackage{hyperref}
\usepackage{subfigure} 
\usepackage{ae} 
\usepackage{amsmath}
\usepackage{amsfonts}
\usepackage{amssymb}
\usepackage{floatflt}
\usepackage{pdfpages}
%\usepackage{fancyhd}
\usepackage{threeparttable}
\usepackage{endnotes}
%\usepackage{struktex}
\usepackage{rotating}
\usepackage{listings}
%\usepackage{listingsutf8}
\usepackage[style=super3colheader]{glossaries}
\usepackage{verbatim} %Mehrzeilige Kommentare durch \begin{content} möglich  ??
\usepackage{blindtext} % Erzeugung von lngeren Textpassage
\usepackage{xcolor}  

\definecolor{hellgrau}{rgb}{0.9,0.9,0.9}
\definecolor{darkgreen}{rgb}{0,0.5,0}
\definecolor{darkblue}{rgb}{0,0,2}

\lstset{inputencoding=utf8x, extendedchars=\true, numbers=left, numberstyle=\small, numbersep=5pt, basicstyle=\ttfamily\scriptsize, backgroundcolor=\color{hellgrau}, commentstyle=\color{darkgreen}, keywordstyle=\color{blue}, showspaces=false, showtabs=false, emph={String}, emphstyle=\color{blue}, emph={Boolean}, emphstyle=\color{blue}, showstringspaces=false}

\author{Andreas Paul}
\usepackage[bottom,hang]{footmisc}
\setlength{\footnotemargin}{0pt}
%\addto{\captionsngerman}{\renewcommand*{\listfigurename}{}}
\glossarystyle{long3colheader}
\renewcommand*{\glsgroupskip}{}

\makeatletter
\makeindex
\makeglossaries
\glsenablehyper 
\renewcommand{\entryname}{K\"urzel}
\renewcommand{\descriptionname}{Beschreibung}
%\renewcommand{\glspageheader}{S.}
%\renewcommand*\l@section{\@dottedtocline{1}{1.5em} {2.3em}}
%\renewcommand*\l@subsection{\@dottedtocline{2}{3.8 em}{3.2em}}
%\renewcommand*\l@subsubsection{\@dottedtocline{3}{ 7.0em}{4.1em}}
%\renewcommand*\l@paragraph{\@dottedtocline{4}{10em }{5em}}
%\renewcommand*\l@subparagraph{\@dottedtocline{5}{1 2em}{6em}}
\renewcommand*\l@figure{\@dottedtocline{1}{1.5em}{ 2.3em}}
\newcommand{\pictext}[1]{\glqq\textit{#1}\grqq }
\renewcommand{\arraystretch}{1.5} %Tabellenreihenhoehe
% Abb. anstatt Abbildung verwenden %
\renewcommand{\figurename}{Abb.}
% Tab. anstatt Tabelle verwenden %
\renewcommand{\tablename}{Tab.} 
\renewcommand\subitem{\@idxitem \hspace*{30\p@}}
 % Tiefe der Nummerierung einstellen %
\setcounter{secnumdepth}{3} 
\makeatother 
\renewcommand{\refname}{Quellenverzeichnis} 
\linespread{1.50} %zeilenabstand 1.5}
\setlength{\headheight}{4.0\baselineskip}
\setlength{\fboxsep}{3mm}
%\textwidth=5.7in
\begin{document}
\setlength{\parindent}{0mm}
\includepdf{Titelblatt.pdf}

\newglossaryentry{DMS}{name={DMS},description={Dokumenten-Management-System},text={DMS}}
\newglossaryentry{SOA}{name={SOA}, description={Service-Oriented Architecture},text={SOA}}
\newglossaryentry{HTTP}{name={HTTP}, description={Hypertext Transfer Protocol},text={HTTP}}
\newglossaryentry{SOAP}{name={SOAP}, description={Simple Object Access Protocol},text={SOAP}}
\newglossaryentry{XML}{name={XML}, description={Extensible Markup Language},text={XML}}
\newglossaryentry{IP}{name={IP},description={Identifikation Protokoll},text={IP}}
\newglossaryentry{CMS}{name={CMS}, description={Content-Management-System},text={CMS}}
\newglossaryentry{ECM}{name={ECM}, description={Enterprise-Content-Management},text={ECM}}
\newglossaryentry{UCM}{name={UCM}, description={Universal-Content-Management},text={UCM}}
\newglossaryentry{FTP}{name={FTP}, description={File Transfer Protokoll},text={FTP}}
\newglossaryentry{AIIM}{name={AIIM}, description={Association for Information and Image Management},text={AIIM}}
% - Die AIIM ist eine Gesellschaft von internationalen Herstellern und Anwendern von Informations- und Dokumenten-Mangement-Systemen.
\newglossaryentry{SSH}{name={SSH}, description={Secure Shell},text={SSH}}
% - Durch eine Secure Shell kann man sich eine verschlüsselte Netzwerkverbindung zum entfernten Rechner aufbauen.
\newglossaryentry{LDAP}{name={LDAP}, description={Lightweight Directory Access Protocol},text={LDAP}}
\newglossaryentry{NSCA}{name={NSCA}, description={Nagios Service Check Acceptor},text={NSCA}}
\newglossaryentry{MIB}{name={MIB}, description={Management Information Base},text={MIB}}
\newglossaryentry{FQDN}{name={FQDN}, description={Fully Qualified Domain Name},text={FQDN}}
\newglossaryentry{DNS}{name={DNS}, description={Domain Name System},text={DNS}}
\newglossaryentry{NRPE}{name={NRPE}, description={Nagios Remote Plugin Executor},text={NRPE}}
\newglossaryentry{OracleUCM}{name={OracleUCM}, description={Content-Management-System von Oracle},text={Oracle UCM}}
\newglossaryentry{DoS}{name={DoS}, description={Denial of Service},text={DoS}}
\newglossaryentry{NCI}{name={NCI}, description={None-Coded Information},text={NCI}}
\newglossaryentry{CI}{name={CI}, description={Coded Information},text={CI}}
\newglossaryentry{PDF}{name={PDF}, description={Portable Document Format},text={PDF}}
\newglossaryentry{OCR}{name={OCR}, description={Optical Character Recognition},text={OCR}}
\newglossaryentry{WSDL}{name={WSDL}, description={Web Services Description Language},text={WSDL}}
\newglossaryentry{SNMP}{name={SNMP}, description={Simple Network Management Protocol},text={SNMP}}
\newglossaryentry{W3C}{name={W3C}, description={World Wide Web Consortium},text={W3C}}
\newglossaryentry{RPC}{name={RPC}, description={Remote Procedure Call},text={RPC}}
\newglossaryentry{UDDI}{name={UDDI}, description={Universal Description, Discovery and Integration protocol},text={UDDI}}
%%%%%
%
% glossary entries
%
%%%%% 
\setlength{\textheight}{630pt}
\addtolength{\voffset}{-40pt}


%\pagenumbering{roman}

%\pagestyle{scrheadings}
%\chead{\leftmark}
%\ihead{\includegraphics[width=0.06\textwidth]{bilder/fz2.png}}

\newpage
 \newcommand{\cappclause}[2]{
   \thispagestyle{empty}
   \
\begin{center}
\begin{Large}\textbf{Eidesstattliche Erklärung}\end{Large}
\end{center}   
   
   \vfill

Hiermit erkläre ich an Eides Statt, dass ich die vorliegende Arbeit selbst angefertigt habe; die aus fremden Quellen direkt oder indirekt übernommenen Gedanken sind als solche kenntlich gemacht. 


Die Arbeit wurde bisher keiner Prüfungsbehörde vorgelegt und auch noch nicht veröffentlicht.


   Ich versichere hiermit wahrheitsgemäß, die Arbeit bis auf die dem
   Aufgabensteller bereits bekannte Hilfe selbständig angefertigt, alle
   benutzten Hilfsmittel vollständig und genau angegeben und alles kenntlich
   gemacht zu haben, was aus Arbeiten anderer unverändert oder mit
   Abänderung entnommen wurde.\\

   \begin{center}
       \raggedright #2\\
       \vspace*{-2ex}
       \dotfill\\
       Ort, Datum \hfill (#1)\\
   \end{center}
}
\renewcommand{\baselinestretch}{1.0}
 
 \cappclause{Andreas Paul}          % author name
        {Karlsruhe, den \today}    % location, date (for legal clause)
\newpage


\pagenumbering{arabic}

\pagestyle{scrheadings}
\ihead{\includegraphics[width=0.06\textwidth]{bilder/fzk2.png}}
%\chead{\rightmark}
\chead{\leftmark - v0.99}
\cfoot{Andreas Paul - Forschungszentrum Karlsruhe}


\section{Abstract}


Dokumenten-Management-Systeme bilden eine zentralen Dienstleistung im Karlsruhe Insitute of Technology.
Diese Systeme sind komplex aufgebaut und benötigen ausgefeilte Kontrollmaßnahmen / Überwachungsroutinen um einen stabilen Betrieb zu garantieren / ermöglichen.
Zum gegenwärtigen Zeitpunkt gibt es keine ausgereifte Überwachungssoftware, die diese Aufgabe zufriedenstellend erfüllt.

Diese Bachelorarbeit beschreibt die Entwicklung von neuen Werkzeugen für die Open Source-Überwachungssoftware Nagios um das Dokumenten-Management-System \gls{OracleUCM}\footnote{Oracle Universal-Content-Management} auf Fehlverhalten hin zu kontrollieren.
Diese sogenannten Plugins lassen sich in bestehende Nagios-basierende Systeme einbinden und erweitern deren Bandbreite an zu überwachenden Elementen.
Da Hauptaugenmerk lag dabei, auf der Simulation von Benutzerverhalten und der Erkennung der dabei auftretenden Fehler.
Die Verantwortlichen sollen durch die Benachrichtigung dieser Fehler sofort alarmiert werden und durch die unterschiedliche Tiefe der Überwachungstests die Problemquellen eingrenzen können.
%, welche möglichst zu Laufzeit erkannt und deren Fehl
%Meldungen können die Problemquellen vom Administrator gefunden und eventuell behoben werden,
%Diese Bachelorarbeit beschäftigt sich mit der Entwicklung eines Plugins für die Open Source-Überwachungsoftware Nagios intensiven Überwachung des Dokumenten-Management-Systems \gls{OracleUCM}\footnote{Oracle Universal-Content-Management} durch  um proaktiv auftretende Fehler zu entdecken.
%Dabei werden die Grundlagen von Dokumenten-Management-Systemen, Aufbau von \gls{OracleUCM} und Nagios beleuchtet und beschrieben.
%Dadurch kann eine geeignete Methode aus den unterschiedlichen Überwachungsmethoden von Nagios ausgewählt werden.
%Notwendige Kenntnisse über Service-orientierte Architektur (\gls{SOA}) und Web Services werden für die Umsetzung angeeignet.

Die Überwachung besteht aus den Ebenen: Statusabfragen, Funktionalitätstests, Auswertung von Logdateien und Benutzersimulation.
Auf dem Windows-Servers der \gls{OracleUCM}-Anwendung wird ein passender Nagios-Agent installiert, der aus einer vorherigen Übersicht ausgewählt wurde.
Die Konfiguration und der Einsatz von bereits erhältlichen Nagios-Plugins decken die ersten drei Ebenen ab.
Die automatisierte Benutzersimulation verwendet verschiedene Web Services der \gls{OracleUCM}-Anwendung.


%Ein Abstract ist eine prägnante Inhaltsangabe, ein Abriss ohne Interpretation und Wertung einer wissenschaftlichen Arbeit.

%Zusammenfassung von allem.

%Aufgabenstellung, Erwartendes Ergebnis
%\begin{itemize}
%\item Objektivität: soll sich jeder persönlichen Wertung enthalten
%\item Kürze: soll so kurz wie möglich sein
%\item Verständlichkeit: klare, nachvollziehbare Sprache und Struktur
%\item Vollständigkeit: alle wesentlichen Sachverhalte sollen explizit enthalten sein
%\item Genauigkeit: soll genau die Inhalte und die Meinung der Originalarbeit wiedergeben
%\end{itemize}
 \newpage

%\thispagestyle{empty}
\renewcommand{\contentsname}{Inhalt}
%\thispagestyle{empty}
\tableofcontents
%\thispagestyle{empty}
\newpage



%\setlength{\textheight}{660pt}
%\addtolength{\voffset}{+90pt}




\section{Einleitung}

%Mit dem Zusammenschluss......ist eine Einrichtung mit ??? Angestellten, ??? Studierenden und ca. 300 externen Mitarbeitern und Gästen entstanden. Die IT-Infrastruktur für den organisatorischen und wissenschaftlichen Betrieb liegt in der Verantwortung des ...(SCC)...das aus der Verschmelzung von ...und...hervorgegangen ist. Für alle Schichten der IT-Infrastruktur und alle angebotenen Dienstleistungen muss der Betrieb duch das Rechenzentrum überwacht werden.

%Die Überwachung des Dokumentenmanagmentsystem, eines wichtigen zentralen Dienstes, war Ziel dieser Arbeit.



%8000 Wissenschaftler und Mitarbeiter, 18000 Studierende mit
%einem Jahresbudget von 0.5 Mrd €


Mit dem Zusammenschluss des Forschungszentrum Karlsruhe und der Universität Karlsruhe (TH) zum Karlsruhe Insitute of Technology (KIT) ist eine Einrichtung mit 8000 Wissenschaftlern und Mitarbeitern, 18000 Studierenden und circa 300 externen Mitarbeitern und Gästen entstanden.

%IWR FZK + RZ Uni

Die IT-Infrastruktur für den organisatorischen und wissenschaftlichen Betrieb liegt in der Verantwortung des Steinbuch Center für Computing (SCC), das aus der Verschmelzung des Rechenzentrums der Universität und dem Institut für Wissenschaftliches Rechnen (IWR) hervorgegangen ist.

Für alle Schichten der IT-Infrastruktur und alle angebotenen Dienstleistungen muss der Betrieb durch das Rechenzentrum überwacht werden.
Die Überwachung des Dokumenten-Management-System (\gls{DMS}), einer zentralen Dienstleistung, ist Ziel dieser Arbeit.


%In Unternehmen werden den Benutzern verschiedene IT-Dienstleistungen angeboten.
%Eine Dienstleistung ist die Bereitstellung einer Plattform für die zentrale Speicherung, Bearbeitung und Verwaltung von Dokumenten.

Die Hauptaufgaben eines Dokumenten-Management-Systems sind die zentrale Speicherung, Bearbeitung und Verwaltung von Dokumenten.
Dabei können diese Dokumente Dateien in unterschiedlicher Form sein wie Microsoft Word Dateien, Excel Tabellen, Dateien im Portable Document Format (\gls{PDF}) oder auch Bilder.
%Die wichtigsten Funktionen

%%Danach kannst Du mit dem Abschnitt ...Eine Dienstleistung ist die Bereitstellung .....in umgeschriebener Form weitermachen. Punkte aus 4.2 aber nicht schon detailliert erläutern, sondern anreissen und eher darauf hinweisen wie wichtig solche Funktionen, wie Revisionierung z.B. für Patentverträge sind. Bei der Beschreibung von "Überwachung" hast Du auch den Nutzen für den RZ-Betrieb und den Kunden in den Vordergrund der "einleitenden Worte" gestellt.

%Die Darbietung / Die Versorgung / Das Angebot / Das Bereitstellen einer solchen Dienstleistung wird mit einem Dokumenten-Management-System (\gls{DMS}) realisiert.
%Vorteile, die für den Einsatz eines Dokumenten-Management-Systems sprechen, sind die Möglichkeiten, die sich durch die computergestützte Erfassung und Indexierung (auch Indizierung genannt) der Dokumente eröffnen.
%Die bekannteste / geläufigste Implementierung dieser Möglichkeiten ist die automatische Verschlagwortung von Dokumenten für die Zuordnung von Deskriptoren zu einem Dokument zur Erschließung der darin enthaltenen Sachverhalte.
%Durch die Aufnahme dieser Schlagwörter in einen Suchindex können Anwender bestimmte Dokumenten gezielt finden und anfordern.
%Ein wichtiger Punkt ist die Versionierung der Dateien in einem Dokumenten-Management-System.
%Dadurch können auf ältere Versionen der Dokumente zugegriffen werden, Änderungen angezeigt oder (komplett) zurückgesetzt werden.
%Durch die zentrale Struktur / Zugriff eines Dokumenten-Management-Systems ist es notwendig Zugriffsrichtlinien für die Dokumenten zu implementieren, die anhand von Gruppen- und Benutzerinformationen gesetzt / ausgewählt werden.

Aufgrund der Vielzahl an angebotenen Dienstleistungen ist es schwierig herauszufinden, ob die angebotenen Dienstleistungen noch fehlerfrei arbeiten oder aus welchem Grund die Benutzer nicht mehr auf einen Dienst zugreifen können.
Für diesen Zweck wurden Überwachungssysteme entwickelt, die den Status von verschiedenen Komponenten und den davon abhängigen Diensten überwachen und bei Veränderungen die Verantwortlichen darüber informieren.

Für einen möglichst störungsfreien Betrieb ist es notwendig, dass die Ergebnisse der Überwachung in periodischen Zeitabständen erneuert werden, damit ein auftretendes Problem schnellstmöglich erkannt und behoben werden kann.
Das Überwachungssystem soll so implementiert werden, dass Fehler erkannt werden, bevor die Nutzung der angebotenen Dienstleistungen davon beeinträchtigt wird.
Dabei muss die zusätzliche Belastung der Netzwerkes und der überwachten Objekte durch die Überwachung eingeplant, die verwendete Netzwerkstruktur und die dadurch entstehende Abhängigkeit (von Netzwerkknoten) beachtet und sicherheitstechnische Aspekte einer automatischen Überwachung bedacht werden.\\


Im Laufe dieser Arbeit soll eine Überwachung eines Dokumenten-Management-Systems unter Berücksichtigung der Funktions- und Arbeitsweise des eingesetzten Dokumenten-Management-Systems durch eine Open Source Überwachungsanwendung realisiert werden.
%popular open source computer system and network monitoring software application 
 

%Einleitung halt. Test
%Kurz was ist Nagios, warum überhaupt überwachen?
%Was soll überwacht werden -> Stellent/UCM kurz was ist das? Warum gerade das überwachen -> Aktive Benutzung durch User - kritisch 
\newpage

\section{Aufgabenstellung}

Um den Mitarbeitern des Karlsruhe Insitute of Technology eine möglichst ausfallsichere Plattform für die zentrale Speicherung, Bearbeitung und Verwaltung von Dokumenten anbieten zu können, soll eine Überwachung implementiert werden.
Diese Überwachung soll nicht nur die Anwendung, sondern auch den darunterliegenden Server bezüglich seiner Systemressourcen berücksichtigen.
Dabei müssen Überwachungselemente gefunden werden, mit deren Überprüfung der eindeutige Zustand der Anwendung festgestellt und der störungsfreie Betrieb sichergestellt werden kann.

Für die Verwaltung von Webseiten, Dokumenten und Bildern wird das Dokumenten"=Management"=System \gls{OracleUCM} der Firma Oracle eingesetzt.
Um die zu überwachenden Objekte zu ermitteln, ist das Verständnis über den Aufbau und der spezifischen Funktions- und Arbeitsweise des verwendeten Dokumenten-Management-Systems notwendig.

Als Überwachungssoftware wird die Open Source"=Software Nagios verwendet.
%Damit der fehlerfreie Betrieb von \gls{OracleUCM} als Dienst durch die Überwachung der ermittelten Überwachungselemente eindeutig festgestellt werden kann, muss sich mit dem Aufbau, der internen Funktionsweise und den verschiedenen Methoden bezüglich der Ermittlung der Statusinformationen untersucht werden.
Zur Realisierung der Überwachung muss auf die interne Logik und auf die verschiedenen Methoden bezüglich der Ermittlung der Statusinformationen eingegangen werden.
Dabei soll eine Übersicht über die unterschiedlichen Überwachungsmethoden von Nagios erstellt und unter Berücksichtigung des späteren Einsatzes bewertet werden.
%Hierbei sind für die spätere Umsetzung beispielsweise die verschlüsselte Datenübertragung zwischen Überwachungs- und Anwendungsserver ein Kriterium.
Mit den ausgewählten Methoden soll die Überwachung auf verschiedenen Ebenen realisiert werden.

Die Klassifizierung der Überwachungselemente ergibt sich aus der Gewichtung der einzelnen Elemente.
%In diesem Schritt werden die zuvor gefunden 
%Die Erreichbarkeit über das Netzwerk bildet die Grundlage für darüber liegende Überwachungsobjekte, wie beispielsweise der Zustand eines Prozesses.
Dabei soll die Anwendung auch reaktiv durch eine Auswertung von Logdateien auf Fehler überwacht werden.


Zur eindeutigen Erkennung von Fehlern, die während der Benutzung durch Anwender auftreten, sollen die typischen Aktionen der Benutzer simuliert werden. 
Für die Realisierung dieser Benutzersimulation muss die Anwendung über eine Schnittstelle verfügen, die sich durch ein Programm über das Netzwerk ansprechen lässt.
Dieses Programm soll die Benutzeraktionen automatisiert durchführen und der Überwachungssoftware Nagios die Ergebnisse der einzelnen Schritte übermitteln, damit der Fehlerzustand sofort erkannt und gleichzeitig seine Ursache eingegrenzt werden kann.

Dabei müssen bei der Programmentwicklung mögliche Konsequenzen aufgrund verschiedener Szenarien bedacht werden.
Sollte die Anwendung bereits durch eine Vielzahl von Benutzern stark belastet sein, wird dadurch auch der Ablauf der Benutzersimulation verzögert.
%In diesem Fall soll die Überwachungssoftware bzw. Benutzersimulation keine falsche Informationen melden.
Eine solche Verzögerung soll von der Überwachungssoftware bzw. Benutzersimulation bei der Auswertung berücksichtigt werden.

Die Nutzung der Anwendung durch die eigentlichen Benutzer darf dabei nicht beeinträchtigt werden.
Da die Ausführung der Benutzersimulation durch Nagios in kurzen Zeitabständen periodisch aufgerufen wird, müssen auch langfristige Auswirkungen wie das Überlaufen der Datenbank der Anwendung oder die Überfüllung des Festplattenspeichers des Anwendungsservers bedacht werden.

Für die Entwicklungsumgebung wird ein eigener Nagios-Server eingesetzt, deshalb muss die entwickelte Lösung auf den bereits vorhanden Nagios-Server exportierbar sein.

%\textit{Export auf vorhanden Nagios-Server ermöglichen.

%Hinzufügen von Services ermöglichen%
%Konfigurieren der Überwachungsparameter ermöglichen (CPU last, df) protokollieren} \newpage
\section{Grundlagen}
In diesem Kapitel werden die grundlegenden Begriffe erläutert, die für das Verständnis der weiterführenden Kapitel notwendig sind.

\subsection{Überwachungssysteme}
\label{monitor}
Überwachungssysteme wurden für den Zweck entwickelt den Status von verschiedenen Objekten meist über das Netwerk zu überwachen und im Falle einer Statusänderung diese Information an die zugewiesenen Kontaktpersonen weiterleitet.

%Bei diesen Objekten kann es sich um viele verschiedene Komponenten handeln.
Generell unterscheidet man zwischen der Überwachung ermöglichten zu Grunde liegenden Hardware den so genannten Hosts und den auf diesen Hardwarekomponenten aufsitzenden Diensten auch Services genannt.

Unter Hosts fallen nicht nur Server bzw. Computer, sondern auch Switches, Router oder auch dedizierte Überwachungshardware wie Sensoren für Temperatur, Luftfeuchtigkeit oder Rauchmelder.
Die Services dieser Hosts weichen je nach Art der Hosts stark voneinander ab.
Auf einem Server kann als Service ein Webserver im Betrieb sein, dessen Funktionalität sich über einen Aufruf einer Webseite überprüfen lässt.
Bei einem Switch können beispielsweise als Service die Übertragungsrate, der Paketverlust oder der Portzustand überwacht werden.

Sehr wichtig ist bei einem Überwachungssystem die Gewichtung der erhaltenen Überwachungsinformationen.


%\begin{center}
%was ist wichtig was nicht, Gewichtung, Klassifizierung, Organisationsstrategie
%Host,Services erklären
%\end{center}
\newpage
Vor der Einführung eines Überwachungssystems muss sich mit den folgenden Punkten auseinandergesetzt werden.

\subsubsection{Ressourcenbelastung}
Die Einführung einer Überwachungssoftware bringt bei größeren Serverlandschaften eine nicht zu verachtende Netzwerk- und Prozessorbelastung mit sich.
Dabei unterscheidet Josephsen die anfallende Belastung in zwei unterschiedliche Arten der Überwachung\footnote{Quelle: \cite{Jose07} S. 4}:

\paragraph{Zentralisierte Überwachung}
Die Durchführung der Überprüfungen findet durch einen zentralen Überwachungsserver statt, der die Informationen über die einzelnen Hosts und Services über das Netzwerk abfragt.
Diese Methode ist in der Regel vorzuziehen, da hierbei die zu überwachenden Geräte weniger belastet werden und die Konfiguration der einzelnen Kontrollschritte zentral möglich ist.

\begin{figure}[ht]
	\centering
	   \fbox{\includegraphics[width=0.5\textwidth]{bilder/dist_mon12.png}}
	   %\fbox{Quelle: \cite{Jose07} S. 5}
		\caption[Zentralistische Bearbeitung]{Zentralistische Bearbeitung}
		\label{distmon2}
\end{figure}
%\footnotetext{Quelle: \url{http://www.nagios-wiki.de/\_media/nagios/howtos/dist\_mon.png}}


\paragraph{Dezentralisierte Überwachung}
Bei einer sehr hohen Anzahl von zu überwachenden Objekten ist eine zentralisierte Ausführung nicht mehr von einem einzelnen Server tragbar.
In diesem Fall ist das Überwachungssystem darauf angewiesen, dass die einzelnen Hosts die kontrollierenden Überprüfungen selbständig durchführen und deren Ergebnisse an den Überwachungsserver weiterzuleiten.

%Kascadierende Überwachungssysteme kannst Du noch erwähnen

%Uberwachungsredundanzen vermeiden: Dein Beispiel mit dem Port führt in beiden Fällen dazu, dass der Test 1 überflüssig ist. Wenn die Webseite über 2 Ports abgefragt wird, z.B. 8000 Intranet, 8080 Internet, so wäre der Test 1 sinnvoll
\begin{figure}[ht]
	\centering
	   \fbox{\includegraphics[width=0.45\textwidth]{bilder/dist_monp.png}}
	   %\fbox{Quelle: \cite{Jose07} S. 5}
		\caption{Ausgelagerte Bearbeitung}
		\label{distmonp}
\end{figure}

Um nicht komplett vom Überwachungsserver abhängig zu sein, kann ein zweiter Überwachungsserver bzw. weitere Server hinzugefügt werden.
Diese können bei einem Ausfall des Hauptüberwachungsservers die Verantwortlichen informieren oder die zu überwachenden Objekte zur Lastenteilung untereinander aufteilen.


%Nagios bietet zusätzlich noch eine weitere, dritte Möglichkeit durch das \textit{Distributed Monitoring} (Verteilte Überwachung) an, siehe Kapitel \ref{dismoni}.

\subsubsection{Netzwerkstruktur und Abhängigkeiten}
Die Überwachung von Hosts und Services über das Netzwerk erzeugt normalerweise immer zusätzlichen \gls{IP}-Traffic.
Das bedeutet, dass jede Überquerung weiterer Netzwerkknoten, die zwischen dem Überwachungsserver und den zu überwachenden Geräten liegen, eine weitere Belastung für das Netzwerk bedeutet, sowie eine Abhängigkeit zwischen Host und Server einführt.

\begin{figure}[ht]
	\centering
	   \fbox{\includegraphics[width=0.5\textwidth]{bilder/dependent.png}}
	   %\fbox{Quelle: \cite{Jose07} S. 5}
		\caption[Zusätzliche Netzwerkabhängigkeit und Netzwerkbelastung]{Zusätzliche Netzwerkabhängigkeit und Netzwerkbelastung\protect\footnote}
		\label{depend}
\end{figure}
\footnotetext{Quelle: \cite{Jose07} S. 5}
\newpage
In der Abbildung \ref{depend} erzeugt der Router 1 die zuvor beschriebene zusätzliche Netzwerkabhängigkeit und Netzwerkbelastung, da der Server 1 bei einem Ausfall des Routers nicht mehr durch den Überwachungsserver erreichbar ist und jede Überprüfung, die vom Überwachungsserver gesendet wird den Router mit dem Routing der Pakete belastet.

Deshalb gilt es laut \cite{Jose07} S. 5 folgende zwei Punkte beim Erstellen eines Überwachungssystems zu beachten:

\paragraph{Überwachungsredundanzen vermeiden}
Redundante Überwachung entsteht dadurch, dass der gleiche Service durch zwei Arten in unterschiedlicher Tiefe geprüft wird.
Ein einfaches Beispiel ist die Überwachung eines Webservers auf dem Standardport 80.
Eine Überwachungsmethode ist es diesen Port abzufragen und die entsprechende Rückantwort des Servers auszuwerten.
Als zweiter Test soll die auf dem Webserver laufende Webseite überwacht werden. 
Dafür kann die jeweilige Webseite über die Adresse nach einem bestimmten Inhalt untersucht werden.

In beiden Fällen wird getestet, ob der Webserver über das Netzwerk ansprechbar ist, jedoch sagt der zweite Test zusätzlich noch aus, dass die korrekte Webseite angezeigt wird, somit wäre der erste Test überflüssig.
Jedoch muss zuvor abgewogen werden, ob eine redundante Überwachung nicht sogar hilfreich bei der Ermittlung der Fehlerursache ist.
%Wenn im oberen Beispiel der Inhalt der überwachten Webseite verändert wird, so können beide Tests die Fehlerursache eingrenzen.
So können beide Tests die Fehlerursache eingrenzen.
Der erste Test überprüft, ob der Webserver erreichbar ist und der zweite Test kann erkennen, ob eine falsche bzw. veraltete Seite ausgeliefert wird.
%, dass der Webserver einwandfrei funktioniert.

\paragraph{Minimale Netzwerkbelastung}
Um bereits stark belastete Netzwerkpunkte zu entlasten, bietet es sich an, die Frequenz mit der die Test über das Netzwerk gesendet werden zu verringern.
Die Aufstellung des Überwachungsservers ist dadurch gerade bei größeren Serverlandschaften sehr wichtig, da durch eine effiziente Platzierung womögliche Flaschenhälse  in Form von veralteten Switches oder ähnlichem vermieden werden können.

\subsubsection{Sicherheitsaspekte}
Um erweiterte Statusinformationen über einen Prozess oder über die Arbeitsspeicherauslastung auszulesen, ist zusätzliche Software auf den Hosts nötig.
Diese Software benötigt oft einen zusätzlichen geöffneten Port auf dem zu überwachendem Rechner, die einen neuen Angriffspunkt für Angreifer darstellen kann.
Außerdem erhält der Überwachungsserver Ausführungsrechte auf dem Client, so dass eine weitere potentielle Sicherheitslücke in einem vermeintlich zuvor sicherem System entsteht.
Jeder, der die Kontrolle über den Überwachungsserver besitzt oder sich als solcher ausgibt, kontrolliert somit gleichzeitig alle anderen überwachten Hosts.

Um dies zu verhindern gibt es verschiedene Ansätze.
Als ersten Ansatz sollte der Port durch den der Überwachungsserver mit dem Host kommuniziert vom Standardwert abweichen, damit nicht sofort erkennbar ist, dass sich eine angreifbare Überwachungssoftware auf dem Rechner befindet.\label{changeport}
Damit die über diesen Port versendeten Informationen nicht für Dritte zugänglich sind, bietet es sich an die auszutauschende Informationen mit einem Algorithmus zu verschlüsseln.
Durch den Einsatz eines Verschlüsselungsalgorithmus werden die Informationen nicht mehr im Klartext ausgetauscht, sondern
Da die Möglichkeit einer Verschlüsselung der Datenübertragung nicht von jeder Überwachungssoftware angeboten wird, gilt diese Option als Auswahlkriterium in der späteren Umsetzung bzw. im produktivem Betrieb.

Des weiteren sollte die Erlaubnis der Abfrage der Überwachungsinformationen anhand der \gls{IP}-Adresse eingeschränkt werden, so dass der Client nur Anfragen des Überwachungsservers akzeptiert.
Durch diese Einschränkung kann vermieden werden, dass sensible Informationen aus den Antworten an unberechtigte Dritte übermittelt werden oder ein Denial of Service-Angriff (\gls{DoS}) durch eine übermäßig hohe Anzahl an Anfragen an den Client gesendet wird, um eine Überlastung des Servers zu erreichen und diesen somit arbeitsunfähig zu machen.

%\begin{itemize}
%\item Verschlüsselung der Informationen, die zwischen dem Server und dem Host hin- und hergesendet werden, damit man nicht die Inforamtionen im Klartext einfach auslesen kann.
%\item Firewall regeln, dazu Bild aus dem Jose07 Buch S9
%\end{itemize}
%\subsubsection{Port- versus Anwendungsüberwachung}
%\begin{itemize}
%\item E2E
%\end{itemize}






























\subsection{Dokumenten-Management-Systeme}

Um ein Dokumenten-Management-System (\gls{DMS})  zu erläutern muss sich zuerst mit dem Begriff des \textbf{"`Dokuments"'} auseinander gesetzt werden.
In \cite{DMS08} S. 2 wird ein Dokument durch folgende Punkte definiert:

\begin{itemize}
\item Ein Dokument fasst inhaltlich zusammengehörende Informationen strukturiert zusammen, die nicht ohne erheblichen Bedeutungsverlust weiter unterteilt werden können. 
\item Die Gesamtheit der Information ist für einen gewissen Zeitraum zu erhalten.
\item Dokumente dienen oft dem Nachweis von Tatsachen.
\item Ein Dokument ist als Einheit ablegbar (speicherbar) und/oder versendbar und/oder wahrnehmbar (sehen, hören, fühlen).
\item Das Dokument ist eigentlich der Träger, der die Informationen speichert, egal ob das Dokument ein Stück Paper, eine Datei auf einem Rechner, ein Videoband oder eine Tontafel etc. ist. Dies bedeutet auch, dass es keine Bindung an Papier oder ein geschriebenes Wort gibt.
\end{itemize}

Desweiteren gibt es eine Differenzierung in zwei Definitionen:

\begin{quote}"`Als \textbf{Dokument im konventionellen Sinne} werden Dokumente bezeichnet, die als körperliches Dokumente (z. B. Papier) vorliegen, ursprünglich als körperliches Dokument vorlagen oder für die Publizierung auf einem körperlichen Medium vorgesehen sind.

Die Begrifflichkeit des \textbf{Dokuments im weiteren Sinne} erweitert den Begriff des Dokuments um semantisch zusammengehörende Informationsbestände , die für die Publikation in nicht-körperlichen Medien, z. B. Webseiten, Radio, Fernsehen o. ä. vorgesehen sind. Derartige Dokumente werden oft dynamisch gestaltet und zusammengestellt."' \begin{flushright}\cite{DMS08} S. 2\end{flushright}\end{quote}

Unter \textbf{Dokumenten-Management} werden primär die Verwaltungsfunktionen Erfassung, Bearbeitung, Verwaltung und Speicherung von Dokumenten verstanden. \cite{DMS08} S. 344.

Darunter fallen laut \cite{DMS08} S. 3 folgende Punkte:

\begin{itemize}
\item Kennzeichnung und Beschreibung von Dokumenten (auch Metadaten des Dokuments genannt) 
\item Fortschreibung, Versionierung und Historienverwaltung von Dokumenten
\item Ablage und Archivierung von Dokumenten
\item Verteilung und Umlauf von Dokumenten
\item Suche nach Dokumenten bzw. Dokumenteninhalten
\item Schutz der Dokumente vor Verfälschung, Missbrauch und Vernichtung
\item Langfristiger Zugriff auf die Dokumente und Lesbarkeit der Dokumente
\item Lebenslauf und Vernichtung von Dokumenten
\item Regelung von Verantwortlichkeiten für Inhalt und Verwaltung von Dokumenten
\end{itemize}

Der Begriff \textbf{"`Dokumenten-Management-System"'} muss auch in zwei verschiedene Sichtweisen differenziert werden:
\begin{quote}"`Bei \textbf{Dokumenten-Management-Systemen im engeren Sinne} geht es um die Logik der Verwaltung von Dokumenten, deren Status, Struktur, Lebenzyklus und Inhalt. Dokumente werden beschrieben, klassifiziert und in einer bestimmten logischen Struktur eingeordnet, damit sie einfach wieder gefunden werden können. Dokumente entstehen, werden verändert und (irgendwann) vernichtet.

Den \textbf{Dokumenten-Management-Systemen im weiteren Sinne} ordnet man auch noch weitere Funktionalitäten zu, wie z. B. Schrifterkennung, automatische Indizierung, [...], Publizierung. Hier lassen sich die Grenzen nicht mehr genau bestimmten!"' \begin{flushright}\cite{DMS08} S. 5\end{flushright}\end{quote}

\subsection{Content-Management-Systeme}
Bei einem Content-Management-System (\gls{CMS})  steht nicht mehr das eigentliche Dokument im Vordergrund, sondern vielmehr der enthaltene Informationsgehalt des Dokuments.
Der Unterschied zwischen einem DMS und einem CMS besteht laut \cite{DMS08} S. 114 im/in Folgenden/m:

\begin{quote}"`Abgrenzend zum Dokumenten-Management handelt es sich beim Content-Management nicht vordergründig um die Verwaltung von Dokumenten, sondern um die Verwaltung von Informationseinheiten, die miteinander verknüpft sein können. [...] Je nach ausprägung kann nun ein konkretes System als Dokumenten-Management-System mit Content-Management-Funktionen definiert werden und umgekehrt. [...]
Der Ansatz des Content-Management unterscheidet sich vom "`klassischen"' Dokumenten-Mangement vor allem in Bezug auf die betrachteten Objekte: Ein DMS hat als kleinestes Objekt der Betrachtung eines einzelnen Dokument. [...] Content-Management ist auf logische Informationseinheiten ausgerichtet. Es ist z.B. das Ziel des Content-Managements, Inhalte, die auf mehrere Quellen verteilt sind, neue zusammenzustellen und daraus z.B. ein neues Dokument zu generieren."'
\begin{flushright}\cite{DMS08} S. 114f\end{flushright}\end{quote}

Die folgende Abbildung soll den (charakteristischen) Unterschied zwischen CMS-Systemen und DMS-Systemen verdeutlichen.

\begin{center}
Bild \cite{DMS08} S. 115
\end{center}

Wie zuvor beschrieben ist die Sichtweise eines DMS nur auf die einzelnen Dokumente beschränkt, während ein CMS einzelne Elemente / Informationen aus den Dokumenten extrahieren und daraus ggf. ein neues Dokument generieren kann. Die Sichtweise des CMS wird durch das gestrichelte Polygon dargestellt, welches hier dokumentenübergreifend abgebildet ist.

Der (theoretische/beabsichtigte) Zweck, weshalb ein CMS-System eingesetzt wird, ist laut Oracle folgendermaßen definiert:

\begin{quote}"`The key to a successful content management implementation is unlocking the value of content by making it as easy as possible for it to be consumed. This means that any piece of content must be available to any consumer, no matter what their method of access."'
\begin{flushright}\cite{UCM07} S. 12\end{flushright}\end{quote}

Ein CMS soll die Informationen jedes/jedwedem (Inhalts) extrahieren/aufnehmen und jedes Einzelteil / Element dieser Information den Benutzern zugänglich machen, unabhängig von der Art des Zugriffs.
Dieses Konzept soll in Abbildung \ref{ucm-a2a} verdeutlicht werden.

\begin{figure}[ht]
	\centering
	   \fbox{\includegraphics[width=0.95\textwidth]{bilder/ucm.png}}
		\caption["`any-to-any"' Content-Management Konzept]{"`any-to-any"' Content-Management Konzept\protect\footnote}
		\label{ucm-a2a}
\end{figure}
\footnotetext{Quelle: \cite{UCM07} S. 12}

Das CMS steht hier in der Mitte der Abbildung als Medium zwischen den verschiedenen Inhalten, eingestellt von den \textit{Contributors} (links), und den Anwendern, die auf transformierte Versionen der Inhalte durch unterschiedliche Arten zugreifen (rechts).


\subsection{Enterprise-Content-Management-Systeme - optional}
In diesem Zusammenhang / Kontext sei auch der Begriff Enterprise-Content-Management (\gls{ECM}) genannt.
Laut der "`Association for Information and Image Management"' (\gls{AIIM}\footnote{Die AIIM ist eine Gesellschaft von internationalen Herstellern und Anwendern von Informations- und Dokumenten-Mangement-Systemen}), welche sich mit umfasst dieser Begriff die Verwaltungfunktionen von Unternehmensinformationen in unterschiedlichen Dokumentformaten.\footnote{Quelle: \url{http://www.aiim.org/What-is-ECM-Enterprise-Content-Management.aspx}}
Diese Funktionen werden laut \cite{DMS08} S. 116 durch verschiedene "`Systeme wie Dokumenten-Management, Groupware, Workflow, Input- und Output-Management, (Web-)Content-management, Archivierung, Records-Management und andere"' bereitsgestellt.


\subsection{Universal-Content-Management-Systeme}



























\subsection{[cronjobs]}

%\subsection{[Farbraum]}
\subsection{[ds - ADS Benutzer]}
\subsection{[Metadaten allg]}
\subsection{[alse+-true+-]}
%\subsection{[evtl Oracle DB]}







 \newpage
\section{Nagios}

\subsection{Allgemein}
Nagios dient zum Überwachen von Hosts und deren Services in komplexen Infrastrukturen(Host und Services erklären?) und wurde von dem Amerikaner Ethan Galstad seit 1999\footnote{Quelle: \url{http://www.netsaint.org/changelog.php}} - damals unter der Vorgängerversion NetSaint - entwickelt und bis heute gepflegt.

Der Name Nagios ist ein Akronym für "´Nagios Ain't Gonna Insist On Sainthood"', dabei ist der Begriff Sainthood eine Anspielung auf die Vorgängerversion NetSaint.\footnote{Quelle: \cite{NagiosFAQ}}

Galstad gründete aufgrund der vielfältigen(ansturmmäßig) und positiven Resonanz am 9. November 2007 die "`Nagios Enterprises LLC"', welche Nagios als kommerzielle Dienstleistung anbietet.
Die Software selbst blieb weiterhin unter der freien Lizenz "`GNU General Public License version 2"'\footnote{Quelle: \url{http://www.gnu.org/licenses/old-licenses/gpl-2.0.txt}} verfügbar.
Diese erlaubt Einblick in den Programmcode und das Modifizieren der Anwendung nach eigenen Vorstellungen.

Nagios erfreut sich hoher Beliebtheit aufgrund der (bereits vorhandenen [macht kein Sinn hohe Beliebtheit aufgrund der großen community?]) großen Community, die Tipps, Ratschläge und auch eigene Nagios-Plugins kostenlos anbietet.
Außerdem können selbst mit geringen Programmierkenntnissen zusätzliche Skripte zur Überwachung geschrieben werden, wenn ein spezieller Anwendungsfall dies erfordert.

Nagios benötigt eine Unix-ähnliche Plattform und kann nicht unter Windows-Betriebssystemen betrieben werden.

\textit{3 mal mit Nagios angefangen ...}

\newpage
\subsection{Aufbau / Architektur}


Barth schreibt über Nagios: \begin{quote}"`Die große Stärke von Nagios - auch im Vergleich zu anderen Netzwerküberwachungstools - liegt in seinem modularen Aufbau: Der Nagios-Kern enthält keinen einzigen Test, stattdessen verwendet er für Service- und Host-Checks externe Programme, die als \textit{Plugins} bezeichnet werden."'
\begin{flushright}\cite{Barth08} S. 25\end{flushright}\end{quote} 

Dieser "`Kern"' beinhaltet das komplette Benachrichtigungssystem mit Kontaktadressen und Benachrichtigungsvorgaben (Zeit, Art, zusätzliche Kriterien), die Hosts- und Servicedefinitionen inklusive deren Gruppierungen und schließlich das Webinterface.
Siehe hierfür auch Abbildung \ref{nagios-proc}.

\begin{figure}[ht]
	\centering
	   \fbox{\includegraphics[width=0.9\textwidth]{bilder/nagios-proc.png}}
		\caption[Logische Nagios Struktur]{Logische Nagios Struktur\protect\footnote}
		\label{nagios-proc}
\end{figure}
\footnotetext{Quelle: \url{http://www.nagioswiki.org/w/images/8/81/Plugins.png}}

Damit Nagios die gewünschten Server überwachen kann, müssen sie der Anwendung zuerst bekannt gemacht werden.
Dies wird über das Anlegen einer Konfigurationsdatei mit einem Host-Objekt erreicht.
Dabei richtet sich die Definition des Host-Objektes nach dem Schema, welches für alle Objektedefinitionen (Services, Kontakt, Gruppen, Kommandos etc.) gilt:

\begin{lstlisting}[captionpos=b, caption=Nagiosschema für Objektdefinitionen, label=schmeaobj, breaklines = false]
define object-type {
	parameter value
	parameter value
	...
}
\end{lstlisting}


Eine gültige Host-Definition muss mindestens folgende Elemente besitzten:

\begin{lstlisting}[captionpos=b, caption=Definition eines Hostobjektes, label=hostobj, breaklines = true, language=sh]
define host{
	host_name		example.kit.edu #Referenzname des Servers
	alias			Oracle UCM Server #Weitere Bezeichnung
	address			example.kit.edu #FQDN des Rechners
	max_check_attempts	4 #Anzahl der Checks zum Wechsel von Hard- zu Soft-State
	check_period		24x7 #Zeitraum der aktiven Checks
	contact_groups		UCM-admins #Zu alarmierende Benutzergruppe
	notification_interval	120 #Minuten bis Alarmierung wiederholt wird
	notification_period	24x7 #Zeitraum der Benachrichtigungen
}
\end{lstlisting}

Gewöhnlich / In der Praxis werden öfters verwendete Attribute wie die Kontaktgruppe oder der Zeitraum für die aktiven Checks durch Verwendung eines übergeordneten Host-Objektes nach unten vererbt.
Dadurch müssen nurnoch die spezifischen Inforamtionen und der Name des übergeordneten Host-Objektes eingetragen werden.
\begin{lstlisting}[captionpos=b, caption=Verkürzte Definition eines Hostobjektes, label=vhostobj, breaklines = true, language=sh]
define host {
        use             windows-server #Oberklasse dieses Host-Objektes
        host_name       example.kit.edu
        alias           Oracle UCM Server
        address         example.kit.edu
}
\end{lstlisting}

Mit dieser Hostdefinition wird der Rechner im Webinterface von Nagios bereits angezeigt:

\begin{figure}[ht]
	\centering
	   \fbox{\includegraphics[width=0.95\textwidth]{bilder/example-ping.png}}
		\caption{Anzeige des Servers im Webinterface von Nagios}
		\label{check-swap}
\end{figure}

Jedoch wird nur die Erreichbarkeit über das Netzwerk mit einem Ping überwacht.
%Umfangreichere / Komplexere / Weitergehende / 
Um andere Dienste zu überwachen müssen die gewünschten Plugins explizit aus dem Nagios Repertoire dem zu überwachendem Computer mit einem ähnlichen Schema zugeteilt werden.

Eine beispielhafte Servicedefinition für die Überwachung des Webservers auf dem Host \textit{example.kit.edu} wird in Codelisting \ref{servdef} gezeigt.

\begin{lstlisting}[captionpos=b, caption=Verkürzte Definition eines Hostobjektes, label=servdef, breaklines = true, language=sh]
define service{
        use                     generic-service #Oberklasse dieses Service-Objektes
        host_name               example.kit.edu
        service_description     HTTP Server #Bezeichnung des Checks
        check_command           check_http #Angabe des NagiosPlugins (hier ohne Parameter)
        }

\end{lstlisting}

Die Plugins werden durch die Servicedefinitionen mit den jeweiligen Hosts verbunden und durch das Attribut \textit{check\_command} mit ggf. veränderten Parametern durch Nagios aufgerufen.

Nagios wird in einem festlegbarem / veränderbarem Zeitintervall alle vom Benutzer definierten Host- und Servicechecks überprüfen und die Ergebnisse der entsprechenden Plugins verarbeiten / auswerten.

Weiterhin beschreibt Barth die Plugins folgendermaßen:
\begin{quote}"`Jedes Plugin, das bei Host- und Service-Checks zum Einsatz kommt, ist ein eigenes, selbständiges Programm, das sich auch unabhängig von Nagios benutzen lässt."' \begin{flushright}\cite{Barth08} S. 105\end{flushright}\end{quote} 

Daher lassen sich die Parameter eines Plugins folgendermaßen überprüfen:

\begin{figure}[ht]
	\centering
	   \fbox{\includegraphics[width=0.85\textwidth]{bilder/check-swap-white.png}}
		\caption{Beispielhafte manuelle Ausführung eines Servicechecks}
		\label{check-swap}
\end{figure}

Die Ausgabe des Plugins gibt den Zustand des Services an; in diesem Fall wird kein Schwellwert überschritten, daher die Meldung \pictext{SWAP OK}.
Dieses Plugin liefert noch zusätzliche Performance-Informationen, die mit externen Programmen ausgewertet, gespeichert und visualisiert werden können.
Standardmäßig werden die Performancedaten von der normalen Ausgabe mit einem \pictext{|} getrennt.
Jedoch können auch Werte aus der normalen Textausgabe verwendet werden, so dass in diesem Beispiel keine Berechnung des Prozentsatzes notwendig wäre.

Um diesen Service mit den angegebenen Schwellwerten von Nagios überwachenzulassen, muss folgende Servicedefinition in die Konfigurationsdatei eingetragen werden:

\begin{lstlisting}[captionpos=b, caption=Beispielhafte (Definition) eines Servicechecks, label=servdef, breaklines = true, language=bash]
# Define a service to check the swap disk space
# on the local machine.  Warning if =< 20% free, 
# critical if =< 10% free space on swap partition.

define service{
 use         generic-service 
 host_name   example.kit.edu 	
 service_description  Swap Disk Space
 check_command   check_swap!-w 20% -c 10%	# Angabe des zu verwendenden Plugins mit WARNING (respektiv) CRITICAL Schwellwertparameter
}
\end{lstlisting}



Dabei wird in vier verschiedene Rückgabewerte / Antworten der Plugins unterschieden:

%\setlength{\tabcolsep}{50pt}
\begin{table}[!htbp]
\centering

\begin{tabular}{l l p{7cm}}
%\hline
\textbf{Status} \hspace{2 mm} & \textbf{Bezeichnung} \hspace{3 mm} & \textbf{Beschreibung}\\
\hline
%\textit{features} & complete\tnote{1} & complete\tnote{1} \\
%\hline
0 & OK & Alles in Ordnung \\
%\hline
1 & WARNING & Die Warnschwelle wurde überschritten, die kritische Schwelle ist aber noch nicht erreicht.\\
%\hline
2 & CRITICAL & Entweder wurde die kritische Schwelle überschritten oder das Plugin hat den Test nach einem Timeout  abgebrochen. \\
%\hline
3 & UNKNOWN &  Innerhalb des Plugins trat ein Fehler auf (zum Beispiel weil falsche Parameter verwendet wurden)\\
%\hline
\bottomrule
\end{tabular}
\caption[Rückgabewerte für Nagios Plugins]{Rückgabewerte für Nagios Plugins\protect\footnote} %\protect\footnote
%\end{twoparttable}
\end{table}
\footnotetext{Quelle: \cite{Barth08} S. 105f}

Anhand dieser Werte wertet Nagios gezielt den Status des jeweiligen Objektes (Host oder Service) aus.
Weiterhin gibt es weiche (Soft States) und harte Zustände (Hard States):

\begin{figure}[ht]
	\centering
	   \fbox{\includegraphics[width=0.85\textwidth]{bilder/hs-states.png}}
		\caption[Beispiel für den zeitlichen Verlauf durch vers. Zustände]{Beispiel für den zeitlichen Verlauf durch vers. Zustände\protect\footnote}
		\label{hs-states}
\end{figure}
\footnotetext{Quelle: \cite{Barth08} S. 95}
Ausgehend von einem OK Zustand wird in diesem Beispiel jede fünf Minuten periodisch überprüft, ob sich der Status des überwachten Objektes verändert hat.                                                                                                                                                                                         Nach zehn Minuten wird ein Umschwenken / Änderung des Zustandes durch das jeweilige Plugin gemeldet.
Hier im Beispiel wechselt der Zustand nach CRITICAL, zunächst allerdings als Soft State
Daher wird durch Nagios noch keine Benachrichtigung versendet, da es sich um eine Falschmeldung, auch False Positive genannt, handeln kann.
Aufgrund einer kurzweiligen / kurzfristigen (besseres Wort? peak mäßig) hohen Auslastung des Netzwerkes oder um ein kurzzeitiges Problem, welches sich von alleine wieder normalisiert. (Bspw. Prozessorauslastung)

Um einen False Positive auszuschliessen, wird der im Soft State befindliche Service bzw. Host mit einer höheren Frequenz überprüft.
Sollten diese Überprüfungen den vorherigen Zustand bestätigen, verfestigt sich der aktuelle Zustand, man spricht nun von einem Hard State / wechselt der Zustand in den Hard State.
Erst in diesem Moment werden die entsprechenden Kontaktpersonen über den in diesem Beispiel kritischen Zustand benachrichtigt.
Sollte sich der Zustand wieder in den Normalzustand begeben und dieser Zustandsübergang wird von dem (von Nagios ausgeführten) Plugin festgestellt, wird dies an den Nagios Server gemeldet.

Ein Übergang zu dem OK Status wird sofort als Hard State festgelegt / festgehalten / festgesetzt / und führt dadurch zur sofortigen Benachrichtigung durch Nagios.

\begin{itemize}
\item Betroffene OSI Schichten auflisten und erklären
\item Wie werden die Info von Nagios gesammelt und wie gespeichert -> FlapDetection
\item FLapping \footnote{\url{http://nagios.sourceforge.net/docs/2\_0/flapping.html}}
\item Benachrichtigung durch email oder sms sogar per Telefon geht usw.
\item Hierarchie \url{http://nagios.sourceforge.net/docs/3_0/networkreachability.html}
\end{itemize}

\subsection{Überprüfungsmethoden}
%Service Checks und deren / ihre Realisierung / Ausführung / (
Dienste, die im Netzwerk zur Verfügung stehen (Netzwerkdienste), wie ein Web- oder \gls{FTP}-Server , lassen sich einfach / simpel direkt über das Netz auf ihren Zustand (hin) überprüfen /  testen.
Hierfür muss dem entsprechende Plugin lediglich die Netzwerkadresse mitgeteilt werden, siehe Abbildung \ref{check-http} als beispielhafte Überprüfung eines Webservers.

\begin{figure}[ht]  
	\centering
	   \fbox{\includegraphics[width=0.85\textwidth]{bilder/check-http.png}}
		\caption{Beispielhafte manuelle Ausführung eines netzwerkbasierenden Servicechecks / HTTP Server Check}
		\label{check-http}
\end{figure}

(Bitte beachten, dass das Plugin immernoch auf dem Nagios Server ausgeführt wird / sich immernoch auf dem Nagios Server befindet)

Dienste, die sich nicht standardmäßig / ohne weiteres / ohne weitere Anpassung(en) über das Netzwerk überprüfen lassen, wie die Kapazität einer Festplatte auf einem entfernten Server(, das (Laufen) eines Prozesses) oder die Durchsuchung einer Logdatei nach bestimmten (Stop)wörtern.

Nagios bietet verschiedene Möglichkeiten an solche Dienste zu überprüfen:

\begin{figure}[ht]
	\centering
	   \fbox{\includegraphics[width=0.65\textwidth]{bilder/nagios-kern.png}}
		\caption[Verschiedene Überwachungsmöglichkeiten von Nagios]{Verschiedene Überwachungsmöglichkeiten von Nagios\protect\footnote}
		\label{nagios-kern}
\end{figure} 
\footnotetext{Quelle: \cite{Barth08} S. 98}

\paragraph{Methode 1 - Netzwerkdienste}
Der zuvor, in Abbildung \ref{check-http}, gezeigte Test eines netzwerkbasierenden Dienstes wird im obigen Bild mit dem Client-Rechner (mit der Nummer) 1 abgebildet.
%Die Überprüfung von nicht netzwerkbasierenden Diensten soll mit den restlichen Client-Rechnern aufgezeigt werden.
Dies ist die einfachste Überwachungsmethode, da keine zusätzlichen Programme oder aufwändige Konfiguration benötigt wird.
Vorteilhaft ist auch, dass der Dienst über das Netzwerk getestet wird, so wie der Benutzer auch auf den Dienst zugreift.
Damit können auch gleichzeitig andere Knotenpunkte wie Switches überwacht werden.

\paragraph{Methode 2 - SSH}

Falls es sich beim Client um ein Unixderivat handelt, ist der entfernte Zugriff auf diesen Client per SSH\footnote{Durch eine Secure Shell (\gls{SSH}) kann man sich eine verschlüsselte Netzwerkverbindung zum entfernten Rechner aufbauen.}-Dienst möglich.
Dazu muss auf dem Client ein \gls{SSH}-Benutzerkonto angelegt sein, mit dem sich Nagios anmelden kann und die öffentlichen Schlüssel (zwischen Nagios Server und Client) ausgetauscht werden, damit keine passwortabhängige Benutzerauthentifizierung (Eingabe von PW) notwendig ist.
Danach können lokale Ressourcen, wie Festplattenkapazität oder Logdateien mit dem entsprechenden Plugin direkt auf dem entfernten Rechner überwacht werden.
Damit der Client diese Plugins verwenden kann, müssen sich die gewünschten Plugins (auch) auf dem Client (lokal) befinden.
Eine beispielhafte Verwendung mit dem dafür gedachten Nagios Plugin \pictext{check\_by\_ssh} (von dieser Überwachungsmethode) wird in Abbildung \ref{check-ssh} gezeigt.

\begin{figure}[ht]
	\centering
	   \fbox{\includegraphics[width=0.85\textwidth]{bilder/check_by_ssh.png}}
		\caption{Beispielhafte manuelle Ausführung eines Servicechecks über SSH}
		\label{check-ssh}
\end{figure}

(Hier beachten, dass kein Passwort abgefragt wird, daher zuvor Schlüsselaustauschen)

\paragraph{Methode 3 - NRPE}

Eine alternative Möglichkeit solche Dienste auf entfernten Rechnern zu überwachen, ist durch den sogenannten Nagios Remote Plugin Executor (\gls{NRPE}).
Hier muss auf dem Client ein "`Agent"' installiert werden, welcher einen Port öffnet mit dem der Agent mit dem Nagios Server kommuniziert.

\begin{figure}[ht]
	\centering
	   \fbox{\includegraphics[width=0.9\textwidth]{bilder/nrpe.png}}
		\caption[Aktive Checks mit NRPE]{Aktive Checks mit NRPE\protect\footnote}
		\label{aktivchecks}
\end{figure}
\footnotetext{Quelle: \url{http://www.nagios.org/images/addons/nrpe/nrpe.png}}

Der Nagios Server kann dann Anforderungen über das Nagios-Plugin \pictext{check\_nrpe} an den Client verschicken.
Ein Aufruf dieses Plugins ist dem des \pictext{check\_by\_ssh} Plugins, siehe dazu Abbildung \ref{check-ssh}, sehr ähnlich.

Der Nachteil dieser Variante ist ein zusätzlich geöffneter Port und der höhere / erhöhte Aufwand beim Installieren des Agenten im Gegensatz zum (vermutlich /meistens) bereits laufendem \gls{SSH}-Dienst.
Zusätzlich gibt es nur die Möglichkeit die Anfragen auf diesem Port auf bestimmte IPs zu beschränken, jedoch nicht den Zugriff durch ein Passwort zu sichern.
Dafür beschränkt sich der \gls{NRPE} (lediglich) auf die auf dem entfernten Client liegenden Nagios Plugins und kann nicht System- bzw. Benutzerkommandos aufrufen, wie bspw. das \pictext{rm} Kommando zum Löschen von Dateien, welche durch den Einsatz von \pictext{check\_by\_ssh} standardmäßig möglich wären.
Sicherheitstechnisch gesehen ist die SSH-Variante kritischer, da es einem Angreifer ermöglicht auf diese System- bzw. Benutzerkommandos zuzugreifen, wenn er die Kontrolle über den Nagios Server erlangt.
Beide Verfahren unterstützen die Verschlüsselung der Datenübertragung zwischen Nagios-Server und Client, so dass keine Informationen im Klartext übertragen werden.

\paragraph{Methode 4 - SNMP}

%nur grob angerissen / kurz / 
Diese Variante wird nur verkürzt behandelt, da sich diese Arbeit hauptsächlich mit der Überwachung von Servern beschäftigt und nicht von Netzwerkkomponenten wie Switches oder Router, die nur durch das Simple Network Management Protocol (\gls{SNMP}) überwacht werden können, wenn mehr Informationen als eine schlichte Erreichbarkeit überprüft / gesammelt werden soll.

Barth schreibt über diese Variante / Überwachungsmethode:
\begin{quote}"`Mit dem Simple Network Management Protocol \gls{SNMP} lassen sich ebenfalls lokale Ressourcen übers Netz abfragen [...]. Ist auf dem Zielhost ein \gls{SNMP}-Daemon installiert [...] kann Nagios ihn nutzen, um lokale Ressourcen wie Prozesse, Festplatten oder Interface-Auslastung abzufragen."' \begin{flushright}\cite{Barth08} S. 101\end{flushright}\end{quote} 

Durch \gls{SNMP} kann auf die strukturierte Datenhaltung der \gls{MIB}\footnote{Die Management Information Base (\gls{MIB}) dient als \gls{SNMP}-Informationstruktur und besteht aus einem hierarchischen, aus Zahlen aufgebauten Namensraum. Ähnliche Struktur wie andere hierarchische Verzeichnisdiensten wie \gls{DNS} oder \gls{LDAP}. Quelle: \cite{Barth08} S.233} in den entfernten Netzwerkknoten zugegriffen werden.
%###########################
Die \gls{MIB}-Struktur ist folgendermaßen aufgebaut:

\begin{figure}[ht]
	\centering
	   \fbox{\includegraphics[width=0.95\textwidth]{bilder/mib.png}}
		\caption[Struktur der Management Information Base]{Struktur der Management Information Base\protect\footnotemark}
		\label{munin-mib}
\end{figure}
\footnotetext{Quelle: \cite{Mu08} S. 156}
Anhand dieser Anordnung können die \gls{SNMP}-Plugins von Nagios den gewünschten Wert über das Netzwerk abfragen.
Bei einem Switch werden die auslesbaren Informationen vom Hersteller bestimmt.
Wenn auf einem Rechner eigene Ergebnisse in der \gls{MIB} abgespeichert werden sollen, muss dies durch einen \gls{SNMP}-Daemon eingetragen werden.
Dessen Konfiguration ist im Vergleich zu den anderen Überwachungsmethoden deutlich komplexer.

Es gibt zwei verschiendene Möglichkeiten Dienste mit \gls{SNMP} zu überwachen.
Der Server frägt aktiv den Inhalt der entsprechenden MIB Einträgen periodisch ab oder der Client sendet asychron seine Statusmeldungen an den Nagios-Server.
Beim letzteren spricht man auch von so genannten \gls{SNMP}-Traps.


\begin{itemize}
\item Warum wird SNMP nicht verwendet?
\item klartextübertragung bis SNMP 2c
\item Schreibrechte können Schaden anrichten
\item Brute Force attacken ausgesetzt
\item Beschränkte Ausgabemögliochkeit / maximale Datengrösse der Ausgabe -> Logüberwachung nur mit Aufwand möglich autarkes Programm von Nöten, dann muss selbst das Ergebniss in die MIB geschrieben werden, damit Nagios darauf Zugriff erlangt -> zu aufwändig im Vergleich mit Agenten
\end{itemize}

%Einen beispielhaften Zugriff auf SNMP-fähige Geräte wird in Abbildung \ref{munin-snmp} gezeigt.

%\begin{figure}[ht]
%	\centering
%	   \fbox{\includegraphics[width=0.5\textwidth]{bilder/snmp.png}}
%		\caption[Beispielhafter Zugriff auf SNMP-fähige Geräte]{Beispielhafter Zugriff auf SNMP-fähige Geräte\protect\footnotemark}
%		\label{munin-snmp}
%\end{figure}
%\footnotetext{Quelle: \cite{Mu08} S. 156}

%##################

Weiterhin unterscheidet man generell zwischen aktiven und passiven Checks.
\paragraph{Methode 5 - NSCA}
%asynchron!
Bei passiven Tests führt der zu überwachende Computer das statuserzeugende Plugin selbst aus und sendet es über ein weiteres Plugin zum Nagios-Server.
Hierfür muss das Testprogramm bzw. Script und das entsprechende Plugin \pictext{send\_ncsa}, welches zum Versenden der Informationen zuständig ist, auf dem Host vorhanden sein.
Auf der anderen Seite muss der \pictext{\gls{NSCA}} (Nagios Service Check Acceptor) auf dem Nagios-Server als Daemon gestartet sein, damit die übermittelten Ergebnisse von Nagios entgegengenommen werden.

Folgende Abbildung soll das Prinzip der passiven Checks verdeutlichen:
%Das Prinzip der passiven Checks lässt sich durch folgende vereinfachte Abbildung
\begin{figure}[ht]
	\centering
	   \fbox{\includegraphics[width=0.9\textwidth]{bilder/nsca.png}}
		\caption[Passive Checks mit NSCA]{Passive Checks mit NSCA\protect\footnote}
		\label{passivchecks}
\end{figure}
\footnotetext{Quelle: \url{http://www.nagios.org/images/addons/nsca/nsca.png}}

Das Testprogramm \textit{Remote Application} wird selbständig vom zu überwachenden Rechner \textit{Remote Host} aufgerufen und übermittelt durch das \pictext{send\_ncsa} Plugin die Ergebnisse über das Netzwerk an den Nagios-Server \textit{Monitoring Host}.
Da auf diesem der NSCA als Daemon läuft können die Ergebnisse an die Nagios-Anwendung zur Auswertung weitergegeben werden.

\begin{itemize}
\item Kurz agenten, zeigen auf f. Kapitel -> SNMP erklären (MIB, OID) Sicherheitsrisiko
\end{itemize}

\subsection{Überwachungslogik (mit Alarmierung/Benachrichtigung)}
\label{dismoni}
\begin{center}
TODO: Distributed Monitoring bezug auf allg überwachungssysteme
\end{center}


\subsection{Plugins}
Gedacht für Linux umgebung

Verschiedene Möglichkeiten Checks zu realisieren unter Unix Systemen:

Leicht programmierbar -> perl 
Extra Plugins für Windows

\subsection{(Windows) Agenten oder allgemein Einholen von Infos}
Warum nicht einfach alles über SNMP? -> ODI muss man erst beantragen, hoher Aufwand und dann doch nicht so universell/alles abdeckend wie aktive checks, man kann keine logfiles durchuchen -> könnte es aber als standalone prog auf dem client laufen lassen und dieser sendet dann passive checks

Sagen das auf alten NSClient verzichtet wird und OpMon Agent nicht behandelt
\begin{enumerate}
\item Bilder ausm Nagios Buch Seite 472ff!
\item NSClient++
\item NC\_Net
\item NRPE\_NT
\end{enumerate}

Zusammenfassung?


Welche wird jetzt eingesetzt und warum?

Erwähne sichheitsstechnisch Parameter erlauben oder nicht erlauben

Dabei sagen, dass wenn nicht erlaubt sind keine zentrale Konfiguration der Checks auf dem Nagios server möglich ist -> abwägen

\subsection{Visualisierung der eingesammelten Daten}

 \newpage
\section{Oracle UCM}

\subsection{Allgemein}
Oracle Universal Content Management basiert auf der Software Stellent von der gleichnamigen Firma Stellent, welche im November 2006\footnote{Quelle: \cite{OraPress}} von Oracle gekauft wurde.

\subsection{Aufbau}

Die Architektur des \gls{OracleUCM}-Systems gliedert sich in separate Komponenten auf wie in Abbildung \ref{ucm-arch} gezeigt wird.

\begin{figure}[ht]
	\centering
	   \fbox{\includegraphics[width=0.5\textwidth]{bilder/basic_architecture.png}}
		\caption[Oracle UCM Architektur]{Oracle UCM Architektur\protect\footnote}
		\label{ucm-arch}
\end{figure}
\footnotetext{Quelle: \cite{ClubOra}}

Die Anwendung \gls{OracleUCM} ist aus folgenden Kernkomponenten aufgebaut:

\paragraph{Content Server}

Der Content Server ist das Herzstück der Oracle UCM Anwendung und basiert auf einer Java-Anwendung.
Er dient als Grundgerüst (Framework) für darüber liegende Funktionen, da er für die Ablage der Dokumente sowie deren Verwaltung, siehe Abbildung \ref{lifecircle}, verantwortlich ist.

\begin{figure}[ht]
	\centering
	   \fbox{\includegraphics[width=0.85\textwidth]{bilder/contenserver.png}}
	  % \fbox{Quelle: \cite{Huff06} S. 17}
		\caption[Beispielhafter Einsatz eines Content Servers]{Beispielhafter Einsatz eines Content Servers\protect\footnote}
		\label{ucm-cs}
\end{figure}
\footnotetext{Quelle: \cite{Huff06} S. 17}

Dieses Framework ist als Service-Oriented Architecture (\gls{SOA}) aufgebaut.
Im Kontext des Content Servers wird als Service ein diskreter Aufruf einer Funktion verstanden.
Dabei kann diese Funktion das Hinzufügen, die Bearbeitung, die Konvertierung oder das Herunterladen eines Dokumentes bedeuten.
Diese Services und ihre einzelnen Funktionen werden durch das \gls{SOA}-Framework verdeckt und stehen als Web Service zur Verfügung.

\paragraph{Inbound Refinery}
Die Inbound Refinery ist für die Konvertierung der Dokumente zuständig und ist keine interne Komponente des Content Servers, sondern kann sich auch auf einem anderen Server befinden.
Dabei werden spezielle Add-ons (Filter) für die Konvertierung verwendet.
In zeitlichen Abständen überprüft die Inbound Refinery, ob die bisher eingecheckten Dokumente konvertiert werden müssen, und speichert die konvertierte Datei in den Web Layout-Ordner.


\paragraph{Data Storage} Der Content Server verwaltet die Datenbank, die die Metadaten über die Dokumente beinhaltet.
Diese Metadaten werden für die Versionierung, Verwaltung und Suchanfragen verwendet.

\paragraph{Content Storage} Der Content Storage liegt auf dem Dateisystem und ist in Vault und Web Layout aufgeteilt.

\paragraph{Vault und Web Layout}
Der \textbf{Vault} ist ein Ordner auf dem Server in dem die Originaldateien der Benutzer in ihrem nativen Format gespeichert werden.
Im Gegensatz dazu werden im \textbf{Web Layout} die konvertierten Versionen der Dokumente abgelegt. Beispielsweise eine \gls{PDF}-Version einer Microsoft Word-Datei.

\paragraph{Search Engine}
Eine Suchanfrage eines Benutzers wird zuerst an den Webserver gesendet, der die Anfrage an den Content Server weitergibt.
Der Content Server verwendet anschließend seine Search Engine um ein Suchergebnis zu erhalten.
Das Suchergebnis wird dem Webserver übergeben, der das Ergebnis an den Benutzer sendet.
Die Search Engine verwendet einen Suchindex, der aus den Metadaten und Referenzen zu den Volltextversionen der Dokumente besteht.

\paragraph{Webserver}
Der Webserver ist hauptsächlich für die Präsentation und Ausgabe der gespeicherten Dokumenten und Informationen zuständig.
Dabei ist er auch für die Authentifizierung der Benutzer zuständig.


\subsection{Konkrete Verwendung}

\gls{OracleUCM} wird als Enterprise-Content-Management für die Verwaltung von Webseiten, Dokumenten und Bilder im Forschungszentrum Karlsruhe eingesetzt.

Dabei wird im konkreten Anwendungsfall \gls{OracleUCM} als Bilddatenbank verwendet.
Diese Bilddatenbank nimmt Fotos und Bilder der Benutzer entgegen (\textit{Einchecken}) und konvertiert das Originalbild dabei in andere Bildversionen wie beispielsweise eine verkleinerte Version für Webseiten.

Dieser typische Ablauf soll durch Abbildung \ref{bdbanw} verdeutlicht werden.
\begin{figure}[ht]
	\centering
	   \fbox{\includegraphics[width=0.9\textwidth]{bilder/bdb.png}}
		\caption{Bilddatenbank als Anwendung}
		\label{bdbanw}
\end{figure}

Die untere Tabelle zeigt die verschiedenen Bildversionen einer bereits konvertierten Bilddatei.\\

Da die Bilddatenbank unter einem Windows-Server betrieben werden soll, muss dies für die Überwachung bei der Auswahl der Überwachungselemente und Realisierung der Überwachung durch Nagios berücksichtigt werden.

\newpage
\section{Überwachungselemente}
\label{elemente}
Die Überwachung eines Dienstes über ein Netzwerk verteilt sich auf verschiedenen Ebenen mit unterschiedlichen Gewichtungen.
Zum Beispiel stellt das simple Senden eines Pings an den entsprechenden Server die niedrigste und primitivste Stufe dar, da hier lediglich die Netzwerkschnittstelle des Servers auf ihre Funktionalität und dabei der Status der Netzwerkstrecke getestet wird.
Ob die Anwendung überhaupt auf dem Server läuft und in welchem Zustand sie sich befindet, muss auf eine andere Weise herausgefunden werden.

Dabei lassen sich aus den verschiedenen Überwachungselemente vier Kategorien Statusabfragen, Funktionalitätstests, Auswertung von Logdateien und Benutzersimulation bilden.

\subsection{Statusabfragen}
\label{syschecks}
Diese Kategorie besteht aus einfacheren Überprüfungen, die jeweils den Status des Überwachungselementes überwachen.
Dabei können weitere Untergruppen gebildet werden:

\paragraph{System}
\begin{itemize}
\item \textbf{Ping} Überprüft, ob der Rechner vom Nagios-Server über das Netzwerk erreichbar ist.
\item \textbf{Prozessorauslastung} Überwacht die Auslastung des Prozessors und schlägt bei ungewöhnlich hohen Werten Alarm.
\item \textbf{Festplattenspeicherausnutzung} Überwacht die Speicherplatzauslastungen der verschiedenen Festplattenpartitionen, damit immer genügend Speicherplatz für Anwendungen und Betriebssystem verfügbar ist.
\item \textbf{Arbeitsspeicherauslastung} Beobachtet wie viel Arbeitsspeicher vom System verwendet wird und wie viel davon noch zur Verfügung steht.
\end{itemize}

\paragraph{Prozesse}
\begin{itemize}
\item \textbf{IdcServerNT.exe} Der Prozess der \gls{OracleUCM}-Awendung.
\item \textbf{IdcAdminNT.exe} Der Prozess für das Administration-Webinterface von \gls{OracleUCM}.
\item \textbf{w3wp.exe} Der Prozess des Webservers Microsoft "`Internet Information Services"'
\end{itemize}

\paragraph{Services}
\begin{itemize}
\item \textbf{IdcContentService}  Den Zustand des \pictext{sccdms01}-Dienst überprüfen.
\item \textbf{IdcAdminService}  Den Zustand des \pictext{sccdms01\_admin}-Dienst für die Administration überprüfen.
\item \textbf{Zeitsynchronisationsdienst} Überprüfen, ob der \pictext{W32TIME}-Dienst, der für den Zeitabgleich mit einem Zeitserver zuständig ist, läuft und die Abweichung zwischen Client und Zeitserver festhalten.
\item \textbf{Antivirusdienst} Den Zustand des Dienstes \pictext{Symantec AntiVirus} überprüfen, der für die Updates des Virusscanners notwendig ist.
\end{itemize}

\subsection{Überwachung der Funktionalität}
\label{funztest}
Durch die vorherigen Tests kann herausgefunden werden, ob eine Anwendung oder ein Dienst auf dem Server gestartet wurde.
Die Funktionalität kann durch solche Überprüfungen jedoch \textbf{nicht} sichergestellt werden.
Da beispielsweise der Prozess bzw. Dienst des Webservers gestartet ist, jedoch keine Webseite aufgerufen werden kann.
%Da beispielsweise der Webserver aufgrund eines kritischen Fehlers nicht erreichbar ist, der Prozess bzw. Dienst dennoch läuft.
Daher muss eine weitere Art von Überprüfungen die Anwendungen auf ihre Funktionalität testen.

\begin{itemize}
\item \textbf{Webserver} Aufruf einer Webseite auf dem Server. Wenn auf diese Anfrage eine gültige Antwort in Form einer Statuscode-Meldung erfolgt, kann die korrekte Funktion des Webservers festgestellt werden.

\item \textbf{Webinterface des Oracle UCM} Zusätzlich wird mit dieser Abfrage die Integration des Content-Management-Systems in den Webserver überwacht, da hier nicht nur der Webserver, sondern eine \gls{OracleUCM} spezifische Webseite abgefragt wird.

\item \textbf{Benutzeranmeldung am Oracle UCM} Hier wird getestet, ob sich ein Benutzer erfolgreich am System anmelden kann.
Dies wird mit Anmeldungsdaten eines lokalen Benutzers und eines Active Directory-Benutzers durchgeführt um gleichzeitig die Verbindung zum Active Directory-Server zu testen.

\item \textbf{Oracle Datenbank} Überprüft den Verbindungsaufbau zur Datenbank. Wenn keine Verbindung zur Oracle-Datenbank möglich ist, können keine neuen Informationen gespeichert werden. 

%\item \textbf{Status von Cronjobs} In periodischen Zeitabständen werden Programme aufgerufen, deren Aufruf und Endstatus/Endergebniss überwacht werden muss. 
%Damit nicht das vorherige Ergebnis zu einem False Negative führt, müssen hier zusätzliche Zeitinformationen/zeitliche Parameter beachtet/bedacht werden.

\item \textbf{Anzahl Datenbankverbindungen} Anzahl der Verbindungen zur Datenbank, da aus Performanzgründen eine Obergrenze mit einer maximalen Anzahl festgelegt ist.
\end{itemize}

\subsection{Auswerten von Logdateien}
\label{checklog}
%Die zwei bisherigen Kategorien beinhalten simple Zustandsüberprüfungen oder aktive Funktionaltests.
In dieser Kategorie werden zusätzlich verschiedene Logdateien auf spezielle Warnungs- und Fehlermeldungen anhand eindeutigen Stopwörtern untersucht.
Dies ist notwendig um reaktiv Fehlverhalten der Anwendung zu erkennen, das nicht mit den vorherigen Überwachungselementen entdeckt wurde.
Des weiteren können durch die Analyse der Logdateien etwaige Alarmmeldungen der bisherigen Tests bestätigt, begründet oder aufgehoben werden.
Somit bietet das Auswerten der Logdateien zusätzliche Sicherheit False Positive- oder False Negative-Meldungen auszuschließen.

Die Oracle UCM Anwendung erstellt drei verschiedene Arten von Logdateien:\footnote{Quelle: \cite{UCMlog09}}

\begin{itemize}
\item \textbf{Content Server Log} 
\item \textbf{Inbound Refinery Log}
\item \textbf{Archiver Log}
\end{itemize}

Um alle Logs ohne Probleme im Internetbrowser anzuzeigen, liegen alle Logdateien im HTML-Format vor.
Alle drei Arten von Logs bestehen jeweils aus 30 verschiedenen Dateien, die sich täglich abwechseln.
Dadurch wird für jeden Tag im Monat eine separate Datei verwendet, um bei vielen Warnungs- und Fehlermeldungen durch die chronologische Anordnung den Überblick zu behalten.
Dabei werden die Logdateien zwangsweise nach 30 Tagen nacheinander überschrieben.

Diese Rotation der Logdateien muss bei der Durchsuchung nach Stopwörter beachtet werden, damit stets die aktuelle Logdatei überwacht wird und keine veralteten Informationen für False Positive-Meldungen durch Nagios sorgen.

%wieso nur plugin check\_log und nicht einfahc umfangreicheres Standalone Programmw ie syslog\_nd oder 8pussy

\subsection{Benutzersimulation}

Ein Dokument nimmt in seinem Lebenszyklus, siehe Abbildung \ref{lifecircle}, verschiedene Zustände an.
So kommt es nach dem Einchecken in den Zustand \pictext{genwww}.
Der darauf folgende Zustand \pictext{fertig} gibt die erfolgreiche Konvertierung des Dokumentes bekannt.
Die folgende Indizierung wird durch den Zustand \pictext{freigegeben} angezeigt.

Die Benutzersimulation hat die Aufgabe alle Schritte der Zustandsänderung durch typische Abfragen zu überprüfen.

\begin{itemize}

\item \textbf{Einchecken von Dokumenten} Damit die eigentliche Aufgabe des Dokumentenverwaltungssystem überwacht werden kann, werden verschiedene Datenformate testweise eingecheckt. 
Dabei wird die Antwort der Anwendung auf das Hinzufügen der Dateien analysiert.

\item \textbf{Konvertierung} Da das hinzugefügte Dokument nicht nur einfach auf dem Server gespeichert wird, sondern dabei auch in ein anderes Format umgewandelt wird, muss diese Konvertierung zusätzlich überwacht werden. 
Wird beispielsweise ein Bild eingecheckt, wird dieses mehrfach in verschiedenen Auflösungen oder in einem anderen Bilddateiformat gespeichert. 
Ob diese Transformation erfolgreich ablief, kann anhand dieser neuen Dateien festgestellt werden.

\item \textbf{Indizierung} Bei dem Einchecken sollen auch gleichzeitig zusätzliche Informationen über das Dokument festgehalten werden. 
Diese Informationen können beispielsweise der Name des Autors, das Erstellungsdatum der Datei oder - bei Bildern - der verwendete Farbraum sein. 
Bei der Suche nach einem Dokument können diese Informationen als zusätzliche Suchkriterien verwendet werden.
Daher muss überprüft werden, ob diese Dateien richtig ausgelesen, der Datenbank hinzugefügt und vom Anwender abgefragt werden können.
Dabei werden auch zuvor ausgewählte Testdateien verwendet.

\item \textbf{Suchfunktion} Nach einer erfolgreichen Indizierung muss das eingecheckte Dokument per Suchanfrage gefunden werden.
%Ob die Suche und Indizierung erfolgreich abgelaufen ist, wird zusätzlich überprüft. 
Dabei wird der Suchbegriff an dem Dateinamen des Testbildes festgelegt.
\end{itemize}

\subsection{Zusammenfassung}

Die Basis für die alle anderen Tests bildet die Systemüberwachung.
An erster Stelle der Systemüberwachung steht die schlichte Erreichbarkeit über das Netzwerk per Ping.
Wenn der Server nicht erreichbar ist, können auch keine weiterführende Prüfungen durchgeführt werden.
Zur Systemüberwachung gehören auch allgemeine Informationen über die Systemressourcen wie freier Festplattenspeicher oder Prozessorauslastung.
Die nächste Stufe bildet die Überprüfung der laufenden Prozesse und der Status verschiedener Dienste bzw. Services.
Sollten bestimmte Prozesse nicht gefunden werden oder wichtige Dienste nicht gestartet sein, können auf diese Prozesse und Dienste aufbauende Checks nicht funktionieren.
Beispielweise kann der Funktionalitätstest der Benutzeranmeldung nicht realisierbar sein, wenn bereits zuvor in der Systemüberwachung der Prozess für den Webserver \gls{IIS} nicht gefunden werden konnte.\\


Alle Überwachungselemente lassen sich inklusive ihrer Abhängigkeiten in Form der Pyramide (Abbildung \ref{moniele}) darstellen.

\begin{figure}[ht]
	\centering
	   \fbox{\includegraphics[width=1\textwidth]{bilder/pyramide_fertig2.png}}
		\caption{Überwachungselemente}
		\label{moniele}
\end{figure} \newpage
\section{Umsetzung}
In diesem Kapitel wird die Vorgehensweise der zuvor beschriebenen Problemstellungen erörtert.

\subsection{Aufbau der Testumgebung}

Die für die Umsetzung notwendigen Ressourcen in Form eines Test-Servers und einer virtuellen Maschine.
\paragraph{Aufsetzen eines Nagios-Test-Systems}
Da die einzelnen Überwachungselemente in der Überwachungssoftware Nagios sukzessiv eingetragen werden müssen, ist ein häufiges Neustarten der Nagios-Anwendung notwendig, damit die neuen Konfigurationsdateien übernommen werden.

Damit dies nicht auf dem bereits verwendetem Nagios-Server durchgeführt werden muss, wird ein Nagios-Testserver für diesen Zweck eingesetzt.

Da Nagios ein Unix-ähnliches Betriebssystem erfordert, wird für diesen Zweck die Linux-Distribution Debian als Betriebssystem des Testservers verwendet.
Diese Distribution wird auch auf den Produktivservern des KIT verwendet.

\paragraph{Bilddatenbank als virtuelle Maschine}
Für die Simulation der verschiedenen Fehlerzuständen der einzelnen Überwachungselemente wird eine virtuelle Maschine mit einer \gls{OracleUCM} Prototypinstallation  als Entwicklungsplattform verwendet.

\subsection{Auswahl der geeigneten Überwachungsmethode}
\label{sectunixagents}
Wie in Kapitel \ref{methoden} aufgeführt, bietet Nagios verschiedene Möglichkeiten Informationen über zu überwachende Objekte zu sammeln.

\begin{itemize}
\item Überwachung direkt über das Netzwerk
\item Ausführung der Plugins durch \gls{SSH}-Verbindung
\item Ausführung von selbst vorkonfigurierten Kommandos durch \gls{NRPE}
\item Abfrage von Informationen durch \gls{SNMP}
\item Passiver Erhalt der Ergebnisse durch \gls{NSCA}
\end{itemize}

Für Dienste, die sich über das Netzwerk erreichen lassen, können die dafür entwickelten Nagios-Plugins direkt vom Nagios-Server verwendet werden.
Da Windows als Betriebssystem auf dem \gls{OracleUCM}-Server eingesetzt wird, kann die \gls{SSH}-Variante nicht eingesetzt werden.
Die \gls{NRPE}-Methode wird für die Auswertung der Logdateien verwendet.
Die Überwachung per \gls{SNMP} wird im Karlsruhe Insitute of Technology nicht eingesetzt, da man eine \gls{DoS}-Attacke durch das Senden von vielen \gls{SNMP}-Traps zu dem Nagios-Server verhindern will.\label{snmpkom}
%der höheren Komplexität dieser Variante und des begrenzten zeitlichen Rahmens dieser Arbeit nicht benutzt
Die passive Vorgehensweise mit \gls{NSCA} wird für die Umsetzung nicht benötigt und deshalb nicht verwendet.

Für die Überwachung von Windows-Servern wurde eine weitere Methode entwickelt, die auf dem Prinzip von \gls{NRPE} basiert.
Diese Variante wird NSClient genannt und benötigt, wie \gls{NRPE}, die Installation eines Nagios-Agenten auf dem zu überwachenden Server.
Daher muss eine Übersicht über verschiedene Nagios-Agenten erstellt werden.

\subsection{Übersicht Nagios-Agenten}
In diesem Unterkapitel werden die populärsten Agenten für Unix und Windows Betriebssysteme aufgelistet und nach den Punkten Sicherheit, subjektiver Aufwand für die Konfiguration und Art der Abfragemethode (aktiv oder passiv) verglichen.

\subsubsection{Unix-Agenten}
Für die auf Unix basierenden Betriebssysteme werden fünf verschiedene Möglichkeiten angeboten, die in Abbildung \ref{nagios-kern} als verschiedene Überwachungsmöglichkeiten von Nagios aufgelistet wurden.

%Tabelle Unix-Agenten


\begin{table}[h!]
\centering
\begin{threeparttable}[ht]
\begin{tabular}{l p{1.5cm} l p{1.5cm} l p{1.5cm} l p{1.5cm} l p{1.5cm} l p{1.5cm} p{1.5cm} p{1.5cm} p{1.5cm} p{1.5cm}}
 & \begin{turn}{50}\textbf{SSH}\end{turn} & \begin{turn}{50}\textbf{NRPE}\end{turn} & \begin{turn}{50}\textbf{SNMP}\end{turn} & \begin{turn}{50}\textbf{SNMP Traps}\end{turn} & \begin{turn}{50}\textbf{NSCA}\end{turn}\\ 
\hline
\textbf{Methode} & & & & & \\
\textit{aktiv} & \checkmark & \checkmark & \checkmark & - & - \\
\textit{passiv} & - & - & - & \checkmark & \checkmark\\
\textbf{Sicherheit} &  &  &  &  &  \\
\textit{Passwort} & - & - & \checkmark (v3) & \checkmark (v3) & -\\
\textit{Accesslist}\tnote{*} & \checkmark &  \checkmark & \checkmark (v2) & \checkmark (v2) & \checkmark \\
\textit{Verschlüsselung} &  \checkmark & \checkmark & \checkmark (v3) & \checkmark (v3) &  \checkmark \\
\textbf{Aufwand}\tnote{**} & \begin{footnotesize}leicht\end{footnotesize} & \begin{footnotesize}normal\end{footnotesize} & \begin{footnotesize}hoch\end{footnotesize} & \begin{footnotesize}hoch\end{footnotesize} & \begin{footnotesize}normal\end{footnotesize} \\
\end{tabular}
%\footnotesize
%* Einschränkung der Abfrage der Überwachungsinformationen anhand der \gls{IP}-Adresse
%\\
%** Subjektive Einschätzung
\begin{tablenotes}\footnotesize
      \item[*] Einschränkung der Abfrage der Überwachungsinformationen anhand der \gls{IP}-Adresse
        \item[**] Subjektive Einschätzung
    \end{tablenotes}
\caption{Übersicht der verschiedenen Unix-Agenten}
\end{threeparttable}
\end{table}




Dabei werden drei Agenten genannt, die eine aktive Ausführung der Nagios-Plugins benutzten.
Alle drei Agenten unterscheiden sich jedoch in den Punkten Sicherheit und Aufwand.
Der auf \gls{SSH} basierende Agent besitzt einen relativ geringen Aufwand für die Installation, da für den Aufbau der Kommunikation zwischen Nagios-Server und Client nur der öffentliche Schlüssel des Servers auf dem Client eingetragen werden muss.
Dadurch kann der Nagios-Server sich ohne Passwortabfrage an dem zu überwachendem Host anmelden und die sich darauf befindlichen Nagios-Plugins ausführen.
Da auf den meisten Unix-Servern bereits ein \gls{SSH}-Server läuft und deshalb kein weiterer Port geöffnet oder eine weitere Software installiert werden muss, ist diese Methode den anderen meist vorzuziehen.

Bei der \gls{NRPE}-Methode wird eine weitere Softwarekomponente auf dem Client installiert, die einen separaten Port für die Kommunikation mit dem Nagios-Server öffnet.
Wie bei dem Aufruf per \gls{SSH} müssen sich hier die Nagios-Plugins bereits auf dem Rechner befinden.
Dabei gilt als Unterschied dieser zwei ähnlichen Methoden zu beachten, dass für die Ausführung der Checks per \gls{SSH} ein extra Benutzerkonto auf dem Client erstellt werden muss und somit beliebige Systembefehle ausgeführt werden können, während die Ausführung von Kommandos bei \gls{NRPE} nur auf vorkonfigurierte Befehle beschränkt ist, wie in Kapitel \ref{sshnrpe} aufgeführt.
\label{unixagents}
Da \gls{SNMP} plattformunabhängig funktioniert ist es möglich diese Variante bei Unix- sowie bei Windowsservern einzusetzen.
Die verwendete \gls{SNMP}-Version bestimmt welche Sicherheitsmerkmale zur Verfügung stehen.
Zwar gibt es bereits seit Version 1 die Möglichkeit den Zugriff per Passwort in drei Gruppen aufzuteilen: kein Zugriff, Leserecht und Lese- mit Schreibrecht\footnote{Quelle: \cite{Barth08} S. 237}, jedoch wird dieses Passwort im Klartext übertragen, so dass es leicht auslesbar ist.
Auch die \gls{SNMP}-Version 2 inklusive der erweiterten Version 2c verwendet die gleiche unsichere Authentifizierung.
Erst ab Version 3 wird das Passwort verschlüsselt übertragen.
Während Barth behauptet, dass man bei \gls{SNMP} generell kein Passwort verwenden soll\footnote{Quelle: \cite{Barth08} S. 238}, da es leicht per Netzwerkmitschnittprogramme, wie WireShark, ausgelesen werden kann, wird in \cite{Jose07} S. 121 klargestellt, dass die Version 3 eine verschlüsselte Authentifizierung durch den MD5- oder SHA-Algorithmus ermöglicht.

Die passive Variante über \gls{SNMP} bei der der Client die Ergebnisse der Checks an den Nagios-Server sendet, auch \gls{SNMP}-Traps genannt, funktioniert nach dem gleichen Prinzip.
Da das Auslesen der \gls{MIB} per \gls{SNMP} im Gegensatz zu den anderen Varianten deutlich komplexer ist, wird der Aufwand als hoch eingestuft.

Ein weiterer Vertreter, der passive Checks ermöglicht, ist der \gls{NSCA}-Agent.
Wie die anderen Unix-Agenten bietet es die Möglichkeit den Datenaustausch zwischen Nagios-Server und Client zu verschlüsseln.
Alle Unix-Agenten erlauben es den Zugriff auf die Nagios-Plugins auf bestimmte \gls{IP}-Adressen zu beschränken.
Die Liste mit diesen \gls{IP}-Adressen nennt man auch \textit{Accesslist}.

\subsubsection{Windows-Agenten}
Da die zu überwachende Oracle UCM Anwendung auf einem Windows-Server betrieben wird und die bereits vorgestellten Agenten mit Ausnahme der \gls{SNMP}-Varianten nur unter Unix einsetzbar sind, müssen zusätzlich die explizit für Windows entwickelten Nagios-Agenten untersucht werden.
Dabei wird die Auswahl der Kandidaten auf vier Bewerber beschränkt, siehe Tabelle \ref{tab:winagents}.
%Tabelle Windows-Agenten
%\vspace{1.6cm}
\begin{table}[!cht]
\centering
\begin{threeparttable}
\begin{tabular}{l p{1.3cm} l p{1.3cm} l p{1.3cm} l p{1.3cm} l p{1.3cm} l p{1.3cm} p{1.3cm} p{1.3cm} p{1.3cm} p{1.3cm}}
 & \begin{turn}{50}\textbf{NSClient}\end{turn} & \begin{turn}{50}\textbf{NRPE\_NT}\end{turn} & \begin{turn}{50}\textbf{NC\_net}\end{turn} & \begin{turn}{50}\textbf{NSClient++}\end{turn} & \begin{turn}{50}\textbf{OpMon Agent}\end{turn}\\ 
\hline
\textbf{Methode} & & & & & \\
\textit{aktiv} & \checkmark & \checkmark & \checkmark & \checkmark & \checkmark\\
\textit{passiv} & - & - & \checkmark & \checkmark & -\\
\textit{check\_nt}\tnote{1} & \checkmark & - & \checkmark & \checkmark & \checkmark\\
\textit{NRPE}\tnote{2} & - & \checkmark & \checkmark & \checkmark & \checkmark\\
\textbf{Sicherheit} &  &  &  &  &  & \\
\textit{Passwort} & \checkmark & \checkmark & - & \checkmark & \checkmark\\
\textit{Accesslist}\tnote{3} & - & - & \checkmark & \checkmark & \checkmark\\
\textit{Verschlüsselung} & - & \checkmark & \checkmark & \checkmark & -\\
\textbf{Aufwand}\tnote{4} & \begin{footnotesize}normal\end{footnotesize} & \begin{footnotesize}hoch\end{footnotesize} & \begin{footnotesize}normal\end{footnotesize} & \begin{footnotesize}normal\end{footnotesize} & \begin{footnotesize}normal\end{footnotesize}\\
\end{tabular}
\begin{tablenotes}\footnotesize
		\item[1] Kompatibelität mit dem Standard check\_nt Plugin
		\item[2] Erlaubt Ausführung von vorkonfigurierten Kommandos
        \item[3] Einschränkung der Abfrage der Überwachungsinformationen anhand der \gls{IP}-Adresse
        \item[4] Subjektive Einschätzung
    \end{tablenotes}
\caption{Übersicht der verschiedenen Windows-Agenten}
\label{tab:winagents}
\end{threeparttable}
\end{table}

\newpage
Der NSClient-Dienst liefert die Möglichkeit lokale Windows-Ressourcen über das Netzwerk mit eigenem Port (Standport 1248) abzufragen.
Das Plugin \textit{check\_nt} wurde explizit für diesen NSClient-Dienst entwickelt und steht durch die Nagios-Plugins standardmäßig zur Verfügung.
Dadurch können die grundlegende Informationen für die Systemüberwachung aus Kapitel \ref{syschecks}, wie Zustände von Prozesse, Services, CPU-Auslastung, Festplattenplatz, usw. abgefragt werden.
Der Zugriff auf den NSClient-Dienst per \textit{check\_nt} wird in Abbildung \ref{fig:cknt} gezeigt.
\begin{figure}[ht]
	\centering
	   \fbox{\includegraphics[width=0.8\textwidth]{bilder/monitoring-windows.png}}
		\caption[Abfrage von Windows-Ressourcen durch \textit{check\_nt}]{Abfrage von Windows-Ressourcen durch \textit{check\_nt}\protect\footnote}
		\label{fig:cknt}
\end{figure}
\footnotetext{Quelle: \url{http://nagios.sourceforge.net/docs/3\_0/images/monitoring-windows.png}}

Diese Abfrage kann durch die Ausführung auf der Kommandozeile getestet werden:

\begin{figure}[ht]
	\centering
	   \fbox{\includegraphics[width=0.75\textwidth]{bilder/check_nt-stdp.png}}
		\caption{Zugriff auf den NSClient-Dienst durch check\_nt}
		\label{fig:ckntsh}
\end{figure}

Der erste und zugleich älteste Agent NSClient wird nicht mehr aktiv entwickelt und ist als aktuellste Version 2.0.1 aus dem Jahre 2003 bereits recht alt.
Daher wird auch keine Verschlüsselung der ein- und ausgehenden Daten unterstützt.
Auch bietet NSClient keine Möglichkeit aktiv vom Nagios-Server aus Nagios-Plugins oder weite Programme auszuführen, die sich auf dem zu überwachendem Host befinden.

Um dies auch für Windows-Server zu ermöglichen gibt es eine auf Windows portierte \gls{NRPE}-Variante, die sich NRPE\_NT nennt.
Hier lassen sich die Plugins direkt über den Nagios-Server aufrufen und die ausgetauschten Informationen werden verschlüsselt über das Netzwerk übertragen.

Beide bisher genannte Windows-Agenten bieten keine Möglichkeit eine \textit{Accesslist} anzulegen, erst das Programm NC\_net bietet diese Möglichkeit inklusive dem Sicherheitsmerkmal Verschlüsselung an.
Außerdem können durch den eingebauten \gls{NRPE}-Dienst aktiv Nagios-Plugins auf dem Client aufgerufen werden.
Als Besonderheit lassen sich durch NC\_net sowohl aktiv als auch passiv Testergebnisse an den Nagios-Server übertragen. 

Der Nagios-Agent NSClient++ besitzt diesselben Merkmale wie NC\_net, jedoch kann der Nagios-Server noch über ein Passwort zusätzlich verifiziert werden.

Die Möglichkeit Informationen per \gls{SNMP} und \gls{SNMP}-Traps abzufragen ist auch unter Windows möglich.
Dabei gelten die gleichen Richtlinien, Hinweise und Einschränkungen wie zuvor in Kapitel \ref{unixagents} aufgeführt.

\subsubsection{Auswahl und Konfiguration des Nagios-Agenten}

\paragraph{Auswahl}
Anhand der im vorherigen Kapitel beschriebenen Übersicht der Windows-Agenten und der daraus resultierenden Übersichtstabelle \ref{tab:winagents} wird ein geeigneter Kandidat für die Testumgebung ausgewählt.
Da nur ein Windows-Agent alle drei Sicherheitsmerkmale anbietet und dabei aktive und passive Überwachungsmethoden erlaubt, fällt die Wahl auf das Programm \textbf{NSClient++}.

\paragraph{Installation und Konfiguration}

Der Nagios-Agent NSClient++ kann im Gegensatz zu den meisten anderen Windows-Agenten komfortabel über einen graphischen Benutzerdialog installiert werden.
Während des Installationsvorgangs kann auch festgelegt werden welche Komponenten installiert werden sollen.
Dabei werden diese Komponenten nicht standardmäßig geladen, sondern im nächsten Dialogfenster auswählbar:

\begin{figure}[ht]
	\centering
	   \fbox{\includegraphics[width=0.55\textwidth]{bilder/nsc2.png}}
		\caption{Konfiguration des NSClient++ während der Installation}
		\label{nscs2}
\end{figure}

Außerdem können direkt während der Installation die \gls{IP}-Adresse bzw. der \gls{FQDN} des Nagios-Servers und das gewünschte Passwort eingetragen werden.

Durch die während des Installationsprozesses geladenen Komponenten für den NSClient- und \gls{NRPE}-Dienst können die Standard-Nagios-Plugins \textit{check\_nt} und \textit{check\_nrpe} mit dem Windows-Server verwendet werden.
Dabei läuft die Kommunikation zwischen dem Nagios- und dem Windows-Server folgendermaßen ab:

\begin{figure}[ht]
	\centering
	   \fbox{\includegraphics[width=0.7\textwidth]{bilder/nagios-active-nsclient-and-nrpe.png}}
		\caption[Kommunikation zwischen Nagios und NSClient++]{Kommunikation zwischen Nagios und NSClient++\protect\footnote}
		\label{nscs2}
\end{figure}
\footnotetext{\url{http://nsclient.org/nscp/}}
%http://nsclient.org/nscp/raw-attachment/wiki/doc/usage/nagios/nagios-active-nsclient.png

Bei Windows-Server mit vielen Verbindungen und Diensten können Remote Procedure Calls (\gls{RPC}), bei denen dynamisch Ports ab 1025 verwendet werden, bereits den Standardport des NSClient-Dienstes (1248) unter Umständen bereits vor dem Start des Dienstes belegen.\footnote{Quelle: \cite{Barth08} S. 481}
Um dies zu verhindern wurde der Port des NSClient-Dienstes beim NSClient++ bereits vom Entwickler auf einen höheren Port (12489) gewechselt.

Alle bisherigen Einstellungen können in der Konfigurationsdatei \textit{NSC.ini}, die sich in dem Installationsverzeichnis des NSClient++ befindet, verändert werden.
In dieser Datei befinden sich noch mehr Einstellungsmöglichkeiten; im Folgenden werden nur für die Umsetzung relevanten (notwendig essentiell benötigten) Parameter aufgelistet.

\begin{lstlisting}[captionpos=b, caption=NSClient++ Konfigurationsdatei, label=code:nsc, breaklines = true, language=sh]
;# NSCLIENT PORTNUMMER
;#  Die Portnummer des NSClient-Dienstes
port=13596

;# NRPE PORTNUMMER
;#  Die Portnummer des NRPE-Dienstes
port=13597

;# SSL SOCKET
;#  Die Aktivierung von SSL der Kommunikation zwischen Nagios- und Windows-Server 
use_ssl=1

;# NRPE BEFEHLSDEFINITIONEN
;# Definitionen der Befehle, die durch den NRPE-Dienst aufrufbar sind
check_uname=scripts\check_uname.exe
check_reflog=scripts\check_logfiles.exe -f scripts\logfile.cfg
\end{lstlisting}

Damit die vorgenommen Änderung übernommen werden, muss der Dienst des NSClient++ neu gestartet werden. 

%\begin{figure}[ht]
%	\centering
%	   \fbox{\includegraphics[width=0.85\textwidth]{bilder/nsc3.png}}
%		\caption{NSClient++ Windowsdiensteintrag}
%		\label{nscs3}
%\end{figure}

Durch das Ausweichen auf höher gelegene Portnummern können die zuvor genannten Probleme aufgrund der \gls{RPC}s verhindert werden.

Der bereits höher liegende Standardport des NSClient-Dienstes beim NSClient++ wird zusätzlich noch abgeändert, damit die Tatsache, dass sich ein Nagios-Agent auf dem Computer befindet, nicht sofort ersichtlich ist.
Dieser sicherheitstechnische Aspekt wurde bereits in Kapitel \ref{changeport} behandelt.

Damit die \gls{SSL}-Verschlüsselung zwischen den Servern aktiviert wird, muss man es explizit in der Konfigurationsdatei mit der Option \textit{use\_ssl=1} angeben.

Die Definitionen der \gls{NRPE}-Kommandos dienen dafür, dass durch den Nagios-Server per \textit{check\_nrpe} mit dem Befehlsnamen der darauf folgende Befehl ausgeführt wird.

Aufgrund der abgeänderten Portnummer muss man den Port bei dem Aufruf explizit angeben.
Ein Aufruf eines solchen \gls{NRPE}-Kommandos vom Nagios-Server wird in der folgenden Abbildung gezeigt:

\begin{figure}[ht]
	\centering
	   \fbox{\includegraphics[width=0.85\textwidth]{bilder/nrpe-check.png}}
		\caption{Aufruf eines NRPE-Kommandos}
		\label{nrpecheck}
\end{figure}



Anhand des Befehlsnamens \textit{check\_uname} führt der \gls{NRPE}-Dienst die in der Konfigurationsdatei eingetragene Datei \textit{check\_uname.exe} aus.

Der Aufruf um Informationen durch den NSClient-Dienst abzufragen sieht ähnlich aus:

\begin{figure}[ht]
	\centering
	   \fbox{\includegraphics[width=0.85\textwidth]{bilder/nsclient-check.png}}
		\caption{Aufruf des NSClient-Dienstes}
		\label{ntcheck}
\end{figure}

Die Servicedefinition des vorherigen NSClient-Aufrufs muss wie nachfolgend/folgt in der Nagios-Konfiguration eingetragen:
\begin{lstlisting}[captionpos=b, caption=Servicedefinition des NSClient-Checks, label=nt-servdef, breaklines = true, language=sh]
define service{
        use                     generic-service
        host_name               example.kit.edu
        service_description     Uptime
        check_command           check_nt!-p 13596 -s secret -v UPTIME
        }
\end{lstlisting}

Damit nicht jeder einzelne Serviceeintrag abgeändert werden muss, falls sich der Port oder das Passwort des zu überwachenden Computers ändert, können eigene Befehlsdefinitionen erstellt werden.

\begin{lstlisting}[captionpos=b, caption=Server spezifische Befehlsdefinition, label=cus-nt-servdef, breaklines = true, language=sh]        
define command{
        command_name    check_nt_example
        command_line    /usr/lib/nagios/plugins/check_nt -H $HOSTNAME$ -p 13597 -p secret -v $ARG1$
        }
\end{lstlisting}

Dadurch muss nur diese Befehlsdefinition bei einer Änderung bearbeitet werden.
Die vorherige Servicedefinition in Listing \ref{ntcheck} kann dann in verkürzter Form eingetragen werden:

\begin{lstlisting}[captionpos=b, caption=Verkürzte Servicedefinition des NSClient-Checks, label=nt-servdef, breaklines = true, language=sh]
define service{
        use                     generic-service
        host_name               example.kit.edu
        service_description     Uptime
        check_command           check_nt_example!UPTIME
        }
\end{lstlisting}




\subsection{Umsetzung der Systemüberwachung}

Die in Kapitel \ref{syschecks} aufgelisten Prozesse und Services können durch den NSClient-Dienst vom Nagios-Server überwacht werden.
Dafür wird der in Listing \ref{cus-nt-servdef} definierte verkürzte Befehl für \textit{check\_nt} benutzt.

\begin{lstlisting}[captionpos=b, caption=Prozess- und Service-Check Servicedefintionen, label=procservdef, breaklines = true, language=sh]
#Prozess des IIS Webservers
define service{
        use                     generic-service
        host_name               example.kit.edu
        service_description     IIS Prozess
        check_command           check_nt_example!PROCSTATE -l w3wp.exe
        }

#Zeitdienst
define service{
        use                     generic-service
        host_name               example.kit.edu
        service_description     Zeitdienst
        check_command           check_nt_example!SERVICESTATE -l W32TIME
        }
\end{lstlisting}

Mit diesen zwei Einträgen wird der Prozess des \gls{IIS}-Webservers und der Status des Dienstes zum Zeitabgleich überwacht.
Andere Prozesse und Dienste lassen sich nach dem gleichen Schema überwachen.

Nach einem Neustart von Nagios werden beide Einträge im Webinterface angezeigt:

\begin{figure}[ht]
	\centering
	   \fbox{\includegraphics[width=0.95\textwidth]{bilder/servproc.png}}
		\caption{Prozess- und Dienstüberwachung im Nagios-Webinterface}
		\label{servprocgui}
\end{figure}

Die Festplattenspeicherausnutzung und die Prozessorauslastung wird auf ähnliche Weise überwacht.
Hierbei muss beachtet werden, dass die Testergebnisse nicht eindeutig sind, im Gegensatz zu der Service- und Prozessüberwachung.
Wann Nagios alarmieren soll muss vom Anwender in Form von Parametern festgelegt werden.

\begin{lstlisting}[captionpos=b, caption=Überwachung der Festplatten- und Prozessorauslastung, label=cpuhdddef, breaklines = true, language=sh]
#Belegung der Partition C:
define service{
        use                     generic-service
        host_name               example.kit.edu
        service_description     C:\ Drive Space
        check_command           check_nt_example!USEDDISKSPACE -l c -w 85 -c 100
        }
        
#CPU Auslastung der letzten 5 Minuten
define service{
        use                     generic-service
        host_name               example.kit.edu
        service_description     CPU Load
        check_command           check_nt_example!CPULOAD -l 5,80,100
        }
\end{lstlisting}

Für diese Festplattenüberwachung versendet Nagios eine Warnung, wenn der belegte Speicherplatz auf der C-Partition die 85\% Marke überschreitet und meldet einen kritischen Fehler bei 100\%.
Bei der Prozessorüberwachung schlägt Nagios Alarm, wenn der Mittelwert der Auslastung in den letzten fünf Minuten mehr als 80\% bzw. 100\%  betragen hat.


\subsection{Umsetzung der Funktionlitätstest}

Für die Ausführung der einfachen Funktionlitätstest aus Kapitel \ref{funztest} werden Benutzerinformationen zur Anmeldung benötigt.
Nagios besitzt extra hierfür die Möglichkeit diese Benutzerinformationen in Variablen zu speichern, damit sie nicht einzeln bei jeder Servicedefinition verändert werden müssen.
Da sich die Definition dieser Variablen in einer externen Datei befindet, können die Zugriffsrechte auf diese Datei eingeschränkt werden, wodurch die Anmeldedaten bei den Servicedefinitionen nicht auslesbar sind.

\begin{lstlisting}[captionpos=b, caption=Funktionalitätstest der Benutzeranmeldung, label=userauthdef, breaklines = true, language=sh]
#Anmeldung an Oracle UCM mit lokalem Benutzerkonto
define service{
        use                     generic-service
        host_name               example.kit.edu
        service_description     Anmeldung Oracle UCM als lokaler Benutzer
        check_command           check_http!-u "/bdb/idcplg?IdcService=LOGIN&Action=GetTemplatePage&Page=HOME\_PAGE&Auth=Internet"  -a $USER3$:$USER4$ -e "Sie sind angemeldet als" -S
        }
\end{lstlisting}


Dabei werden dem Nagios-Plugin \textit{check\_http} mit dem \pictext{u}-Parameter die URL zur Benuteranmeldungseite und mit dem Parameter \pictext{a} der benutzername und -passwort mitgegeben.
Der nach dem Parameter \pictext{e} folgende String wird dann in der Antwort des Servers gesucht.
Sollte dieser String nicht gefunden werden ist die Authentifizierung fehlgeschlagen und es wird durch Nagios eine Meldung versendet.
Mit dem Parameter \pictext{S} wird angegeben, dass eine \gls{SSL}-verschlüsselte Verbindung zum Webserver über HTTPS hergestellt werden soll, ansonsten würden die Benutzerinformationen im Klartext übertragen werden, wodurch sie leicht für Angreifer auslesbar wären.

Für das Auslesen von Informationen aus der Statusseite der Oracle UCM-Anwendung wurde ein einfaches BASH-Script entwickelt:

\begin{lstlisting}[captionpos=b, caption=Auslesen der Verbindungen zur Datenbank, label=dbcon, breaklines = true, language=sh]
#!/bin/bash
E_BADARGS=2
if [ ! -n "$6" ]
then
        echo "Usage: `basename $0` -url <URL> -u <username> -p <password>"
        exit $E_BADARGS
fi

DBCONNECTIONS=$(wget -qO-  --user $4 --password $6 $2 | grep "System Database")
DBCONNECTIONS=${DBCONNECTIONS##*>}
DBCONNECTIONS=${DBCONNECTIONS%% *}
echo $DBCONNECTIONS
\end{lstlisting}

Dabei muss als URL die Seite mit den Datenbankverbindungen und gültige Benutzerinformationen mitgegeben werden.
Anschliessend wird die aufgerufene Seite nach der gewünschte Informationen untersucht und ausgegeben.

Dieses einfaches Script kann auch dazu verwendet um andere Informationen von der Statusseite abzufragen.

\subsection{Auswertung der Logdateien}

Um die drei genannten Logdateien aus Kapitel \ref{checklog} auszuwerten wird durch den \gls{NRPE}-Dienst das Plugin \textit{check\_logfiles} von Gerhard Laußer\footnote{Quelle: \url{http://www.consol.de/opensource/nagios/check-logfiles}} eingesetzt.

Dieses Plugin besitzt bereits einige nützliche Funktionen für die Überwachung von Logdateien.
Durch das Setzten eines Zeitstempels filtert das Plugin veraltete Einträge heraus und untersucht nur neu hinzugekommene Zeilen.
Der Rotationsalgorithmus der \gls{OracleUCM}-Logdateien kann dem Plugin durch die Verwendung einer Konfigurationsdatei mitgeteilt werden.

Die Konfigurationsdatei für das \textit{check\_logfiles}-Plugin wird im folgendem Listing gezeigt:


\begin{lstlisting}[captionpos=b, caption=Konfigurationsdatei für \textit{check\_logfiles}, label=chklogcfg, breaklines = true, language=sh]
@searches = ({
  tag => 'ucmlogs',
  type => 'rotating::uniform',
  logfile => 'D:/bdb2/weblayout/groups/secure/logs/bdb/IdcLnLog.htm',
  rotation => 'refinery\d{2}\.htm',
  warningpatterns => [
        'Cannot identify file', #Testbild korrupt
	  'Bad CRC value in IHDR chunk', #Konvertierung nicht erfolgreich
	  'Der Dateiname darf nicht länger als 80 Zeichen sein', #Zu langer Dateiname
    ],
});
\end{lstlisting}

Mit dem \pictext{tag}-Attribut wird diese Auswertung eindeutig identifizierbar gemacht, da man in der gleichen Konfigurationsdatei mehrere Logdateien bzw. weitere Durchsuchungen definieren kann.
Das Attribut \pictext{type} muss hier so gesetzt werden, da die aktuelle Logdatei und die wegrotierte Logdatei das gleichen Namensschema benutzten.
Der Pfad zu den Logdateien wird über das Attribut \pictext{logfile} gesetzt.
Da die einzelnen Logdateien mit einem bestimmten Namensmuster erstellt werden (siehe Kapitel \ref{checklog}), muss dieses Namensmuster hier direkt angegeben werden.
In diesem Falle durchsucht das \textit{check\_logfiles}-Plugin alle Logdateien mit dem Namen \textit{refinery00.htm} bis \textit{refinery99.htm}.
Alle gefundenen Dateien werden dem Datum nach sortiert und die aktuellste Datei wird untersucht.
Nach welchen Stopwörtern gesucht werden soll wird mit dem Attribut \pictext{warningpatterns} angegeben, sofern mindestens einer dieser Strings gefunden wurde liefert das Plugin ein WARNING als Rückmeldung inklusive der Zeile in dem das Stopwort gefunden wurde.

\begin{figure}[ht]
	\centering
	   \fbox{\includegraphics[width=0.95\textwidth]{bilder/logcheckwarn.png}}
		\caption{Ausgabe der betreffenden Zeile in der Logdatei}
		\label{checklogwarn}
\end{figure}


\subsection{Benutzersimulation}

Um die Funktionalität der Anwendung eindeutig festzustellen werden typische Benutzeraktionen simuliert und die Ergebnisse an Nagios übermittelt.

Solche typische Aktionen sind das Einchecken eines Bildes, Suche nach einem Bild und schließlich die Anforderung des Originalbildes und der konvertierten Bilder.
%Benutzertätigkeiten, Handlungen
Da Nagios standardmäßig ein Plugin periodisch jede fünf Minuten aufruft, würde sich die Festplatte und die Datenbank des \gls{OracleUCM}-Servers im Laufe der Zeit an ihre Kapazitäten stoßen.
Daher werden nach der Anforderung und Überprüfung der Bilder alle Testbilder vom Server entfernt.

Der Ablauf der Benutzersimulation soll in verkürzter Form durch folgendes Struktogramm verdeutlicht werden:

\begin{figure}[ht]
	\centering
	   \includegraphics[width=0.9\textwidth]{bilder/Benutzersimulation.png}
		\caption{Geplanter Ablauf der Benutzersimulation}
		\label{user-sim}
\end{figure}



Per Webservice soll ein Testbild an den Server geschickt und eingecheckt werden.
In diesem ersten Schritt wird auch gleichzeitig die Erreichbarkeit der Anwendung über das Netzwerk getestet.

Wenn die Übertragung des Bildes erfolgreich wird anschließend nach dem soeben eingecheckten Bild per Dateinamen gesucht um die Funktionalität der Indizierung zu kontrollieren.

Sollte das Bild gefunden werden, wird es vom Nagios-Server angefordert und auf seine Korrektheit überprüft.
Der gleiche Test wird mit den konvertieren Bildversionen durchgeführt, um die Funktion der Konvertierung zu überwachen.

Falls alle Tests erfolgreich waren, wird das Testbild und alle konvertierten Bilder vom \gls{OracleUCM}-Server gelöscht.
Bei den anderen Szenarien gibt das Plugin den Wert 2 für den Status CRITICAL zurück.\\

Die Realisierung dieser Simulation wird durch zwei Plugins realisiert.

\paragraph{Einchecken eines Testbildes}
Das erste Plugin dient zum Einchecken des Testbildes.
Dabei ruft der Nagios-Server ein auf \gls{PHP}-basierendes Script auf.
In diesem Script wird die \gls{PHP}-Klasse \textit{nuSOAP} eingebunden, damit man vereinfacht auf Web Services zugreifen kann.
Die Kommunikation zwischen Client und Server bei der Benutzung eines Web Services findet, wie im Kapitel \ref{webservice} beschrieben, im \gls{XML}-Format statt.
Um den Aufwand zu vermeiden diese \gls{XML}-Datei immer selbst zu erstellen, wird mit Hilfe der \gls{WSDL}-Datei auf dem \gls{OracleUCM}-Server die benötigten Parameter beim Aufruf eines Web Services von \textit{nuSOAP} ausgelesen.

In der folgenden Abbildung werden aus der \gls{WSDL}-Datei alle möglichen Anforderungsparameter für den Web Service \textit{CheckInUniversal} gezeigt.

\begin{figure}[ht]
	\centering
	   \fbox{\includegraphics[width=0.9\textwidth]{bilder/wsdlscrn.png}}
		\caption{Anforderungsparameter für CheckInUniversal aus der WSDL-Datei}
		\label{wsdl1}
\end{figure}


%\begin{lstlisting}[captionpos=b, caption=Anforderungsparameter aus der WSDL-Datei, label=1stwsdl, breaklines = true, language=xml]
%<s:element name="CheckInUniversal">
% <s:complexType>
%  <s:sequence>
%   <s:element minOccurs="0" maxOccurs="1" name="dDocTitle" type="s:string"/>
%   <s:element minOccurs="0" maxOccurs="1" name="dDocType" type="s:string"/>
%   <s:element minOccurs="0" maxOccurs="1" name="dDocAuthor" type="s:string"/>
%   <s:element minOccurs="0" maxOccurs="1" name="dSecurityGroup" type="s:string"/>
%   <s:element minOccurs="0" maxOccurs="1" name="dDocAccount" type="s:string"/>
%   <s:element minOccurs="0" maxOccurs="1" name="primaryFile" type="s0:IdcFile"/>
%  </s:sequence>
% </s:complexType>
%</s:element>
%\end{lstlisting}

Im PHP-Script werden nach dem Einlesen dieser \gls{WSDL}-Datei und der Authentifizierung am \gls{OracleUCM}-Server die benötigten Parameter beim Aufruf des Web Services gesetzt und die Ausgabe des Servers ausgewertet.

\begin{lstlisting}[captionpos=b, caption=Aufruf des Web Services CheckInUniversal, label=1stplugin, breaklines = true, language=PHP]
$soap = new soapclient($WSDL-URL,  //WSDL-Datei einlesen 
array('login' => $user, 'password' => $password)); //Authentifizierung am Oracle UCM-Server

//Aufruf des Web Services
$ergebnis = $soap->CheckInUniversal(array(
	'dDocAuthor'=>$user, //Autor des Bildes
	'dDocTitle'=>'testBild4nagios', //Titel des Bildes
	'dSecurityGroup'=>'private', //Sichtbarkeit des Bildes
	'dDocAccount'=>'NAGIOS/TEST', //Angabe einer Gruppe
	'dInDate'=>date("d.m.y H:i"), //Aktuelles Datum
	'dDocType'=>'Picture', //Dokumententyp
	'doFileCopy'=>'1', //Datei nur kopieren, nicht verschieben
	'dDocFormat'=>'image/png', //MIME-Type
	'primaryFile'=>array(
		'fileName'=>'testBild4nagios',
 		'fileContent'=>$content) //Byteweise eingelesenes Bild
));
[...]
//Auswertung der Antwort des Servers
if (ereg(' erfolgreich eingecheckt.', $output)) {
  echo('CHECKIN OK - '.$output);
  die(0); //Einchecken erfolgreich
} else {
 echo('CHECKIN CRITICAL - '.$output);
 die(2); //Einchecken fehlgeschlagen
}
\end{lstlisting}

Die Bilddatei muss für die Übertragung über \gls{HTTP} zuerst byteweise eingelesen werden und anschließend mit dem base64-Algorithmus kodiert werden.
Dabei übernimmt die nuSOAP-Klasse die base64-Enkodierung.

Der Ablauf dieses Plugins soll durch folgende Abbildung verdeutlicht werden: 
\begin{figure}[ht]
	\centering
	   \fbox{\includegraphics[width=0.93\textwidth]{bilder/wsdl.png}}
		\caption{Einchecken eines Testbildes}
		\label{usersim}
\end{figure}

\begin{enumerate}
\item Im ersten Schritt wird das \gls{PHP}-Script vom Nagios-Server aufgerufen und liest das Testbild ein. Anschließend verwendet es die \gls{WSDL}-Datei des Web Services um das entsprechende \gls{XML}-Dokument zu erstellen. Diese \gls{XML}-Datei wird anhand der nuSOAP-Klasse \gls{SOAP}-konform an den \gls{OracleUCM}-Servers gesendet.
\item Die \gls{XML}-basierende Rückantwort des Servers wird auch anhand der \gls{WSDL}-Datei erstellt und kann vom \gls{PHP}-Script ausgewertet werden.
\end{enumerate}


\paragraph{Validierung der Indizierung und Konvertierung}
Der zweite Teil der Benutzersimulation überprüft, ob das Testbild erfolgreich indiziert und konvertiert wurde.
Dabei verwendet es die gleichen Grundfunktionen wie das erste Plugin.
Jedoch wird anstatt dem Web Service \textit{CheckInUniversal} die \gls{WSDL}-Datei des Web Services \textit{AdvancedSearch} verwendet und aufgerufen.

\begin{lstlisting}[captionpos=b, caption=Überprüfen der Indizierung anhand einer Suchanfrage, label=2stplugin, breaklines = true, language=PHP]
$ergebnis = $soap->AdvancedSearch(
	'queryText'=>"dDocTitle <substring> `testBild4Nagios"
);
\end{lstlisting}

Durch diesen Aufruf wird anhand seines Titels nach einem zuvor eingechecktem Testbild gesucht.
Dadurch kann überprüft werden, ob das Bild korrekt vom Server angenommen und indiziert wurde.
In der Rückantwort des Servers befindet sich unter anderem die eindeutige Identifikationsnummer des Testbildes.
Diese Nummer wird für die Validierung der Konvertierung verwendet.
\begin{lstlisting}[captionpos=b, caption=Überprüfen der Indizierung anhand einer Suchanfrage, label=2stplugin, breaklines = true, language=PHP]
//Test des Originalbildes
$ergebnisGet = $soap->GetFileByID('dID'=>$dID);
[...]
if(!mb_eregi('PNG', $outputGetOrig))
{
        echo('SEARCH CRITICAL - Originalbild ist nicht im PNG Format!');
        die(2); //Originalbild korrupt
}
//Test der Thumbnailversion des Testbildes
$ergebnisGetThumbnail = $soap->GetFileByID(array('dID'=>$dID, 'rendition' => 'Thumbnail'));
[...]
if(!mb_eregi('JFIF', $outputGetThumbnail))
{
        echo('SEARCH CRITICAL - Thumbnailversion des Testbildes ist nicht im JPEG Format!');
        die(2); //Thumbnailversion korrupt
}
[...] //Überprüfung der anderen konvertierten Bilder
\end{lstlisting}

Sofern das Bild gefunden wurde, wird es vom Plugin anschließend angefordert und nach dem Dateityp untersucht.
Die Konvertierung wird dadurch überprüft indem die konvertierten Versionen des Testbildes angefordert werden und wie das Originalbild auf einen gültigen Dateityp getestet werden.

Der Ablauf dieses Plugins ist dem ersten sehr ähnlich, siehe Abbildung \ref{usersim2}

\begin{figure}[ht]
	\centering
	   \fbox{\includegraphics[width=0.93\textwidth]{bilder/wsdl-valid.png}}
		\caption{Validierung der Indizierung und Konvertierung}
		\label{usersim2}
\end{figure}

\begin{enumerate}
\item Zuerst wird die \gls{WSDL}-Datei für den Web Service \textit{AdvancedSearch} eingelesen. Eine Suchanfrage nach dem Testbild wird wieder über eine \gls{XML}-Datei an den Server gesendet. Wenn eine Datei gefunden wurde, wird das Originalbild und die konvertierten Versionen anhand der Identifikationsnummer angefordert.
\item Diese Bilder werden wieder byteweise innerhalb der \gls{XML}-Datei der Serverantwort an den Nagios-Server übertragen und auf ihre Korrektheit überprüft. Sollten alle bisherigen Tests ohne Probleme abgelaufen sein, wird das Testbild und die konvertierten Bilder vom Server entfernt.
\end{enumerate}

Standardmäßig bietet \gls{OracleUCM} keinen Web Service zum Löschen von Dokumenten an.
Daher muss zunächst eine \gls{WSDL}-Datei dafür erstellt werden, siehe Abbildung \ref{cwsdl}.

\begin{figure}[ht]
	\centering
	   \fbox{\includegraphics[width=0.93\textwidth]{bilder/cwsdl2.png}}
		\caption{Anlegen eines eigenen Web Services}
		\label{cwsdl}
\end{figure}


Der Name der \gls{WSDL}-Datei und des Web Services kann beliebig gewählt werden, solange als Service die entsprechende interne Bezeichnung zum Löschen von Dokumenten \pictext{DELETE\_REV} verwendet wird.\footnote{Quelle: \cite{Huff06} S. 379}
Um Zweideutigkeiten zu vermeiden, wird als Anforderungsparameter die eindeutige Identifikationsnummer verwendet.

Der eigene Web Service wird wie die anderen im Anschluss aufgerufen:

\begin{lstlisting}[captionpos=b, caption=Aufruf des eigenen Web Services, label=2stplugin2, breaklines = true, language=PHP]
$ergebnisDelete = $soap->DeleteRevisionByID('dID'=>$dID);
\end{lstlisting}
Die erstellten PHP-Dateien müssen noch Nagios als Befehl hinzugefügt werden.
\begin{lstlisting}[captionpos=b, caption=Befehldefinitionen der Benutzersimulation, label=phpdef, breaklines = true, language=sh]
#Einchecken eines Testbildes
define command{
        command_name    check_ucm_checkin
        command_line    /usr/lib/nagios/plugins/check_ucm/nagiosCheckin.php
        }
#Validierung der Indizierung und Konvertierung
define command{
        command_name    check_ucm_search
        command_line    /usr/lib/nagios/plugins/check_ucm/nagiosSearch.php
        }
\end{lstlisting}

Die Aufteilung der Benutzersimulation in zwei Nagios-Kommandos sorgt dafür, dass es keine Garantie für die Reihenfolge der Ausführung der Plugins gibt.
Aufgrund diese Asynchronität und dem Umstand, dass die Indizierung und Konvertierung des Testbildes abhängig von der Auslastung des \gls{OracleUCM}-Servers ist, wird die Anzahl der Ausführungen des Plugins für den Wechsel von Soft- zum Hardstate erhöht.

\begin{lstlisting}[captionpos=b, caption=Angepasste Servicedefinition für die Benutzersimulation, label=phpdef, breaklines = true, language=sh]
define service{
        use                     generic-service
        host_name               example.kit.edu
        service_description     Oracle UCM Search Delete
        max_check_attempts      8
        check_command           check_ucm_search
        }
\end{lstlisting}

%\begin{itemize}
%\item \url{http://www.w3schools.com/soap/default.asp} Web Services und SOAP
%\item \url{http://www.w3schools.com/wsdl/wsdl_summary.asp} WSDL
%\item max attempts bei Search erhöhen, da Auslastung der InboundRef -> möglichst keine/geringe False Positives -> auf Ausblick verweisen, Rahmenbedingungen müssen im Feld in der Praxis erst noch gefunden werden
%\end{itemize}



 \newpage
\section{Ergebnis}

\begin{itemize}
\item Vllt. Übersicht wie was überwacht wird
\item Screenshots von Nagios
\item Exportierfähigkeit, was muss alles auf dem Live-Nagios Server gemacht werden
\end{itemize}

\begin{figure}[ht]
	\centering
	  \fbox{\includegraphics[width=0.9\textwidth]{bilder/demo.png}}
		\caption{Webinterface von Nagios}
		\label{user-sim}
\end{figure} \newpage
\section{Zusammenfassung und Ausblick}

\begin{itemize}
\item Geeignete Stopwörter für Logdateien müssen noch gefunden / eruiert werden
\item Passende Schwellwertdefinitionen können erst nach einer gewissen Laufzeit festegelegt werden
\item Export der entwickelten Überwachung auf den produktiven Haupt-Nagios Server
\end{itemize} \newpage
\chead{GLOSSAR}
\makeglossaries
\printglossaries
\newpage
\chead{\leftmark}
\listoffigures
\newpage

\renewcommand{\lstlistlistingname}{Codelistingverzeichnis}
\lstlistoflistings
\newpage
\renewcommand{\listtablename}{Tabellenverzeichnis}
\listoftables
\newpage

\renewcommand{\refname}{Quellenverzeichnis} 
%\subsection{Literaturverzeichnis}
%\subsection{Quellverzeichnis}

%\subsubsection{Literaturverzeichnis}

%\bibliographystyle{unsrt}   % this means that the order of references
%			    % is dtermined by the order in which the
%			    % \cite and \nocite commands appear
%\bibliography{bach}  % list here all the bibliographies that
%			     % you need. 

\begin{thebibliography}{xxxxxxxxxx}
%\paragraph{Literatur}

	 \bibitem[DMS08]{DMS08}Götzer; Schmale; Maier; Komke (2008) \pictext{Dokumenten-Management - Informationen im Unternehmen effizient nutzen} 4. Auflage,\\
	 dpunkt.verlag GmbH Heidelberg,  
	 ISBN13: 978-3-89864-529-4, Einsichtnahme: 25.06.2009

	 \bibitem[Barth08]{Barth08}Wolfgang Barth (2008) \pictext{Nagios - System- und Netzwerk-Monitoring} 2. Auflage, \\
	 ISBN13: 978-3-937514-46-8, Einsichtnahme: 25.05.2009
	 
	 \bibitem[Huff06]{Huff06}Brian Huff (2006) \pictext{The Definitive Guide to Stellent Content Server Development}, \newline ISBN13: 978-1-59059-684-5, Einsichtnahme: 25.05.2009

	 \bibitem[Jose07]{Jose07}David Josephsen (2007) \pictext{Bulding a monitoring infrastructure with Nagios}, \newline ISBN13: 0-132-23693-1, Einsichtnahme: 16.06.2009

	 \bibitem[SOA07]{SOA07}Hurwitz; Bloor; Baroudi; Kaufman; (2007) \pictext{Service Oriented Architecture For Dummies}, \newline ISBN13: 978-0-470-05435-2, Einsichtnahme: 29.07.2009
	 
		 \bibitem[Melzer08]{Melzer08}Ingo Melzer (2007) \pictext{Service-orientierte Architekturen mit Web Services: Konzepte - Standards - Praxis}, \newline ISBN13: 978-9-8274-1993-4, Einsichtnahme: 29.07.2009	 
		 
		 	 \bibitem[Munin08]{Mu08} Gabriele Pohl und Michael Renner (2008) 
	 \pictext{Munin - Graphisches Netzwerk- und System-Monitoring}, \newline ISBN13: 978-3-937514-48-2, Einsichtnahme: 05.04.2009\\	
	 	
%\paragraph{Internetquellen}
	 	 	 	 \bibitem[Nagios]{Nagios} Ethan Galstad (2009) \pictext{Nagios Enterprises}, \\ Quelle: \url{http://nagios.com/} \newline Stand: 2009, Einsichtnahme: 14.07.2009
 	
 		 	 	 \bibitem[NagiosFAQ]{NagiosFAQ} Ethan Galstad (2009) \pictext{What does Nagios mean?}, \\ Quelle: \url{http://support.nagios.com/knowledgebase/faqs/index.php} \newline Stand: 02.06.2009, Einsichtnahme: 09.06.2009 	 
	 
	 \bibitem[UCM07]{UCM07} Ohne Verfasser (2007) \pictext{Oracle Application Server Documentation Libary - Oracle Content Management 10gR3}, \\ Quelle: \url{http://download-west.oracle.com/docs/cd/E10316_01/cs/cs_doc_10/documentation/integrator/getting_started_10en.pdf} \newline Einsichtnahme: 16.06.2009
	 
	 	 \bibitem[UCMlog09]{UCMlog09} Unbekannter Author "`vramanat"' (2009) \pictext{Universal Content Management 10gR3 - Content Server Log File Information}, \\ Quelle: \url{http://www.oracle.com/technology/products/content-management/cdbs/loginfo.pdf} \newline Stand: 20.01.2009, Einsichtnahme: 05.06.2009
	 	 
	 	 \bibitem[OraPress]{OraPress} Letty Ledbetter (2009) \pictext{Oracle Press Release - Oracle Buys Stellent}, \\ Quelle: \url{http://www.oracle.com/corporate/press/2006_nov/stellent.html} \newline Stand: 02.11.2006, Einsichtnahme: 16.06.2009
		 \newpage 	 
	 	 	 \bibitem[W3WS04]{W3WS04} Booth; Haas; McCabe u.a. (2004) \pictext{Web Services Architecture - W3C Working Group Note 11 February 2004}, \\ Quelle: \url{http://www.w3.org/TR/ws-arch/wsa.pdf} \newline Stand: 11.02.2004, Einsichtnahme: 29.07.2009

	 \bibitem[NSClient]{NSClient}  Michael Medin (2009) \pictext{Using NSClient++ from nagios}, \\ Quelle: \url{http://nsclient.org/nscp/wiki/doc/usage/nagios} (modifiziert)\newline Stand: 14.06.2009, Einsichtnahme: 12.07.2009
	 
	 	 \bibitem[ClubOra]{ClubOra} Unbekannter Author "`Sadik"' (2008) \pictext{Architecture of Oracle UCM}, \\ Quelle: \url{http://www.club-oracle.com/forums} \newline Stand: 25.10.2008, Einsichtnahme: 12.07.2009
	 	 
	 	 	 	 \bibitem[nuSOAP]{nuSOAP} Ahm Asaduzzaman (2003) \pictext{Building XML Web Services with PHP NuSOAP}, \\ Quelle: \url{http://www.devarticles.com/c/a/PHP/Building-XML-Web-Services-with-PHP-NuSOAP/1/} \newline Stand: 06.02.2003, Einsichtnahme: 12.08.2009

%2003-02-06

	% 10-25-2008
	 %http://nsclient.org/nscp/raw-attachment/wiki/doc/usage/nagios/nagios-active-nsclient.png 06/14/09:
\end{thebibliography}

%\section{Anhang} Michael Medin
%\section{\appendixname}
%\appendix


\end{document}

%makeindex bach.glo -s bach.ist -o bach.gls && svn add tex/* bilder/* ppt/* * && svn ci


