\section{Einleitung}
Mit dem Zusammenschluss des Forschungszentrum Karlsruhe und der Universität Karlsruhe (TH) zum Karlsruhe Insitute of Technology (KIT) wurden auch die beiden Rechenzentren im Steinbuch Center für Computing (SCC) vereinigt.

In Unternehmen werden den Benutzern verschiedene IT-Dienstleistungen angeboten.
Eine Dienstleistung ist die Bereitstellung einer Plattform für die zentrale Speicherung, Bearbeitung und Verwaltung von Dokumenten.
Dabei können diese Dokumente Dateien in unterschiedlicher Form sein wie Microsoft Word Dateien, Excel Tabellen, Dateien im Portable Document Format (\gls{PDF}) oder auch Bilder in vielen weiteren Formaten.

Die Darbietung / Die Versorgung / Das Angebot / Das Bereitstellen einer solchen Dienstleistung wird mit einem Dokumenten-Management-System (\gls{DMS}) realisiert.
Vorteile, die für den Einsatz eines Dokumenten-Management-Systems sprechen, sind die Möglichkeiten, die sich durch die computergestützte Erfassung und Indexierung (auch Indizierung genannt) der Dokumente eröffnen.
Die bekannteste / geläufigste Implementierung dieser Möglichkeiten ist die automatische Verschlagwortung von Dokumenten für die Zuordnung von Deskriptoren zu einem Dokument zur Erschließung der darin enthaltenen Sachverhalte.
Durch die Aufnahme dieser Schlagwörter in einen Suchindex können Anwender bestimmte Dokumenten gezielt finden und anfordern.
Ein wichtiger Punkt ist die Versionierung der Dateien in einem Dokumenten-Management-System.
Dadurch können auf ältere Versionen der Dokumente zugegriffen werden, Änderungen angezeigt oder (komplett) zurückgesetzt werden.
Durch die zentrale Struktur / Zugriff eines Dokumenten-Management-Systems ist es notwendig Zugriffsrichtlinien für die Dokumenten zu implementieren, die anhand von Gruppen- und Benutzerinformationen gesetzt / ausgewählt werden.

Aufgrund der Vielzahl an angebotenen Dienstleistungen ist es schwierig herauszufinden, ob die angebotenen Dienstleistungen noch fehlerfrei arbeiten oder aus welchem Grund die Benutzer nicht mehr auf einen Dienst zugreifen können.
Für diesen Zweck wurden Überwachungssysteme entwickelt die den Status der verschiedenen Komponenten und den davon abhängigen Diensten überwachen und bei Veränderungen die Verantwortlichen darüber informiert.

Für einen möglichst störungsfreien Betrieb ist es notwendig, dass die Ergebnisse der Überwachung in periodischen Zeitabständen erneuert werden, damit ein auftretendes Problem schnellstmöglich erkannt und behoben werden kann.
Das Überwachungssystem sollte so implementiert werden, dass Fehler erkannt werden, bevor die Nutzung der angebotenen Dienstleistungen davon beeinträchtigt werden.
Dabei muss die zusätzliche Belastung der Netzwerkes und der überwachten Objekte durch die Überwachung eingeplant, die verwendete Netzwerkstruktur und die dadurch entstehende Abhängigkeit (von Netzwerkknoten) beachtet und sicherheitstechnische Aspekte einer automatischen Überwachung bedacht werden.\\


Im Laufe dieser Arbeit soll eine Überwachung eines Dokumenten-Management-Systems unter Berücksichtigung der Funktions- und Arbeitsweise des eingesetzten Dokumenten-Management-Systems durch eine Open Source (Netzwerk)Überwachungsandwendung realisiert werden.
%popular open source computer system and network monitoring software application 
 

%Einleitung halt. Test
%Kurz was ist Nagios, warum überhaupt überwachen?
%Was soll überwacht werden -> Stellent/UCM kurz was ist das? Warum gerade das überwachen -> Aktive Benutzung durch User - kritisch