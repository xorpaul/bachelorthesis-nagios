\section{Aufgabenstellung}

Um den Mitarbeitern des Karlsruhe Insitute of Technology eine möglichst ausfallsichere Plattform für die zentrale Speicherung, Bearbeitung und Verwaltung von Dokumenten anbieten zu können, soll eine Überwachung implementiert werden.
Diese Überwachung soll nicht nur die Anwendung, sondern auch den darunterliegenden Server bezüglich seiner Systemressourcen berücksichtigen.
Dabei müssen Überwachungselemente gefunden werden, mit deren Überprüfung der eindeutige Zustand der Anwendung festgestellt und der störungsfreie Betrieb sichergestellt werden kann.

Für die Verwaltung von Webseiten, Dokumenten und Bildern wird das Dokumenten-Management-System \gls{OracleUCM}\footnote{Oracle Universal-Content-Management} der Firma Oracle eingesetzt.
Um die zu überwachenden Objekte zu ermitteln, ist das Verständnis über den Aufbau und der spezifischen Funktions- und Arbeitsweise des verwendeten Dokumenten-Management-Systems notwendig.

Als Überwachungssoftware wird die Open Source-Software Nagios verwendet.
%Damit der fehlerfreie Betrieb von \gls{OracleUCM} als Dienst durch die Überwachung der ermittelten Überwachungselemente eindeutig festgestellt werden kann, muss sich mit dem Aufbau, der internen Funktionsweise und den verschiedenen Methoden bezüglich der Ermittlung der Statusinformationen untersucht werden.
Zur Realisierung der Überwachung muss auf die interne Logik und auf die verschiedenen Methoden bezüglich der Ermittlung der Statusinformationen eingegangen werden.
Dabei soll eine Übersicht über die unterschiedlichen Überwachungsmethoden von Nagios erstellt und unter Berücksichtigung des späteren Einsatzes bewertet werden.
%Hierbei sind für die spätere Umsetzung beispielsweise die verschlüsselte Datenübertragung zwischen Überwachungs- und Anwendungsserver ein Kriterium.
Mit den ausgewählten Methoden soll die Überwachung auf verschiedenen Ebenen realisiert werden.

Die Klassifizierung der Überwachungselemente ergibt sich aus der Gewichtung der einzelnen Elemente.
%In diesem Schritt werden die zuvor gefunden 
%Die Erreichbarkeit über das Netzwerk bildet die Grundlage für darüber liegende Überwachungsobjekte, wie beispielsweise der Zustand eines Prozesses.
Dabei soll die Anwendung auch reaktiv durch eine Auswertung von Logdateien auf Fehler überwacht werden.


Zur eindeutigen Erkennung von Fehlern, die während der Benutzung durch Anwender auftreten, sollen die typischen Aktionen der Benutzer simuliert werden. 
Für die Realisierung dieser Benutzersimulation muss die Anwendung über eine Schnittstelle verfügen, die sich durch ein Programm über das Netzwerk ansprechen lässt.
Dieses Programm soll die Benutzeraktionen automatisiert durchführen und der Überwachungssoftware Nagios die Ergebnisse der einzelnen Schritte übermitteln, damit der Fehlerzustand sofort erkannt und gleichzeitig seine Ursache eingegrenzt werden kann.

Dabei müssen bei der Programmentwicklung mögliche Konsequenzen aufgrund verschiedener Szenarien bedacht werden.
Sollte die Anwendung bereits durch eine Vielzahl von Benutzern stark belastet sein, wird dadurch auch der Ablauf der Benutzersimulation verzögert.
%In diesem Fall soll die Überwachungssoftware bzw. Benutzersimulation keine falsche Informationen melden.
Eine solche Verzögerung soll von der Überwachungssoftware bzw. Benutzersimulation bei der Auswertung berücksichtigt werden.

Die Nutzung der Anwendung durch die eigentlichen Benutzer darf dabei nicht beeinträchtigt werden.
Da die Ausführung der Benutzersimulation durch Nagios in kurzen Zeitabständen periodisch aufgerufen wird, müssen auch langfristige Auswirkungen wie das Überlaufen der Datenbank der Anwendung oder die Überfüllung des Festplattenspeichers des Anwendungsservers bedacht werden.

Da als Entwicklungsumgebung ein eigener Nagios-Server eingesetzt werden soll, muss die entwickelte Lösung auf den bereits vorhanden Nagios-Server exportierbar sein.

%\textit{Export auf vorhanden Nagios-Server ermöglichen.

%Hinzufügen von Services ermöglichen%
%Konfigurieren der Überwachungsparameter ermöglichen (CPU last, df) protokollieren}