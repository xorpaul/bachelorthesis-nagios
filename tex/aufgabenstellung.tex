\section{Aufgabenstellung}

Um den Mitarbeitern des Forschungszentrums Karlsruhe eine möglichst ausfallsichere Plattform für die zentrale Speicherung, Bearbeitung und Verwaltung von Dokumenten anzubieten soll eine Elemente gefunden und der einwandfreie Status/Zustand dieser Objekte mit den passenden Überprüfungen sichergestellt werden 

reaktiv -> logs


Im Forschungszentrum Karlsruhe wird für die Verwaltung von Webseiten, Dokumenten und Bilder das Dokumenten-Management-System \textit{\gls{OracleUCM}} der Firma Oracle  eingesetzt.

Im Laufe dieser Arbeit soll eine Überwachung eines Dokumenten-Management-Systems unter Berücksichtigung der Funktions- und Arbeitsweise des eingesetzten Dokumenten-Management-Systems durch eine Open Source (Netzwerk)Überwachungsandwendung realisiert werden.

Sich mit Nagios und dem DMS Oracle UCM auseinandersetzten.

Einsatz von Oracle UCM als ECM für 

Überwachungselemente finden und der Überwachung realisieren.

Benutzersimulation -> damit Störung frühzeitig erkannt werden können,

Verwendung von Nagios als Vorraussetzung, da Standard
Keinen "Schaden" am DMS anrichten
Sichheitsaspekte nicht unbeachtet lassen, Klartextübertragung.

Export auf vorhanden Nagios-Server ermöglichen.

Hinzufügen von Services ermöglichen
Konfigurieren der Überwachungsparameter ermöglichen (CPU last, df)

Oracle UCM werden im FZK eingesetzt, bisher nur rudimentäre Überwachung durch Nagios möglich.
Diese Arbeit soll die spezifischen Überwachungselemente erurieren und umsetzten.