%Tabelle Windows-Agenten
%\vspace{1.6cm}
\begin{table}[!cht]
\centering
\begin{threeparttable}
\begin{tabular}{l p{1.3cm} l p{1.3cm} l p{1.3cm} l p{1.3cm} l p{1.3cm} l p{1.3cm} p{1.3cm} p{1.3cm} p{1.3cm} p{1.3cm}}
 & \begin{turn}{50}\textbf{NSClient}\end{turn} & \begin{turn}{50}\textbf{NRPE\_NT}\end{turn} & \begin{turn}{50}\textbf{NC\_net}\end{turn} & \begin{turn}{50}\textbf{NSClient++}\end{turn} & \begin{turn}{50}\textbf{OpMon Agent}\end{turn}\\ 
\hline
\textbf{Methode} & & & & & \\
\textit{aktiv} & \checkmark & \checkmark & \checkmark & \checkmark & \checkmark\\
\textit{passiv} & - & - & \checkmark & \checkmark & -\\
\textit{NSClient}\tnote{1} & \checkmark & - & \checkmark & \checkmark & \checkmark\\
\textit{NRPE}\tnote{2} & - & \checkmark & \checkmark & \checkmark & \checkmark\\
\textbf{Sicherheit} &  &  &  &  &  & \\
\textit{Passwort} & \checkmark & \checkmark & - & \checkmark & \checkmark\\
\textit{Accesslist}\tnote{3} & - & - & \checkmark & \checkmark & \checkmark\\
\textit{Verschlüsselung} & - & \checkmark & \checkmark & \checkmark & -\\
\textbf{Aufwand}\tnote{4} & \begin{footnotesize}normal\end{footnotesize} & \begin{footnotesize}hoch\end{footnotesize} & \begin{footnotesize}normal\end{footnotesize} & \begin{footnotesize}normal\end{footnotesize} & \begin{footnotesize}normal\end{footnotesize}\\
\end{tabular}
\begin{tablenotes}\footnotesize
		\item[1] Kompatibilität mit dem NSClient-Dienst
		\item[2] Erlaubt Ausführung von vorkonfigurierten Kommandos
        \item[3] Einschränkung der Abfrage der Überwachungsinformationen anhand der \gls{IP}-Adresse
        \item[4] Subjektive Einschätzung
    \end{tablenotes}
\caption{Übersicht der verschiedenen Windows-Agenten}
\label{tab:winagents}
\end{threeparttable}
\end{table}