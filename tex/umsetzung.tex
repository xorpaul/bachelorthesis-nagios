\section{Umsetzung}
In diesem Kapitel wird die Vorgehensweise der zuvor beschriebenen Problemstellungen erörtert.

\subsection{Überprüfen der Prozesse und Services}
\begin{itemize}
\item Prozesse
\item Services
\item Bsp Aufruf
\end{itemize}

\subsection{Einrichten des Windows Agenten}
\begin{itemize}
\item Port ändern -> RPC
\item Verschlüsselung durch PW und Algo
\item Hinzufügen der Plugins
\item Bsp Aufruf aktiver Check
\end{itemize}

.Net 2.0 Framework essentiell
NC\_Net installieren
nagios server ip zur sicherheit angeben
port ändern
pw hinzufügen
-> dienst starten

test vom nagios host:

\begin{lstlisting}[captionpos=b, caption=Aufruf eines aktiven Checks, label=activecheckexample, breaklines = true, language=bash]
root@iwrpaul:/usr/local/nagios/libexec# ./check_nc_net -H secret.kit.edu -p 123456 -s secret -v RUNSCRIPT -l check_uname.exe
Operating System OK - Microsoft(R) Windows(R) Server 2003 Standard Edition Service Pack 2
\end{lstlisting}

Das auf dem Nagios Server liegende Script \pictext{check\_nc\_net} stellt eine Verbindung zum angegebenen Server her und führt die mit dem Parameter \pictext{l} angegebene Datei aus. Dafür muss sich diese Datei in dem Script Verzeichnis des NC\_Net befinden.


Danach command definition hinzufügen, weil PW und Port verändert wurde:
\begin{lstlisting}[captionpos=b, caption=Nagios-Befehls Definition für den Host, label=activecheckexample, breaklines = true, language=bash]
# 'check_nt_bdb' command definition
#	_NSCLIENT_PORT	13599
#	_NSCLIENT_PW	KAnqloaQk
#
define command{
    command_name    check_nc_net_bdb
	command_line 	/usr/lib/nagios/plugins/check_nc_net -H $HOSTNAME$ -p 13599 -s KAnqloaQk -v $ARG1$
        }
\end{lstlisting}

Danach
\begin{itemize}
\item Logfiles check.exe 
\item batchloader.exe script
\end{itemize}